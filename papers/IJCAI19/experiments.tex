%!TEX root=./main.tex


% pgf settings: shrink the tick labels a bit
\pgfplotsset{every tick label/.append style={font=\scriptsize}}

\newcommand{\scatterplotsize}{8cm}
\newcommand{\scatterplotxlabelshift}{1.5ex}
\newcommand{\scatterplotylabelshift}{-3ex}

\section{Experiments}
\label{experiments}

We implemented our approach in Fast Downward (FD) \cite{helmert:jair-06}.  
\rebecca{Genral setting of experiments, timeout memory, processor ...}

\joerg{Rebecca/Michael: check out the IPC net-benefit benchmarks. Reviewers may naturally expect us to experiment with those, given our strong focus on oversubscription planning (actually this question came up in the discussion with the NASA guys yesterday). In the net-bnefit benchmarks, goal facts have rewards which we don't need. The question is whether, stripping away these rewards and imposing a plan-cost bound, we would get benchmarks not already covered by bour IPC experiments anyway. If the answer is "no", we can just say so in the paper. If the answer is "yes", it would be good (though probably not absolitely necessary) to experiment with these domains as well. In any case, we should know what the answer is.}

\subsubsection*{Oversubscription IPC Domains}

As a benchmark we used oversubscription versions of the optimal STRIPS IPC domains~\rebecca{cite} up to IPC'18. This includes all instances that are solvable with a 30 min timeout and 4GB of memory by $A^*$ with \emph{lmcut}~\cite{x}. All have less than 31 goal facts. This goal fact limitation is introduced by the duplicate checking of the sub-goals. We introduced a cost bound by scaling the optimal cost with $0.25$, $0.5$ and $0.75$.

The base planner configurations, used to solve/prove unsolvability of a meta search node,
are greedy best first search with $\hff$ and preferred operators ($hff$) and  
conjunction learning $\hc$ with $\hff$ as its base heuristic. \rebecca{ask Marcel how it is called}
\rebecca{Modification of hC to find deadends with an cost bound}

We compare the coverage against two baselines, the number of instances which are solved by \emph{lmcut} without a cost bound and the number of instances which are proved unsolvable by $\hc$ given a cost bound. The second baseline corresponds to the start node in the \emph{SysW} search.
The results are displayed in Table~\ref{table:coverage_ipc}.

The table provides some insight into how much harder it is to compute MUGS vs. solving the unbounded and cost-bounded problems. Table~\ref{table:coverage_ipc} shows cost bounds of 0.25/0.5/0.75 vs. the lmcut baseline results in 9/23/31 domains with lower coverage.
Proving unsolvability with the $\hc$ baseline for the same cost bounds has better coverage in 8/21/26 domains.

Comparing SysS and SysW, with cost bound 0.25 both show better coverage in 4 domains. Among those domains, in \woodworking\ and \openstacks\ SysS explores a much smaller fraction of the meta-search tree than SysW (0.02 vs. 0.99 and 0.06 vs. 0.99). With a cost bound of 0.5 SysW has better coverage in more domains (8 vs. 6). With cost 0.75 both show better coverage in 7 domains. Although SysW explores the smaller fraction of the meta-search tree, SysS still demonstrates better coverage overall. \rebecca{In this setting, finding a plan is easier than proving unsolvability?}

The table shows that $\hc$ is useful with SysW, but for SysS only with cost bound is $0.25$.

The average MUGS size for a cost bound of 0.25 is small (1.32). It is often the case that for these problems, you cannot reach any of the goal facts. In that case, the MUGS will be the goals.

\michael{Say something more about the \# MUGS, or remove the columns.}

%\begin{itemize}
%	\item conjunction learning is useful with every cost bound in SysW,
%		if you use SysS only for cost bound 0.25
%	\item for cost bound 0.25: SysS has a better overall coverage 
%		bestOfSysS vs bestOfSysW 4 vs 4
%		based on the large difference in the meta search tree coverage of openstacks and woodworking.
%	\item for cost bound 0.5: SysW slightly better overall coverage 
%		bestOfSysS vs bestOfSysW 8 vs 6
%	\item for cost bound 0.75: SysS  better overall coverage 
%		bestOfSysS vs bestOfSysW 7 vs 7
%		although the meta search space is larger for SysS, checking solvebility
%		works better than proving unsolvebility
%	\item best of 0.25/0.5/0.75 vs lmcut baseline: 9/23/31 solved less
%	\item best of 0.25/0.5/0.75 vs first SysW node baseline: 8/21/26 solved less
%	\item  For the cost bound $0.25$ SysS has clearly a smaller 
%		meta search space, you only have to add e few goals until the task is unsolvable. 
%	\item increase cost bound -> SysS meta search space get smaller and SysW gets larger
%		for cost bound 0.75 SysW has a smaller search space
%	\item \# MUGS reasonable small for explanation ?
%	\item \# MUGS increases and than decreases again why ?
%	\item \rebecca{not only the number of MUGS but also their size could be interesting}
%\end{itemize}

\setlength{\tabcolsep}{2pt}
\renewcommand{\arraystretch}{0.8}
\begin{figure*}[ht]
	\tiny
	\centering 	\tiny
	\begin{tabular}{l|rrr|rrr|rrr||rrr|rrr|rrr||rrr|rrr|rrr}
		& \multicolumn{9}{c||}{0.25} & \multicolumn{9}{c||}{0.5} & \multicolumn{9}{c}{0.75}\\
		& \multicolumn{3}{c|}{coverage} & \multicolumn{3}{c|}{avg time} & & & & \multicolumn{3}{c|}{covergae} & \multicolumn{3}{c|}{avg time} & & & & \multicolumn{3}{c|}{coverage} & \multicolumn{3}{c|}{avg time} & & &\\\hline
		& C & C nr & max & C & C nr & max & \#gs & \#n & fn & C & C nr & max & C & C nr & max & \#gs & \#n & fn & C & C nr & max & C & C nr & max & \#gs & \#n & fn\\\hline
		airport (28) & 0.93 & 0.93 & 0.86 & 0.0179 & 0.0296 & 15.5606 & 2.4 & 8.7 & 0.76 & 0.71 & 0.75 & 0.68 & 8.1089 & 3.8531 & 2.5639 & 1.9 & 4.4 & 0.71 & 0.57 & 0.57 & 0.68 & 3.4586 & 2.4734 & 0.256 & 1.0 & 2.4 & 0.61\\
		barman (4) & 1.00 & 1.00 & 1.00 & 0.003 & 0.0075 & 0.0382 & 3.0 & 7.0 & 0.88 & 1.00 & 1.00 & 1.00 & 42.121 & 26.4318 & 4.7071 & 3.0 & 7.0 & 0.88 & 0.00 & 0.00 & 1.00 & - & - & - & - & - & -\\
		blocks (28) & 1.00 & 0.93 & 0.96 & 0.0003 & 0.0006 & 0.0045 & 6.2 & 391.6 & 0.96 & 0.96 & 0.93 & 0.75 & 0.0006 & 0.0022 & 0.2815 & 6.6 & 138.7 & 0.93 & 0.89 & 0.86 & 0.61 & 0.0037 & 0.0072 & 0.7265 & 6.1 & 50.9 & 0.72\\
		data-net (12) & 0.00 & 0.00 & 1.00 & - & - & - & - & - & - & 0.00 & 0.00 & 1.00 & - & - & - & - & - & - & 0.00 & 0.00 & 1.00 & - & - & - & - & - & -\\
		depot (7) & 1.00 & 1.00 & 1.00 & 0.001 & 0.0027 & 0.065 & 4.0 & 36.1 & 0.94 & 1.00 & 1.00 & 1.00 & 1.2503 & 0.325 & 5.7004 & 7.0 & 34.6 & 0.91 & 0.43 & 0.43 & 0.57 & 4.9737 & 0.767 & 3.0968 & 2.7 & 11.0 & 0.68\\
		driverlog (13) & 1.00 & 1.00 & 1.00 & 0.0002 & 0.0007 & 0.0221 & 7.0 & 553.9 & 0.98 & 0.85 & 0.92 & 0.77 & 0.0062 & 0.0167 & 0.1872 & 13.4 & 387.4 & 0.86 & 0.69 & 0.69 & 0.62 & 2.4397 & 0.8843 & 4.7626 & 7.9 & 189.8 & 0.49\\
		elevators (40) & 0.00 & 0.00 & 1.00 &  &  &  &  &  &  & 0.00 & 0.00 & 1.00 & - & - & - & - & - & - & 0.00 & 0.00 & 0.83 & - & - & - & - & - & -\\
		floortile (13) & 0.54 & 0.46 & 0.46 & 0.0027 & 0.0077 & 0.0846 & 88.7 & 2881.7 & 0.99 & 0.15 & 0.15 & 0.15 & 0.4098 & 0.0973 & 0.3864 & 66.0 & 407.5 & 0.80 & 0.08 & 0.15 & 0.15 & 10.6455 & 2.4284 & 1.6908 & 30.0 & 142.0 & 0.28\\
		freecell (15) & 1.00 & 1.00 & 1.00 & 0.0007 & 0.004 & 0.1086 & 4.0 & 15.0 & 0.94 & 1.00 & 1.00 & 1.00 & 2.1944 & 0.3569 & 6.5571 & 4.7 & 15.0 & 0.94 & 0.87 & 0.87 & 0.93 & 3.5732 & 0.7317 & 24.2341 & 3.3 & 12.2 & 0.76\\
		ged (15) & 0.00 & 0.00 & 0.67 & - & - & - & - & - & - & 0.00 & 0.00 & 0.67 & - & - & - & - & - & - & 0.00 & 0.00 & 0.67 & - & - & - & - & - & -\\
		grid (2) & 1.00 & 1.00 & 1.00 & 0.0057 & 0.0068 & 0.0158 & 1.5 & 4.0 & 0.69 & 1.00 & 1.00 & 1.00 & 0.0131 & 0.0162 & 0.7963 & 1.5 & 4.0 & 0.69 & 1.00 & 1.00 & 1.00 & 0.1839 & 0.38 & 30.1363 & 1.0 & 3.0 & 0.56\\
		gripper (7) & 0.71 & 0.71 & 0.71 & 0.0036 & 0.0031 & 0.0089 & 77.4 & 1085.8 & 0.98 & 0.43 & 0.57 & 0.57 & 0.0547 & 0.0164 & 0.0076 & 32.0 & 97.0 & 0.89 & 0.43 & 0.43 & 0.71 & 0.7262 & 0.4556 & 0.0172 & 12.7 & 42.0 & 0.46\\
		hiking (9) & 1.00 & 1.00 & 1.00 & 0.0078 & 0.0079 & 0.1617 & 1.4 & 1.9 & 0.61 & 1.00 & 1.00 & 1.00 & 10.1549 & 4.5974 & 2.5455 & 1.4 & 1.9 & 0.61 & 1.00 & 1.00 & 1.00 & 173.4486 & 105.979 & 3.7597 & 1.0 & 1.9 & 0.61\\
		logistics (26) & 1.00 & 1.00 & 0.85 & 0.0006 & 0.0018 & 2.5104 & 4.6 & 108.1 & 0.95 & 0.77 & 0.85 & 0.58 & 0.0732 & 0.0309 & 2.7788 & 4.6 & 47.2 & 0.84 & 0.54 & 0.58 & 0.46 & 0.2428 & 0.0851 & 0.1558 & 2.2 & 20.5 & 0.63\\
		miconic (141) & 0.45 & 0.46 & 0.40 & 0.0022 & 0.0033 & 0.0261 & 27.9 & 436.3 & 0.91 & 0.28 & 0.35 & 0.32 & 0.0778 & 0.0119 & 0.0405 & 16.3 & 55.2 & 0.82 & 0.25 & 0.28 & 0.32 & 0.5958 & 0.0925 & 0.0495 & 5.5 & 18.7 & 0.61\\
		movie (30) & 1.00 & 1.00 & 1.00 & 0.0001 & 0.0001 & 0.0001 & 7.0 & 127.0 & 0.99 & 1.00 & 1.00 & 1.00 & 0.0008 & 0.0011 & 0.0003 & 35.0 & 120.0 & 0.94 & 1.00 & 1.00 & 1.00 & 0.0086 & 0.0114 & 0.0013 & 21.0 & 64.0 & 0.50\\
		mprime (22) & 1.00 & 1.00 & 1.00 & 0.0076 & 0.0081 & 0.0206 & 1.3 & 1.7 & 0.59 & 1.00 & 1.00 & 1.00 & 0.0079 & 0.009 & 0.354 & 1.2 & 1.7 & 0.59 & 1.00 & 1.00 & 1.00 & 0.0188 & 0.0235 & 13.4521 & 1.2 & 1.7 & 0.59\\
		mystery (17) & 1.00 & 1.00 & 1.00 & 0.0068 & 0.0083 & 0.039 & 1.4 & 2.2 & 0.63 & 1.00 & 1.00 & 1.00 & 0.0076 & 0.0111 & 1.7946 & 1.4 & 2.2 & 0.63 & 1.00 & 1.00 & 0.88 & 0.0085 & 0.0141 & 1.6148 & 1.2 & 2.2 & 0.63\\
		nomystery (14) & 1.00 & 1.00 & 1.00 & 0.0007 & 0.0032 & 0.062 & 7.3 & 144.1 & 0.96 & 0.86 & 0.86 & 0.71 & 0.1669 & 0.0289 & 1.9467 & 12.8 & 46.0 & 0.92 & 0.57 & 0.57 & 0.57 & 1.9144 & 0.363 & 2.3241 & 5.8 & 17.8 & 0.61\\
		openstacks (47) & 0.15 & 0.15 & 0.51 & 0.0011 & 0.0112 & 0.0709 & 6.4 & 314.4 & 0.98 & 0.11 & 0.11 & 0.47 & 0.0087 & 0.0179 & 0.0091 & 4.6 & 30.8 & 0.96 & 0.11 & 0.11 & 0.43 & 0.4592 & 0.4069 & 0.0198 & 5.2 & 29.2 & 0.91\\
		organic-syn (7) & 1.00 & 0.86 & 0.86 & 0.0004 & 0.0142 & 0.0577 & 4.0 & 38.3 & 0.93 & 1.00 & 0.86 & 0.86 & 0.0017 & 0.0193 & 0.0586 & 4.0 & 38.3 & 0.93 & 1.00 & 0.86 & 0.86 & 0.002 & 0.0224 & 0.0561 & 4.0 & 38.3 & 0.93\\
		organic-syn-s (10) & 0.80 & 0.60 & 0.60 & 0.0005 & 0.011 & 3.3391 & 4.0 & 62.0 & 0.95 & 0.80 & 0.50 & 0.60 & 0.0003 & 0.0059 & 0.0305 & 4.0 & 68.2 & 0.94 & 0.50 & 0.50 & 0.60 & 0.0334 & 0.6536 & 0.0416 & 4.0 & 66.6 & 0.89\\
		parcprinter (24) & 0.00 & 0.00 & 0.42 & - & - & - & - & - & - & 0.00 & 0.00 & 0.42 & - & - & - & - & - & - & 0.00 & 0.00 & 0.42 & - & - & - & - & - & -\\
		parking (5) & 1.00 & 0.00 & 0.00 & - & - & - & - & - & - & 0.00 & 0.00 & 0.00 & - & - & - & - & - & - & 0.00 & 0.00 & 0.00 & - & - & - & - & - & -\\
		pathways-n (5) & 1.00 & 1.00 & 1.00 & 0.0032 & 0.0034 & 0.8366 & 3.2 & 17.8 & 0.81 & 1.00 & 1.00 & 0.80 & 0.0039 & 0.0047 & 0.021 & 2.3 & 6.5 & 0.77 & 0.80 & 0.80 & 0.80 & 0.0101 & 0.0129 & 0.0675 & 1.8 & 6.0 & 0.70\\
		pegsol (2) & 0.00 & 0.00 & 0.00 & - & - & - & - & - & - & 0.00 & 0.00 & 0.00 & - & - & - & - & - & - & 0.00 & 0.00 & 0.00 & - & - & - & - & - & -\\
		pipesworld-nt (17) & 1.00 & 1.00 & 1.00 & 0.0013 & 0.0032 & 0.0284 & 3.7 & 38.7 & 0.89 & 1.00 & 1.00 & 0.94 & 0.639 & 0.3373 & 1.2171 & 4.8 & 23.5 & 0.84 & 0.82 & 0.82 & 0.94 & 8.308 & 8.7389 & 28.7631 & 4.1 & 17.4 & 0.66\\
		pipesworld-t (12) & 1.00 & 1.00 & 1.00 & 0.0017 & 0.0096 & 0.1815 & 3.6 & 34.7 & 0.94 & 0.92 & 0.92 & 0.92 & 0.1758 & 0.1562 & 6.1596 & 5.0 & 29.7 & 0.87 & 0.67 & 0.75 & 0.75 & 0.3663 & 0.3553 & 9.9122 & 3.1 & 14.0 & 0.63\\
		psr (49) & 1.00 & 0.98 & 0.98 & 0.0014 & 0.0016 & 0.0017 & 3.5 & 614.0 & 0.63 & 1.00 & 0.98 & 0.98 & 0.0016 & 0.0027 & 0.004 & 2.5 & 473.6 & 0.55 & 0.98 & 0.94 & 0.96 & 0.0059 & 0.0108 & 0.0836 & 1.8 & 80.0 & 0.47\\
		rovers (8) & 1.00 & 1.00 & 1.00 & 0.0054 & 0.002 & 0.045 & 18.0 & 161.4 & 0.93 & 0.88 & 0.88 & 0.88 & 0.4482 & 0.04 & 0.9443 & 11.4 & 33.0 & 0.84 & 0.63 & 0.88 & 0.75 & 10.5432 & 4.1379 & 2.2399 & 2.2 & 8.6 & 0.55\\
		satellite (7) & 1.00 & 1.00 & 1.00 & 0.0004 & 0.0014 & 0.1674 & 5.6 & 173.9 & 0.97 & 1.00 & 1.00 & 0.86 & 0.0199 & 0.0214 & 0.6192 & 18.7 & 112.5 & 0.94 & 0.71 & 0.86 & 0.57 & 0.1161 & 0.0451 & 0.0826 & 13.3 & 49.8 & 0.73\\
		scanalyzer (23) & 0.57 & 0.39 & 0.39 & 0.0001 & 0.0007 & 0.0061 & 12.9 & 2357.3 & 0.99 & 0.39 & 0.39 & 0.39 & 0.0348 & 0.0097 & 0.08 & 45.8 & 1937.8 & 0.86 & 0.22 & 0.22 & 0.39 & 0.1143 & 0.0165 & 0.0742 & 30.8 & 549.2 & 0.83\\
		snake (7) & 0.86 & 0.14 & 0.14 & 0.0006 & 0.014 & 0.244 & 6.0 & 244.0 & 0.95 & 0.43 & 0.14 & 0.14 & 0.0051 & 0.0301 & 0.2366 & 11.0 & 234.0 & 0.91 & 0.14 & 0.00 & 0.14 & - & - & - & - & - & -\\
		sokoban (50) & 0.00 & 0.00 & 0.98 & - & - & - & - & - & - & 0.00 & 0.00 & 0.94 & - & - & - & - & - & - & 0.00 & 0.00 & 0.84 & - & - & - & - & - & -\\
		storage (15) & 1.00 & 1.00 & 1.00 & 0.0032 & 0.0034 & 0.0047 & 3.6 & 11.4 & 0.81 & 1.00 & 1.00 & 1.00 & 0.1895 & 0.0394 & 0.0953 & 3.7 & 10.1 & 0.75 & 0.93 & 0.93 & 1.00 & 9.2279 & 2.4855 & 0.536 & 1.9 & 5.7 & 0.57\\
		termes (6) & 1.00 & 0.33 & 1.00 & 0.0005 & 0.0203 & 0.0193 & 2.5 & 2880.0 & 0.70 & 0.33 & 0.00 & 0.17 & - & - & - & - & - & - & 0.00 & 0.00 & 0.00 & - & - & - & - & - & -\\
		tetris (6) & 0.83 & 0.33 & 0.33 & 0.0002 & 0.0063 & 0.0146 & 6.5 & 255.0 & 1.00 & 0.50 & 0.33 & 0.33 & 0.0074 & 0.0152 & 0.0243 & 10.0 & 204.0 & 0.80 & 0.33 & 0.33 & 0.50 & 0.6283 & 0.2585 & 0.0736 & 5.5 & 104.5 & 0.41\\
		tidybot (23) & 1.00 & 1.00 & 1.00 & 0.0039 & 0.0462 & 1.7383 & 3.1 & 14.7 & 0.92 & 0.96 & 0.96 & 1.00 & 3.3049 & 2.1666 & 23.4994 & 3.1 & 14.7 & 0.92 & 0.52 & 0.35 & 0.30 & 7.0445 & 6.7422 & 8.1621 & 3.5 & 13.7 & 0.85\\
		tpp (7) & 1.00 & 1.00 & 1.00 & 0.0023 & 0.0027 & 0.0145 & 4.1 & 35.3 & 0.86 & 0.86 & 1.00 & 0.86 & 0.0271 & 0.0111 & 0.1495 & 6.2 & 19.3 & 0.83 & 0.71 & 0.86 & 0.86 & 0.023 & 0.014 & 0.0102 & 2.8 & 7.6 & 0.66\\
		transport (23) & 1.00 & 1.00 & 1.00 & 0.0011 & 0.0015 & 0.0103 & 3.3 & 15.5 & 0.91 & 1.00 & 1.00 & 1.00 & 0.0429 & 0.0536 & 0.5756 & 3.4 & 14.8 & 0.88 & 0.96 & 0.96 & 1.00 & 5.0431 & 2.1916 & 8.1134 & 2.2 & 10.6 & 0.68\\
		trucks (10) & 1.00 & 1.00 & 1.00 & 0.0042 & 0.0146 & 0.4801 & 15.7 & 145.4 & 0.97 & 0.70 & 0.90 & 0.60 & 0.9799 & 0.0891 & 1.225 & 14.8 & 46.5 & 0.89 & 0.30 & 0.40 & 0.60 & 1.5139 & 0.3479 & 0.0774 & 3.7 & 11.0 & 0.65\\
		visitall (14) & 0.71 & 0.57 & 0.64 & 0.0016 & 0.0016 & 0.0017 & 20.1 & 10289.6 & 0.91 & 0.71 & 0.50 & 0.57 & 0.0019 & 0.0023 & 0.0041 & 38.0 & 6928.6 & 0.89 & 0.57 & 0.50 & 0.50 & 0.0044 & 0.0054 & 0.0189 & 38.3 & 4532.0 & 0.74\\
		woodworking (29) & 0.52 & 0.17 & 0.24 & 0.0001 & 0.0004 & 0.0014 & 20.0 & 2147.0 & 1.00 & 0.31 & 0.17 & 0.17 & 0.0011 & 0.0032 & 0.0078 & 33.8 & 1975.0 & 0.93 & 0.17 & 0.17 & 0.17 & 0.0144 & 0.0341 & 0.0485 & 16.8 & 1087.4 & 0.52\\
		zenotravel (13) & 1.00 & 1.00 & 0.92 & 0.0008 & 0.0034 & 0.1252 & 8.3 & 120.8 & 0.94 & 0.69 & 0.77 & 0.62 & 0.0011 & 0.0026 & 0.0228 & 3.8 & 36.6 & 0.89 & 0.69 & 0.69 & 0.62 & 0.4387 & 0.1626 & 0.8227 & 2.4 & 25.3 & 0.66\\\hline
		Sum (862) & 0.65 & 0.60 & 0.76 & 0.0026 & 0.0072 & 0.7059 & 10.9 & 696.7 & 0.88 & 0.55 & 0.55 & 0.68 & 1.9595 & 1.0787 & 1.8231 & 12.2 & 378.0 & 0.84 & 0.46 & 0.46 & 0.62 & 7.2394 & 4.157 & 4.2789 & 7.3 & 212.9 & 0.64\\\hline\hline
		nomystery (13, 25, 25) & 14 & 14 & 14 & 0.0004 & 0.0021 & 0.0262 & 8.6 & 507 & - & 21 & 22 & 2 & 0.0015 & 0.0132 & 0.2042 & 14 & 511 & - & 4 & 0 & 0 & - & - & - & - & - & - \\ 
		rovers (10, 25, 25) & 12 & 0 & 0 & - &  - & - & - & - & - & 8 & 0 & 0 &  - & - & - & - & - & - & 0 & 0 & 0 & - & - & - & - & - & - \\
		tpp (5) & 5 & 5 & 5 & 0.0279 & 0.2685 & 0.4789 & - & - & - &  1 & 0 & 0 & - & - & - & - & - & - & 0 & 0 & 0 & - & - & - & - & - & - \\

	\end{tabular}

 
	\caption{
		Benchmark: oversubscription IPC
		domains with bound $ = x \cdot $ optimal cost with $
		x \in \{0.25, 0.5, 0.75\}$, solvable with lmcut with no cost bound in 
		30 min and with less then 31 goal facts(limitation of implementation).
	}
	\label{table:coverage_ipc}
\end{figure*}



\subsubsection*{Action Set Properties}

\michael{Experimental setup, mention STRIPS versions, etc.}

To benchmark the action set properties we compile the properties into goal facts as described above and perform the meta-search over these goals. The original goals of the problems remain as ``hard goals'' that must be achieved.
We introduce a cost bound by scaling the optimal cost for solving only the hard goals by 1, 1.5 and 2.
We included four IPC domains.
\begin{enumerate}
\item The resource constrained \textit{rovers} domain. Problems were generated with 2 rovers, 5 waypoints. Action properties are to use a specific rover for a sample or an observation, or to use a specific camera for an observation. 
\item The \textit{blocksworld} domain with 2 grippers, modified such that picking up or unstacking a block costs high or low energy depending upon which gripper is used. Problems were generated scaling from 3 to 10 blocks. Action properties are to use a specific gripper to pick up a specific block, or to use any gripper to stack a specific pair of blocks at any point in the plan.
\item The resource constrained \textit{TPP} domain. Problems were generated with 5 markets and 1 depot. Properties are to use or not use particular road segments, and preferred markets for goods.
\item The resource constrained \textit{nomystery} domain, described in the example. Problems were generated with 6 locations and 2 trucks.
\end{enumerate}
For each domain, we vary the number of action set properties between 1 and 10 and the number of hard goals from 4 to 7. For each property and goal count combination, we generated 10 variants.

As a baseline we compare against the coverage of lmcut solving both the hard goals and properties with no cost bound. Figure~\ref{fig:barcharts} displays a cross-section of the results. We chose the goal count for which the scaling behaviour can most eaily be seen.

Each row show the coverage results for one domain and goal count, with cost bounds 1, 1.5, and 2 times the optimal cost of the hard goals only. For each property count we show results for SysS and SysW, with \hff and trap-learning with \hff as its base heuristic. The solid black line shows the coverage of the lmcut baseline. The dotted black line shows the coverage for solving the start node of meta-search tree with the trap-learning.

\begin{figure*}[ht]

\tiny
\begin{tikzpicture}


\begin{axis}[
width = 7cm,
height=4cm,
enlarge x limits = 0.1,
enlarge y limits = 0.1,
ybar,
bar width=1pt,
ymin = 0,
ymax = 10,
at={(0.0\linewidth,-0.0)},
compat=1.6,
ylabel=\small \#goals 4,
]
\addplot+[ybar, bar shift =-2.5pt, red,
]
plot coordinates {
(07, 10)
(05, 10)
(04, 10)
(08, 10)
(01, 10)
(09, 10)
(03, 10)
(06, 10)
(10, 9)
(02, 10)
};
\label{plot:properties_hff_bu_39}
\addplot+[ybar, bar shift =-1.0pt, blue,
]
plot coordinates {
(07, 10)
(05, 10)
(04, 10)
(08, 10)
(09, 9)
(03, 10)
(06, 10)
(01, 10)
(10, 9)
(02, 10)
};
\label{plot:properties_hff_td_39}
\addplot+[ybar, bar shift =1.0pt, darkgreen,
]
plot coordinates {
(07, 10)
(04, 10)
(08, 10)
(09, 10)
(06, 10)
(03, 10)
(05, 10)
(01, 10)
(02, 10)
(10, 10)
};
\label{plot:properties_trap_prefop_bu_39}
\addplot+[ybar, bar shift =2.5pt, orange,
]
plot coordinates {
(07, 10)
(05, 10)
(08, 10)
(09, 10)
(06, 10)
(04, 10)
(03, 10)
(01, 10)
(02, 10)
(10, 10)
};
\label{plot:properties_trap_prefop_td_39}
\addplot+[only marks, mark = -, mark options = {thick}, mark size = 0.2cm, black,
]
plot coordinates {
(07, 10)
(05, 10)
(08, 10)
(09, 10)
(06, 10)
(04, 10)
(03, 10)
(01, 10)
(02, 10)
(10, 10)
};
\addplot+[only marks, mark = -, mark options = {thick, red, dashed}, mark size = 0.2cm, black,
]
plot coordinates {
(02, 10)
(01, 10)
(04, 10)
(03, 10)
(07, 10)
(08, 10)
(06, 10)
(10, 10)
(05, 10)
(09, 10)
};

\end{axis}
\hfill


\begin{axis}[
width = 7cm,
height=4cm,
enlarge x limits = 0.1,
enlarge y limits = 0.1,
ybar,
bar width=1pt,
ymin = 0,
ymax = 10,
at={(0.333333333333\linewidth,-0.0)},
compat=1.6,
]
\addplot+[ybar, bar shift =-2.5pt, red,
]
plot coordinates {
(05, 9)
(04, 10)
(03, 10)
(10, 1)
(02, 10)
(06, 8)
(08, 2)
(07, 5)
(09, 1)
(01, 10)
};
\label{plot:properties_hff_bu_39}
\addplot+[ybar, bar shift =-1.0pt, blue,
]
plot coordinates {
(05, 9)
(04, 10)
(03, 10)
(10, 0)
(02, 10)
(08, 2)
(07, 3)
(06, 7)
(01, 10)
(09, 0)
};
\label{plot:properties_hff_td_39}
\addplot+[ybar, bar shift =1.0pt, darkgreen,
]
plot coordinates {
(05, 8)
(04, 9)
(03, 10)
(10, 2)
(02, 10)
(08, 3)
(07, 5)
(09, 2)
(06, 6)
(01, 10)
};
\label{plot:properties_trap_prefop_bu_39}
\addplot+[ybar, bar shift =2.5pt, orange,
]
plot coordinates {
(05, 10)
(04, 10)
(03, 10)
(10, 2)
(02, 10)
(08, 4)
(07, 6)
(06, 9)
(01, 10)
(09, 3)
};
\label{plot:properties_trap_prefop_td_39}
\addplot+[only marks, mark = -, mark options = {thick}, mark size = 0.2cm, black,
]
plot coordinates {
(05, 10)
(04, 10)
(03, 10)
(10, 10)
(02, 10)
(08, 10)
(07, 10)
(06, 10)
(01, 10)
(09, 10)
};
\addplot+[only marks, mark = -, mark options = {thick, red, dashed}, mark size = 0.2cm, black,
]
plot coordinates {
(02, 10)
(01, 10)
(04, 10)
(03, 10)
(07, 10)
(08, 10)
(06, 10)
(10, 10)
(05, 10)
(09, 10)
};

\end{axis}
\hfill


\begin{axis}[
width = 7cm,
height=4cm,
enlarge x limits = 0.1,
enlarge y limits = 0.1,
ybar,
bar width=1pt,
ymin = 0,
ymax = 10,
at={(0.666666666667\linewidth,-0.0)},
compat=1.6,
]
\addplot+[ybar, bar shift =-2.5pt, red,
]
plot coordinates {
(08, 1)
(09, 0)
(01, 10)
(03, 10)
(04, 10)
(07, 2)
(02, 10)
(06, 6)
(05, 10)
(10, 0)
};
\label{plot:properties_hff_bu_39}
\addplot+[ybar, bar shift =-1.0pt, blue,
]
plot coordinates {
(08, 6)
(06, 9)
(04, 10)
(09, 5)
(01, 10)
(03, 10)
(07, 8)
(02, 10)
(05, 10)
(10, 2)
};
\label{plot:properties_hff_td_39}
\addplot+[ybar, bar shift =1.0pt, darkgreen,
]
plot coordinates {
(08, 2)
(02, 10)
(06, 5)
(04, 10)
(09, 0)
(03, 10)
(07, 3)
(01, 10)
(10, 0)
(05, 9)
};
\label{plot:properties_trap_prefop_bu_39}
\addplot+[ybar, bar shift =2.5pt, orange,
]
plot coordinates {
(08, 9)
(03, 10)
(06, 10)
(04, 10)
(09, 8)
(07, 8)
(02, 10)
(01, 10)
(10, 5)
(05, 10)
};
\label{plot:properties_trap_prefop_td_39}
\addplot+[only marks, mark = -, mark options = {thick}, mark size = 0.2cm, black,
]
plot coordinates {
(08, 9)
(03, 10)
(06, 10)
(04, 10)
(09, 8)
(07, 9)
(02, 10)
(01, 10)
(10, 7)
(05, 10)
};
\addplot+[only marks, mark = -, mark options = {thick, red, dashed}, mark size = 0.2cm, black,
]
plot coordinates {
(02, 10)
(01, 10)
(04, 10)
(03, 10)
(07, 10)
(08, 10)
(06, 10)
(10, 10)
(05, 10)
(09, 10)
};

\end{axis}
\hfill


\begin{axis}[
width = 7cm,
height=4cm,
enlarge x limits = 0.1,
enlarge y limits = 0.1,
ybar,
bar width=1pt,
ymin = 0,
ymax = 10,
at={(0.0\linewidth,-160.0)},
compat=1.6,
ylabel=\small \#goals 5,
]
\addplot+[ybar, bar shift =-2.5pt, red,
]
plot coordinates {
(01, 10)
(08, 9)
(03, 10)
(02, 10)
(09, 9)
(05, 10)
(04, 10)
(07, 9)
(06, 10)
(10, 9)
};
\label{plot:properties_hff_bu_39}
\addplot+[ybar, bar shift =-1.0pt, blue,
]
plot coordinates {
(01, 10)
(08, 6)
(03, 10)
(02, 10)
(09, 5)
(05, 10)
(04, 10)
(07, 7)
(06, 9)
(10, 2)
};
\label{plot:properties_hff_td_39}
\addplot+[ybar, bar shift =1.0pt, darkgreen,
]
plot coordinates {
(01, 10)
(08, 10)
(03, 10)
(02, 10)
(09, 10)
(05, 10)
(04, 10)
(07, 10)
(06, 10)
(10, 10)
};
\label{plot:properties_trap_prefop_bu_39}
\addplot+[ybar, bar shift =2.5pt, orange,
]
plot coordinates {
(01, 10)
(08, 10)
(03, 10)
(02, 10)
(09, 10)
(05, 10)
(04, 10)
(07, 10)
(06, 10)
(10, 10)
};
\label{plot:properties_trap_prefop_td_39}
\addplot+[only marks, mark = -, mark options = {thick}, mark size = 0.2cm, black,
]
plot coordinates {
(01, 10)
(08, 10)
(03, 10)
(02, 10)
(09, 10)
(05, 10)
(04, 10)
(07, 10)
(06, 10)
(10, 10)
};
\addplot+[only marks, mark = -, mark options = {thick, red, dashed}, mark size = 0.2cm, black,
]
plot coordinates {
(02, 10)
(01, 10)
(04, 10)
(03, 10)
(07, 10)
(08, 10)
(06, 10)
(10, 10)
(05, 10)
(09, 10)
};

\end{axis}
\hfill


\begin{axis}[
width = 7cm,
height=4cm,
enlarge x limits = 0.1,
enlarge y limits = 0.1,
ybar,
bar width=1pt,
ymin = 0,
ymax = 10,
at={(0.333333333333\linewidth,-160.0)},
compat=1.6,
]
\addplot+[ybar, bar shift =-2.5pt, red,
]
plot coordinates {
(05, 5)
(06, 2)
(10, 0)
(03, 10)
(04, 9)
(01, 10)
(09, 0)
(02, 10)
(07, 1)
(08, 1)
};
\label{plot:properties_hff_bu_39}
\addplot+[ybar, bar shift =-1.0pt, blue,
]
plot coordinates {
(05, 7)
(06, 4)
(10, 1)
(03, 10)
(04, 9)
(01, 10)
(09, 1)
(02, 10)
(07, 1)
(08, 1)
};
\label{plot:properties_hff_td_39}
\addplot+[ybar, bar shift =1.0pt, darkgreen,
]
plot coordinates {
(05, 6)
(06, 2)
(10, 1)
(03, 10)
(04, 10)
(01, 10)
(09, 1)
(02, 10)
(07, 1)
(08, 1)
};
\label{plot:properties_trap_prefop_bu_39}
\addplot+[ybar, bar shift =2.5pt, orange,
]
plot coordinates {
(05, 9)
(06, 6)
(10, 2)
(03, 10)
(04, 10)
(01, 10)
(09, 2)
(02, 10)
(07, 4)
(08, 3)
};
\label{plot:properties_trap_prefop_td_39}
\addplot+[only marks, mark = -, mark options = {thick}, mark size = 0.2cm, black,
]
plot coordinates {
(05, 10)
(06, 9)
(10, 9)
(03, 10)
(04, 10)
(01, 10)
(09, 9)
(02, 10)
(07, 9)
(08, 9)
};
\addplot+[only marks, mark = -, mark options = {thick, red, dashed}, mark size = 0.2cm, black,
]
plot coordinates {
(02, 10)
(01, 10)
(04, 10)
(03, 10)
(07, 10)
(08, 10)
(06, 10)
(10, 10)
(05, 10)
(09, 10)
};

\end{axis}
\hfill


\begin{axis}[
width = 7cm,
height=4cm,
enlarge x limits = 0.1,
enlarge y limits = 0.1,
ybar,
bar width=1pt,
ymin = 0,
ymax = 10,
at={(0.666666666667\linewidth,-160.0)},
compat=1.6,
]
\addplot+[ybar, bar shift =-2.5pt, red,
]
plot coordinates {
(03, 10)
(04, 7)
(05, 3)
(09, 0)
(01, 10)
(02, 10)
(08, 0)
(06, 1)
(10, 0)
(07, 0)
};
\label{plot:properties_hff_bu_39}
\addplot+[ybar, bar shift =-1.0pt, blue,
]
plot coordinates {
(03, 10)
(04, 8)
(05, 7)
(09, 5)
(01, 10)
(02, 10)
(08, 5)
(06, 7)
(10, 4)
(07, 6)
};
\label{plot:properties_hff_td_39}
\addplot+[ybar, bar shift =1.0pt, darkgreen,
]
plot coordinates {
(03, 9)
(04, 6)
(05, 1)
(09, 0)
(01, 10)
(02, 10)
(08, 0)
(06, 1)
(10, 0)
(07, 0)
};
\label{plot:properties_trap_prefop_bu_39}
\addplot+[ybar, bar shift =2.5pt, orange,
]
plot coordinates {
(03, 10)
(04, 10)
(05, 8)
(09, 5)
(01, 10)
(02, 10)
(08, 6)
(06, 7)
(10, 4)
(07, 8)
};
\label{plot:properties_trap_prefop_td_39}
\addplot+[only marks, mark = -, mark options = {thick}, mark size = 0.2cm, black,
]
plot coordinates {
(03, 10)
(04, 10)
(05, 9)
(09, 8)
(01, 10)
(02, 10)
(08, 8)
(06, 8)
(10, 8)
(07, 8)
};
\addplot+[only marks, mark = -, mark options = {thick, red, dashed}, mark size = 0.2cm, black,
]
plot coordinates {
(02, 10)
(01, 10)
(04, 10)
(03, 10)
(07, 10)
(08, 10)
(06, 10)
(10, 10)
(05, 10)
(09, 10)
};

\end{axis}
\hfill


\begin{axis}[
width = 7cm,
height=4cm,
enlarge x limits = 0.1,
enlarge y limits = 0.1,
ybar,
bar width=1pt,
ymin = 0,
ymax = 10,
at={(0.0\linewidth,-320.0)},
compat=1.6,
ylabel=\small \#goals 6,
]
\addplot+[ybar, bar shift =-2.5pt, red,
]
plot coordinates {
(08, 2)
(09, 2)
(01, 10)
(10, 2)
(06, 5)
(07, 4)
(04, 8)
(05, 5)
(02, 9)
(03, 8)
};
\label{plot:properties_hff_bu_39}
\addplot+[ybar, bar shift =-1.0pt, blue,
]
plot coordinates {
(08, 2)
(09, 2)
(01, 10)
(10, 2)
(06, 2)
(07, 2)
(04, 7)
(05, 4)
(02, 9)
(03, 8)
};
\label{plot:properties_hff_td_39}
\addplot+[ybar, bar shift =1.0pt, darkgreen,
]
plot coordinates {
(08, 8)
(09, 8)
(01, 10)
(10, 8)
(06, 9)
(07, 8)
(04, 10)
(05, 9)
(02, 10)
(03, 10)
};
\label{plot:properties_trap_prefop_bu_39}
\addplot+[ybar, bar shift =2.5pt, orange,
]
plot coordinates {
(08, 8)
(09, 8)
(01, 10)
(10, 7)
(06, 7)
(07, 8)
(04, 8)
(05, 7)
(02, 10)
(03, 8)
};
\label{plot:properties_trap_prefop_td_39}
\addplot+[only marks, mark = -, mark options = {thick}, mark size = 0.2cm, black,
]
plot coordinates {
(08, 10)
(09, 10)
(01, 10)
(10, 10)
(06, 10)
(07, 10)
(04, 10)
(05, 10)
(02, 10)
(03, 10)
};
\addplot+[only marks, mark = -, mark options = {thick, red, dashed}, mark size = 0.2cm, black,
]
plot coordinates {
(02, 10)
(01, 10)
(04, 10)
(03, 10)
(07, 10)
(08, 10)
(06, 10)
(10, 10)
(05, 10)
(09, 10)
};

\end{axis}
\hfill


\begin{axis}[
width = 7cm,
height=4cm,
enlarge x limits = 0.1,
enlarge y limits = 0.1,
ybar,
bar width=1pt,
ymin = 0,
ymax = 10,
at={(0.333333333333\linewidth,-320.0)},
compat=1.6,
]
\addplot+[ybar, bar shift =-2.5pt, red,
]
plot coordinates {
(06, 1)
(10, 0)
(05, 3)
(08, 1)
(07, 1)
(02, 10)
(09, 0)
(01, 10)
(04, 7)
(03, 9)
};
\label{plot:properties_hff_bu_39}
\addplot+[ybar, bar shift =-1.0pt, blue,
]
plot coordinates {
(06, 2)
(10, 0)
(05, 4)
(08, 0)
(07, 0)
(02, 10)
(09, 0)
(01, 10)
(04, 7)
(03, 10)
};
\label{plot:properties_hff_td_39}
\addplot+[ybar, bar shift =1.0pt, darkgreen,
]
plot coordinates {
(06, 2)
(10, 2)
(05, 2)
(08, 2)
(07, 2)
(02, 10)
(09, 2)
(01, 10)
(04, 4)
(03, 7)
};
\label{plot:properties_trap_prefop_bu_39}
\addplot+[ybar, bar shift =2.5pt, orange,
]
plot coordinates {
(06, 3)
(10, 3)
(05, 4)
(08, 3)
(07, 3)
(02, 10)
(09, 3)
(01, 10)
(04, 6)
(03, 8)
};
\label{plot:properties_trap_prefop_td_39}
\addplot+[only marks, mark = -, mark options = {thick}, mark size = 0.2cm, black,
]
plot coordinates {
(06, 7)
(10, 7)
(05, 7)
(08, 7)
(07, 7)
(02, 10)
(09, 5)
(01, 10)
(04, 7)
(03, 8)
};
\addplot+[only marks, mark = -, mark options = {thick, red, dashed}, mark size = 0.2cm, black,
]
plot coordinates {
(02, 10)
(01, 10)
(04, 10)
(03, 10)
(07, 10)
(08, 10)
(06, 10)
(10, 10)
(05, 10)
(09, 10)
};

\end{axis}
\hfill


\begin{axis}[
width = 7cm,
height=4cm,
enlarge x limits = 0.1,
enlarge y limits = 0.1,
ybar,
bar width=1pt,
ymin = 0,
ymax = 10,
at={(0.666666666667\linewidth,-320.0)},
compat=1.6,
]
\addplot+[ybar, bar shift =-2.5pt, red,
]
plot coordinates {
(02, 8)
(01, 10)
(04, 3)
(03, 4)
(06, 1)
(10, 0)
(05, 3)
(09, 0)
(08, 0)
(07, 0)
};
\label{plot:properties_hff_bu_39}
\addplot+[ybar, bar shift =-1.0pt, blue,
]
plot coordinates {
(02, 9)
(01, 10)
(04, 6)
(03, 7)
(07, 4)
(08, 3)
(06, 4)
(10, 1)
(05, 6)
(09, 1)
};
\label{plot:properties_hff_td_39}
\addplot+[ybar, bar shift =1.0pt, darkgreen,
]
plot coordinates {
(02, 10)
(01, 10)
(04, 4)
(03, 5)
(08, 0)
(06, 1)
(10, 0)
(05, 2)
(09, 0)
(07, 0)
};
\label{plot:properties_trap_prefop_bu_39}
\addplot+[ybar, bar shift =2.5pt, orange,
]
plot coordinates {
(02, 10)
(01, 10)
(04, 7)
(03, 8)
(07, 5)
(08, 4)
(06, 7)
(10, 3)
(05, 7)
(09, 3)
};
\label{plot:properties_trap_prefop_td_39}
\addplot+[only marks, mark = -, mark options = {thick}, mark size = 0.2cm, black,
]
plot coordinates {
(02, 10)
(01, 10)
(04, 7)
(03, 8)
(07, 7)
(08, 6)
(06, 7)
(10, 5)
(05, 7)
(09, 5)
};
\label{plot:baseline_sysW_node}
\addplot+[only marks, mark = -, mark options = {thick, red, dashed}, mark size = 0.2cm, black,
]
plot coordinates {
(02, 10)
(01, 10)
(04, 10)
(03, 10)
(07, 10)
(08, 10)
(06, 10)
(10, 10)
(05, 10)
(09, 10)
};
\label{plot:baseline_lmcut}

\end{axis}
\hfill

\node[] (c1) at (2.5,2.8) {\small $x=1.0$};
\node[] (c15) at (8.4,2.8) {\small $x=1.5$};
\node[] (c2) at (14.3,2.8) {\small $x=2.0$};

\node[draw] (test) at (8,-7.5) {
\ref{plot:properties_hff_bu_39}  SysS
\ref{plot:properties_hff_td_39}  SysW
\ref{plot:properties_trap_prefop_bu_39} SysS trap 
\ref{plot:properties_trap_prefop_td_39} SysW trap 
\ref{plot:baseline_lmcut} \hlmcut
\ref{plot:baseline_sysW_node} trap
};

\end{tikzpicture}
\hfill

\caption{nomystery: reference coverage first node in top-down meta search tree;
	optimal coverage always 10}
\end{figure*}

\begin{figure*}[ht]

\tiny
\begin{tikzpicture}


\begin{axis}[
width = 7cm,
height=4cm,
enlarge x limits = 0.1,
enlarge y limits = 0.1,
ybar,
bar width=1pt,
ymin = 0,
ymax = 10,
at={(0.0\linewidth,-0.0)},
compat=1.6,
title=c 100,
ylabel=goals 4,
]
\addplot+[ybar, bar shift =-2.5pt, red,
]
plot coordinates {
(04, 10)
(03, 10)
(02, 10)
(09, 10)
(01, 10)
(08, 10)
(07, 10)
(10, 10)
(06, 10)
(05, 10)
};
\label{plot:properties_hff_bu_47}
\addplot+[ybar, bar shift =-1.0pt, blue,
]
plot coordinates {
(04, 10)
(03, 10)
(02, 10)
(09, 10)
(01, 10)
(08, 10)
(07, 10)
(10, 10)
(06, 10)
(05, 10)
};
\label{plot:properties_hff_td_47}
\addplot+[ybar, bar shift =1.0pt, darkgreen,
]
plot coordinates {
(04, 10)
(03, 10)
(02, 10)
(09, 10)
(01, 10)
(08, 10)
(07, 10)
(10, 10)
(06, 10)
(05, 10)
};
\label{plot:properties_trap_prefop_bu_47}
\addplot+[ybar, bar shift =2.5pt, orange,
]
plot coordinates {
(04, 10)
(03, 10)
(02, 10)
(09, 10)
(01, 10)
(08, 10)
(07, 10)
(10, 10)
(06, 10)
(05, 10)
};
\label{plot:properties_trap_prefop_td_47}
\addplot+[only marks, mark = -, mark options = {thick}, mark size = 0.2cm, black,
]
plot coordinates {
(04, 10)
(03, 10)
(02, 10)
(09, 10)
(01, 10)
(08, 10)
(07, 10)
(10, 10)
(06, 10)
(05, 10)
};

\end{axis}
\hfill


\begin{axis}[
width = 7cm,
height=4cm,
enlarge x limits = 0.1,
enlarge y limits = 0.1,
ybar,
bar width=1pt,
ymin = 0,
ymax = 10,
at={(0.333333333333\linewidth,-0.0)},
compat=1.6,
title=c 150,
]
\addplot+[ybar, bar shift =-2.5pt, red,
]
plot coordinates {
(07, 10)
(08, 10)
(09, 10)
(01, 10)
(02, 10)
(03, 10)
(04, 10)
(05, 10)
(06, 10)
(10, 10)
};
\label{plot:properties_hff_bu_47}
\addplot+[ybar, bar shift =-1.0pt, blue,
]
plot coordinates {
(07, 10)
(08, 10)
(09, 8)
(01, 10)
(02, 10)
(03, 10)
(04, 10)
(05, 10)
(06, 10)
(10, 4)
};
\label{plot:properties_hff_td_47}
\addplot+[ybar, bar shift =1.0pt, darkgreen,
]
plot coordinates {
(07, 10)
(08, 9)
(09, 9)
(01, 10)
(02, 10)
(03, 10)
(04, 10)
(05, 10)
(06, 10)
(10, 9)
};
\label{plot:properties_trap_prefop_bu_47}
\addplot+[ybar, bar shift =2.5pt, orange,
]
plot coordinates {
(07, 10)
(08, 10)
(09, 10)
(01, 10)
(02, 10)
(03, 10)
(04, 10)
(05, 10)
(06, 10)
(10, 10)
};
\label{plot:properties_trap_prefop_td_47}
\addplot+[only marks, mark = -, mark options = {thick}, mark size = 0.2cm, black,
]
plot coordinates {
(07, 10)
(08, 10)
(09, 10)
(01, 10)
(02, 10)
(03, 10)
(04, 10)
(05, 10)
(06, 10)
(10, 10)
};

\end{axis}
\hfill


\begin{axis}[
width = 7cm,
height=4cm,
enlarge x limits = 0.1,
enlarge y limits = 0.1,
ybar,
bar width=1pt,
ymin = 0,
ymax = 10,
at={(0.666666666667\linewidth,-0.0)},
compat=1.6,
title=c 200,
]
\addplot+[ybar, bar shift =-2.5pt, red,
]
plot coordinates {
(08, 10)
(03, 10)
(04, 10)
(05, 10)
(06, 10)
(10, 9)
(07, 10)
(01, 10)
(09, 10)
(02, 10)
};
\label{plot:properties_hff_bu_47}
\addplot+[ybar, bar shift =-1.0pt, blue,
]
plot coordinates {
(05, 10)
(04, 10)
(02, 10)
(10, 8)
(07, 10)
(03, 10)
(08, 10)
(01, 10)
(09, 8)
(06, 10)
};
\label{plot:properties_hff_td_47}
\addplot+[ybar, bar shift =1.0pt, darkgreen,
]
plot coordinates {
(08, 10)
(03, 10)
(05, 10)
(04, 10)
(02, 10)
(06, 10)
(10, 8)
(07, 10)
(01, 10)
(09, 9)
};
\label{plot:properties_trap_prefop_bu_47}
\addplot+[ybar, bar shift =2.5pt, orange,
]
plot coordinates {
(03, 10)
(05, 10)
(04, 10)
(10, 10)
(07, 10)
(08, 10)
(01, 10)
(09, 10)
(06, 10)
(02, 10)
};
\label{plot:properties_trap_prefop_td_47}
\addplot+[only marks, mark = -, mark options = {thick}, mark size = 0.2cm, black,
]
plot coordinates {
(03, 10)
(05, 10)
(04, 10)
(10, 10)
(07, 10)
(08, 10)
(01, 10)
(09, 10)
(06, 10)
(02, 10)
};

\end{axis}
\hfill


\begin{axis}[
width = 7cm,
height=4cm,
enlarge x limits = 0.1,
enlarge y limits = 0.1,
ybar,
bar width=1pt,
ymin = 0,
ymax = 10,
at={(0.0\linewidth,-160.0)},
compat=1.6,
ylabel=goals 5,
]
\addplot+[ybar, bar shift =-2.5pt, red,
]
plot coordinates {
(03, 10)
(08, 10)
(09, 10)
(01, 10)
(06, 10)
(07, 10)
(04, 10)
(05, 10)
(10, 10)
(02, 10)
};
\label{plot:properties_hff_bu_47}
\addplot+[ybar, bar shift =-1.0pt, blue,
]
plot coordinates {
(03, 10)
(08, 9)
(09, 7)
(01, 10)
(06, 10)
(07, 10)
(04, 10)
(05, 10)
(10, 4)
(02, 10)
};
\label{plot:properties_hff_td_47}
\addplot+[ybar, bar shift =1.0pt, darkgreen,
]
plot coordinates {
(03, 10)
(08, 10)
(09, 10)
(01, 10)
(06, 10)
(07, 10)
(04, 10)
(05, 10)
(10, 10)
(02, 10)
};
\label{plot:properties_trap_prefop_bu_47}
\addplot+[ybar, bar shift =2.5pt, orange,
]
plot coordinates {
(03, 10)
(08, 10)
(09, 10)
(01, 10)
(06, 10)
(07, 10)
(04, 10)
(05, 10)
(10, 10)
(02, 10)
};
\label{plot:properties_trap_prefop_td_47}
\addplot+[only marks, mark = -, mark options = {thick}, mark size = 0.2cm, black,
]
plot coordinates {
(03, 10)
(08, 10)
(09, 10)
(01, 10)
(06, 10)
(07, 10)
(04, 10)
(05, 10)
(10, 10)
(02, 10)
};

\end{axis}
\hfill


\begin{axis}[
width = 7cm,
height=4cm,
enlarge x limits = 0.1,
enlarge y limits = 0.1,
ybar,
bar width=1pt,
ymin = 0,
ymax = 10,
at={(0.333333333333\linewidth,-160.0)},
compat=1.6,
]
\addplot+[ybar, bar shift =-2.5pt, red,
]
plot coordinates {
(02, 10)
(10, 3)
(05, 10)
(04, 10)
(07, 7)
(06, 9)
(09, 3)
(01, 10)
(08, 6)
(03, 10)
};
\label{plot:properties_hff_bu_47}
\addplot+[ybar, bar shift =-1.0pt, blue,
]
plot coordinates {
(02, 10)
(10, 2)
(04, 10)
(05, 10)
(07, 7)
(06, 9)
(09, 2)
(01, 10)
(08, 5)
(03, 10)
};
\label{plot:properties_hff_td_47}
\addplot+[ybar, bar shift =1.0pt, darkgreen,
]
plot coordinates {
(02, 10)
(10, 5)
(04, 10)
(05, 10)
(07, 9)
(06, 10)
(09, 8)
(01, 10)
(08, 8)
(03, 10)
};
\label{plot:properties_trap_prefop_bu_47}
\addplot+[ybar, bar shift =2.5pt, orange,
]
plot coordinates {
(02, 10)
(10, 7)
(05, 10)
(04, 10)
(07, 10)
(06, 10)
(09, 10)
(01, 10)
(08, 10)
(03, 10)
};
\label{plot:properties_trap_prefop_td_47}
\addplot+[only marks, mark = -, mark options = {thick}, mark size = 0.2cm, black,
]
plot coordinates {
(02, 10)
(10, 10)
(05, 10)
(04, 10)
(07, 10)
(06, 10)
(09, 10)
(01, 10)
(08, 10)
(03, 10)
};

\end{axis}
\hfill


\begin{axis}[
width = 7cm,
height=4cm,
enlarge x limits = 0.1,
enlarge y limits = 0.1,
ybar,
bar width=1pt,
ymin = 0,
ymax = 10,
at={(0.666666666667\linewidth,-160.0)},
compat=1.6,
]
\addplot+[ybar, bar shift =-2.5pt, red,
]
plot coordinates {
(03, 10)
(02, 10)
(10, 8)
(05, 10)
(04, 10)
(07, 9)
(06, 10)
(09, 9)
(01, 10)
(08, 9)
};
\label{plot:properties_hff_bu_47}
\addplot+[ybar, bar shift =-1.0pt, blue,
]
plot coordinates {
(03, 10)
(02, 10)
(10, 8)
(05, 10)
(04, 10)
(07, 9)
(06, 10)
(09, 9)
(01, 10)
(08, 9)
};
\label{plot:properties_hff_td_47}
\addplot+[ybar, bar shift =1.0pt, darkgreen,
]
plot coordinates {
(03, 10)
(02, 10)
(10, 6)
(05, 10)
(04, 10)
(07, 9)
(06, 10)
(09, 7)
(01, 10)
(08, 8)
};
\label{plot:properties_trap_prefop_bu_47}
\addplot+[ybar, bar shift =2.5pt, orange,
]
plot coordinates {
(03, 10)
(02, 10)
(10, 9)
(05, 10)
(04, 10)
(07, 10)
(06, 10)
(09, 10)
(01, 10)
(08, 10)
};
\label{plot:properties_trap_prefop_td_47}
\addplot+[only marks, mark = -, mark options = {thick}, mark size = 0.2cm, black,
]
plot coordinates {
(03, 10)
(02, 10)
(10, 10)
(05, 10)
(04, 10)
(07, 10)
(06, 10)
(09, 10)
(01, 10)
(08, 10)
};

\end{axis}
\hfill


\begin{axis}[
width = 7cm,
height=4cm,
enlarge x limits = 0.1,
enlarge y limits = 0.1,
ybar,
bar width=1pt,
ymin = 0,
ymax = 10,
at={(0.0\linewidth,-320.0)},
compat=1.6,
ylabel=goals 6,
]
\addplot+[ybar, bar shift =-2.5pt, red,
]
plot coordinates {
(10, 7)
(03, 10)
(05, 10)
(06, 10)
(01, 10)
(02, 10)
(09, 8)
(08, 9)
(07, 9)
(04, 10)
};
\label{plot:properties_hff_bu_47}
\addplot+[ybar, bar shift =-1.0pt, blue,
]
plot coordinates {
(10, 0)
(09, 0)
(02, 10)
(04, 10)
(05, 9)
(07, 4)
(06, 8)
(01, 10)
(08, 1)
(03, 10)
};
\label{plot:properties_hff_td_47}
\addplot+[ybar, bar shift =1.0pt, darkgreen,
]
plot coordinates {
(10, 10)
(04, 10)
(05, 10)
(06, 10)
(01, 10)
(02, 10)
(09, 10)
(08, 10)
(07, 10)
(03, 10)
};
\label{plot:properties_trap_prefop_bu_47}
\addplot+[ybar, bar shift =2.5pt, orange,
]
plot coordinates {
(10, 10)
(09, 10)
(03, 10)
(04, 10)
(05, 10)
(07, 10)
(06, 10)
(01, 10)
(02, 10)
(08, 10)
};
\label{plot:properties_trap_prefop_td_47}
\addplot+[only marks, mark = -, mark options = {thick}, mark size = 0.2cm, black,
]
plot coordinates {
(10, 10)
(09, 10)
(03, 10)
(04, 10)
(05, 10)
(07, 10)
(06, 10)
(01, 10)
(02, 10)
(08, 10)
};

\end{axis}
\hfill


\begin{axis}[
width = 7cm,
height=4cm,
enlarge x limits = 0.1,
enlarge y limits = 0.1,
ybar,
bar width=1pt,
ymin = 0,
ymax = 10,
at={(0.333333333333\linewidth,-320.0)},
compat=1.6,
]
\addplot+[ybar, bar shift =-2.5pt, red,
]
plot coordinates {
(10, 2)
(02, 10)
(03, 10)
(08, 3)
(01, 10)
(09, 2)
(06, 5)
(07, 3)
(04, 9)
(05, 8)
};
\label{plot:properties_hff_bu_47}
\addplot+[ybar, bar shift =-1.0pt, blue,
]
plot coordinates {
(10, 3)
(02, 10)
(03, 10)
(08, 3)
(01, 10)
(09, 3)
(06, 5)
(07, 3)
(04, 9)
(05, 8)
};
\label{plot:properties_hff_td_47}
\addplot+[ybar, bar shift =1.0pt, darkgreen,
]
plot coordinates {
(10, 4)
(02, 10)
(03, 10)
(08, 6)
(01, 10)
(09, 5)
(06, 10)
(07, 7)
(04, 10)
(05, 10)
};
\label{plot:properties_trap_prefop_bu_47}
\addplot+[ybar, bar shift =2.5pt, orange,
]
plot coordinates {
(10, 5)
(02, 10)
(03, 10)
(08, 8)
(01, 10)
(09, 6)
(06, 10)
(07, 9)
(04, 10)
(05, 10)
};
\label{plot:properties_trap_prefop_td_47}
\addplot+[only marks, mark = -, mark options = {thick}, mark size = 0.2cm, black,
]
plot coordinates {
(10, 10)
(02, 10)
(03, 10)
(08, 10)
(01, 10)
(09, 10)
(06, 10)
(07, 10)
(04, 10)
(05, 10)
};

\end{axis}
\hfill


\begin{axis}[
width = 7cm,
height=4cm,
enlarge x limits = 0.1,
enlarge y limits = 0.1,
ybar,
bar width=1pt,
ymin = 0,
ymax = 10,
at={(0.666666666667\linewidth,-320.0)},
compat=1.6,
]
\addplot+[ybar, bar shift =-2.5pt, red,
]
plot coordinates {
(04, 10)
(05, 10)
(02, 10)
(03, 10)
(08, 8)
(09, 8)
(01, 10)
(06, 9)
(07, 8)
(10, 7)
};
\label{plot:properties_hff_bu_47}
\addplot+[ybar, bar shift =-1.0pt, blue,
]
plot coordinates {
(04, 10)
(05, 10)
(02, 10)
(03, 10)
(08, 8)
(09, 8)
(01, 10)
(06, 9)
(07, 9)
(10, 7)
};
\label{plot:properties_hff_td_47}
\addplot+[ybar, bar shift =1.0pt, darkgreen,
]
plot coordinates {
(04, 10)
(05, 10)
(02, 10)
(03, 10)
(08, 6)
(09, 4)
(01, 10)
(06, 10)
(07, 7)
(10, 3)
};
\label{plot:properties_trap_prefop_bu_47}
\addplot+[ybar, bar shift =2.5pt, orange,
]
plot coordinates {
(04, 10)
(05, 10)
(02, 10)
(03, 10)
(08, 10)
(09, 9)
(01, 10)
(06, 10)
(07, 10)
(10, 8)
};
\label{plot:properties_trap_prefop_td_47}
\addplot+[only marks, mark = -, mark options = {thick}, mark size = 0.2cm, black,
]
plot coordinates {
(04, 10)
(05, 10)
(02, 10)
(03, 10)
(08, 10)
(09, 10)
(01, 10)
(06, 10)
(07, 10)
(10, 10)
};

\end{axis}
\hfill

\node[draw] (test) at (8,-8) {
\ref{plot:properties_hff_bu_47} properties-hff-bu
\ref{plot:properties_hff_td_47} properties-hff-td
\ref{plot:properties_trap_prefop_bu_47} properties-trap-prefop-bu
\ref{plot:properties_trap_prefop_td_47} properties-trap-prefop-td
};

\end{tikzpicture}
\hfill

\caption{TPP: reference coverage first node in top-down meta search tree;
	optimal coverage always 10}
\end{figure*}

\begin{figure*}[ht]

\tiny
\begin{tikzpicture}


\begin{axis}[
width = 7cm,
height=4cm,
enlarge x limits = 0.1,
enlarge y limits = 0.1,
ybar,
bar width=1pt,
ymin = 0,
ymax = 10,
at={(0.0\linewidth,-0.0)},
compat=1.6,
title=c 100,
ylabel=goals 05,
]
\addplot+[ybar, bar shift =-2.5pt, red,
]
plot coordinates {
(05, 10)
(04, 10)
(06, 9)
(02, 10)
(10, 6)
(01, 10)
(08, 9)
(09, 7)
(03, 10)
(07, 8)
};
\label{plot:properties_hff_bu_38}
\addplot+[ybar, bar shift =-0.5pt, blue,
]
plot coordinates {
(06, 9)
(10, 4)
(05, 10)
(04, 10)
(02, 10)
(01, 10)
(08, 5)
(09, 5)
(07, 7)
(03, 10)
};
\label{plot:properties_hff_td_38}
\addplot+[ybar, bar shift =0.5pt, darkgreen,
]
plot coordinates {
(06, 10)
(10, 10)
(05, 10)
(04, 10)
(03, 10)
(02, 10)
(01, 10)
(08, 10)
(09, 10)
(07, 10)
};
\label{plot:properties_trap_prefop_bu_38}
\addplot+[ybar, bar shift =2.5pt, orange,
]
plot coordinates {
(06, 10)
(05, 10)
(03, 10)
(07, 10)
(10, 10)
(02, 10)
(01, 10)
(08, 10)
(09, 10)
(04, 10)
};
\label{plot:properties_trap_prefop_td_38}

\addplot+[only marks, mark=-, mark options={thick}, mark size=0.2cm, black]
plot coordinates {
(1, 10)
(2, 10)
(3, 9)
(4, 8)
(5, 8)
(6, 6)
(7, 4)
(8, 6)
(9, 6)
(10, 4)
};
\end{axis}
\hfill


\begin{axis}[
width = 7cm,
height=4cm,
enlarge x limits = 0.1,
enlarge y limits = 0.1,
ybar,
bar width=1pt,
ymin = 0,
ymax = 10,
at={(0.333333333333\linewidth,-0.0)},
compat=1.6,
title=c 150,
]
\addplot+[ybar, bar shift =-2.5pt, red,
]
plot coordinates {
(08, 5)
(01, 10)
(09, 5)
(06, 8)
(07, 5)
(04, 9)
(05, 10)
(10, 4)
(02, 10)
(03, 9)
};
\label{plot:properties_hff_bu_38}
\addplot+[ybar, bar shift =-0.5pt, blue,
]
plot coordinates {
(01, 10)
(08, 4)
(09, 4)
(06, 4)
(07, 4)
(04, 9)
(05, 9)
(10, 4)
(02, 10)
(03, 9)
};
\label{plot:properties_hff_td_38}
\addplot+[ybar, bar shift =0.5pt, darkgreen,
]
plot coordinates {
(08, 10)
(01, 10)
(09, 10)
(06, 10)
(07, 10)
(04, 10)
(05, 10)
(10, 10)
(02, 10)
(03, 10)
};
\label{plot:properties_trap_prefop_bu_38}
\addplot+[ybar, bar shift =2.5pt, orange,
]
plot coordinates {
(08, 10)
(01, 10)
(09, 10)
(06, 10)
(07, 10)
(04, 10)
(05, 10)
(10, 10)
(02, 10)
(03, 10)
};
\label{plot:properties_trap_prefop_td_38}

\end{axis}
\hfill


\begin{axis}[
width = 7cm,
height=4cm,
enlarge x limits = 0.1,
enlarge y limits = 0.1,
ybar,
bar width=1pt,
ymin = 0,
ymax = 10,
at={(0.666666666667\linewidth,-0.0)},
compat=1.6,
title=c 200,
]
\addplot+[ybar, bar shift =-2.5pt, red,
]
plot coordinates {
(08, 8)
(04, 10)
(05, 10)
(10, 6)
(02, 10)
(03, 10)
(09, 9)
(01, 10)
(06, 7)
(07, 6)
};
\label{plot:properties_hff_bu_38}
\addplot+[ybar, bar shift =-0.5pt, blue,
]
plot coordinates {
(04, 10)
(05, 10)
(10, 7)
(02, 10)
(03, 10)
(08, 8)
(09, 9)
(01, 10)
(06, 7)
(07, 6)
};
\label{plot:properties_hff_td_38}
\addplot+[ybar, bar shift =0.5pt, darkgreen,
]
plot coordinates {
(04, 10)
(05, 10)
(10, 10)
(02, 10)
(03, 10)
(08, 10)
(09, 10)
(01, 10)
(06, 10)
(07, 10)
};
\label{plot:properties_trap_prefop_bu_38}
\addplot+[ybar, bar shift =2.5pt, orange,
]
plot coordinates {
(04, 10)
(05, 10)
(10, 10)
(02, 10)
(03, 10)
(08, 10)
(09, 10)
(01, 10)
(06, 10)
(07, 10)
};
\label{plot:properties_trap_prefop_td_38}

\end{axis}
\hfill


\begin{axis}[
width = 7cm,
height=4cm,
enlarge x limits = 0.1,
enlarge y limits = 0.1,
ybar,
bar width=1pt,
ymin = 0,
ymax = 10,
at={(0.0\linewidth,-160.0)},
compat=1.6,
ylabel=goals 06,
]
\addplot+[ybar, bar shift =-2.5pt, red,
]
plot coordinates {
(02, 10)
(03, 10)
(04, 10)
(05, 9)
(10, 7)
(06, 8)
(07, 7)
(08, 8)
(01, 10)
(09, 5)
};
\label{plot:properties_hff_bu_38}
\addplot+[ybar, bar shift =-0.5pt, blue,
]
plot coordinates {
(02, 10)
(03, 10)
(04, 10)
(05, 8)
(10, 4)
(06, 8)
(07, 7)
(08, 8)
(01, 10)
(09, 5)
};
\label{plot:properties_hff_td_38}
\addplot+[ybar, bar shift =0.5pt, darkgreen,
]
plot coordinates {
(02, 10)
(03, 10)
(04, 10)
(05, 10)
(10, 10)
(06, 10)
(07, 10)
(08, 10)
(01, 10)
(09, 10)
};
\label{plot:properties_trap_prefop_bu_38}
\addplot+[ybar, bar shift =2.5pt, orange,
]
plot coordinates {
(02, 10)
(03, 10)
(10, 10)
(04, 10)
(05, 10)
(06, 10)
(07, 10)
(08, 10)
(01, 10)
(09, 10)
};
\label{plot:properties_trap_prefop_td_38}

\addplot+[only marks, mark=-, mark options={thick}, mark size=0.2cm, black]
plot coordinates {
(1, 10)
(2, 9)
(3, 8)
(4, 7)
(5, 5)
(6, 7)
(7, 3)
(8, 3)
(9, 2)
(10, 1)
};
\end{axis}
\hfill


\begin{axis}[
width = 7cm,
height=4cm,
enlarge x limits = 0.1,
enlarge y limits = 0.1,
ybar,
bar width=1pt,
ymin = 0,
ymax = 10,
at={(0.333333333333\linewidth,-160.0)},
compat=1.6,
]
\addplot+[ybar, bar shift =-2.5pt, red,
]
plot coordinates {
(07, 8)
(06, 7)
(01, 10)
(09, 5)
(08, 5)
(03, 10)
(10, 4)
(02, 10)
(05, 9)
(04, 10)
};
\label{plot:properties_hff_bu_38}
\addplot+[ybar, bar shift =-0.5pt, blue,
]
plot coordinates {
(07, 6)
(06, 6)
(01, 10)
(09, 5)
(08, 5)
(03, 10)
(10, 4)
(02, 10)
(05, 9)
(04, 10)
};
\label{plot:properties_hff_td_38}
\addplot+[ybar, bar shift =0.5pt, darkgreen,
]
plot coordinates {
(07, 10)
(06, 10)
(01, 10)
(09, 10)
(08, 10)
(03, 10)
(10, 10)
(02, 10)
(05, 10)
(04, 10)
};
\label{plot:properties_trap_prefop_bu_38}
\addplot+[ybar, bar shift =2.5pt, orange,
]
plot coordinates {
(07, 10)
(06, 10)
(01, 10)
(09, 10)
(08, 10)
(03, 10)
(10, 10)
(02, 10)
(05, 10)
(04, 10)
};
\label{plot:properties_trap_prefop_td_38}

\end{axis}
\hfill


\begin{axis}[
width = 7cm,
height=4cm,
enlarge x limits = 0.1,
enlarge y limits = 0.1,
ybar,
bar width=1pt,
ymin = 0,
ymax = 10,
at={(0.666666666667\linewidth,-160.0)},
compat=1.6,
]
\addplot+[ybar, bar shift =-2.5pt, red,
]
plot coordinates {
(08, 4)
(09, 5)
(01, 10)
(07, 7)
(03, 10)
(10, 4)
(04, 10)
(02, 10)
(05, 9)
(06, 5)
};
\label{plot:properties_hff_bu_38}
\addplot+[ybar, bar shift =-0.5pt, blue,
]
plot coordinates {
(09, 8)
(08, 7)
(07, 8)
(01, 10)
(10, 4)
(02, 10)
(05, 9)
(03, 10)
(04, 10)
(06, 5)
};
\label{plot:properties_hff_td_38}
\addplot+[ybar, bar shift =0.5pt, darkgreen,
]
plot coordinates {
(08, 10)
(01, 10)
(09, 10)
(03, 10)
(06, 10)
(10, 10)
(04, 10)
(02, 10)
(05, 10)
(07, 10)
};
\label{plot:properties_trap_prefop_bu_38}
\addplot+[ybar, bar shift =2.5pt, orange,
]
plot coordinates {
(09, 10)
(01, 10)
(08, 10)
(07, 10)
(10, 10)
(04, 10)
(02, 10)
(05, 10)
(03, 10)
(06, 10)
};
\label{plot:properties_trap_prefop_td_38}

\end{axis}
\hfill


\begin{axis}[
width = 7cm,
height=4cm,
enlarge x limits = 0.1,
enlarge y limits = 0.1,
ybar,
bar width=1pt,
ymin = 0,
ymax = 10,
at={(0.0\linewidth,-320.0)},
compat=1.6,
ylabel=goals 07,
]
\addplot+[ybar, bar shift =-2.5pt, red,
]
plot coordinates {
(05, 7)
(04, 8)
(07, 3)
(06, 4)
(09, 4)
(01, 10)
(08, 4)
(03, 9)
(02, 10)
(10, 2)
};
\label{plot:properties_hff_bu_38}
\addplot+[ybar, bar shift =-0.5pt, blue,
]
plot coordinates {
(04, 8)
(07, 2)
(06, 3)
(09, 2)
(01, 10)
(08, 4)
(05, 7)
(03, 9)
(02, 10)
(10, 2)
};
\label{plot:properties_hff_td_38}
\addplot+[ybar, bar shift =0.5pt, darkgreen,
]
plot coordinates {
(04, 10)
(07, 10)
(06, 10)
(09, 10)
(01, 10)
(08, 10)
(05, 10)
(03, 10)
(02, 10)
(10, 10)
};
\label{plot:properties_trap_prefop_bu_38}
\addplot+[ybar, bar shift =2.5pt, orange,
]
plot coordinates {
(05, 10)
(04, 10)
(07, 10)
(06, 10)
(09, 10)
(01, 10)
(08, 10)
(03, 10)
(02, 10)
(10, 10)
};
\label{plot:properties_trap_prefop_td_38}

\addplot+[only marks, mark=-, mark options={thick}, mark size=0.2cm, black]
plot coordinates {
(1, 9)
(2, 8)
(3, 7)
(4, 7)
(5, 6)
(6, 2)
(7, 0)
(8, 3)
(9, 3)
(10, 2)
};
\end{axis}
\hfill


\begin{axis}[
width = 7cm,
height=4cm,
enlarge x limits = 0.1,
enlarge y limits = 0.1,
ybar,
bar width=1pt,
ymin = 0,
ymax = 10,
at={(0.333333333333\linewidth,-320.0)},
compat=1.6,
]
\addplot+[ybar, bar shift =-2.5pt, red,
]
plot coordinates {
(08, 4)
(09, 3)
(01, 10)
(03, 8)
(02, 10)
(04, 9)
(05, 10)
(06, 6)
(10, 2)
(07, 6)
};
\label{plot:properties_hff_bu_38}
\addplot+[ybar, bar shift =-0.5pt, blue,
]
plot coordinates {
(01, 10)
(08, 5)
(02, 10)
(04, 9)
(03, 8)
(05, 10)
(06, 5)
(10, 5)
(07, 7)
(09, 6)
};
\label{plot:properties_hff_td_38}
\addplot+[ybar, bar shift =0.5pt, darkgreen,
]
plot coordinates {
(09, 10)
(01, 10)
(02, 10)
(04, 10)
(03, 10)
(05, 10)
(06, 10)
(10, 10)
(07, 10)
(08, 10)
};
\label{plot:properties_trap_prefop_bu_38}
\addplot+[ybar, bar shift =2.5pt, orange,
]
plot coordinates {
(01, 10)
(08, 10)
(02, 10)
(03, 10)
(04, 10)
(05, 10)
(06, 10)
(10, 10)
(07, 10)
(09, 10)
};
\label{plot:properties_trap_prefop_td_38}

\end{axis}
\hfill


\begin{axis}[
width = 7cm,
height=4cm,
enlarge x limits = 0.1,
enlarge y limits = 0.1,
ybar,
bar width=1pt,
ymin = 0,
ymax = 10,
at={(0.666666666667\linewidth,-320.0)},
compat=1.6,
]
\addplot+[ybar, bar shift =-2.5pt, red,
]
plot coordinates {
(08, 4)
(07, 8)
(01, 10)
(09, 5)
(02, 10)
(03, 10)
(04, 10)
(05, 10)
(06, 8)
(10, 2)
};
\label{plot:properties_hff_bu_38}
\addplot+[ybar, bar shift =-0.5pt, blue,
]
plot coordinates {
(08, 5)
(07, 9)
(01, 10)
(09, 7)
(03, 10)
(05, 10)
(04, 10)
(06, 8)
(02, 10)
(10, 7)
};
\label{plot:properties_hff_td_38}
\addplot+[ybar, bar shift =0.5pt, darkgreen,
]
plot coordinates {
(08, 10)
(09, 10)
(03, 10)
(05, 10)
(04, 10)
(06, 10)
(02, 10)
(10, 10)
(07, 10)
(01, 10)
};
\label{plot:properties_trap_prefop_bu_38}
\addplot+[ybar, bar shift =2.5pt, orange,
]
plot coordinates {
(08, 10)
(07, 10)
(09, 10)
(03, 10)
(04, 10)
(05, 10)
(06, 10)
(02, 10)
(10, 10)
(01, 10)
};
\label{plot:properties_trap_prefop_td_38}

\end{axis}
\hfill


\begin{axis}[
width = 7cm,
height=4cm,
enlarge x limits = 0.1,
enlarge y limits = 0.1,
ybar,
bar width=1pt,
ymin = 0,
ymax = 10,
at={(0.0\linewidth,-480.0)},
compat=1.6,
ylabel=goals 08,
]
\addplot+[ybar, bar shift =-2.5pt, red,
]
plot coordinates {
(04, 6)
(05, 6)
(02, 10)
(10, 3)
(03, 9)
(08, 3)
(09, 4)
(06, 6)
(07, 4)
(01, 9)
};
\label{plot:properties_hff_bu_38}
\addplot+[ybar, bar shift =-0.5pt, blue,
]
plot coordinates {
(04, 6)
(05, 6)
(02, 10)
(10, 3)
(03, 9)
(08, 3)
(09, 4)
(06, 5)
(07, 5)
(01, 9)
};
\label{plot:properties_hff_td_38}
\addplot+[ybar, bar shift =0.5pt, darkgreen,
]
plot coordinates {
(01, 10)
(04, 10)
(05, 10)
(02, 10)
(10, 10)
(03, 10)
(08, 10)
(09, 10)
(06, 10)
(07, 10)
};
\label{plot:properties_trap_prefop_bu_38}
\addplot+[ybar, bar shift =2.5pt, orange,
]
plot coordinates {
(04, 10)
(05, 10)
(02, 10)
(10, 10)
(03, 10)
(08, 10)
(09, 10)
(01, 10)
(06, 10)
(07, 10)
};
\label{plot:properties_trap_prefop_td_38}

\addplot+[only marks, mark=-, mark options={thick}, mark size=0.2cm, black]
plot coordinates {
(1, 8)
(2, 7)
(3, 7)
(4, 6)
(5, 3)
(6, 3)
(7, 2)
(8, 0)
(9, 1)
(10, 0)
};
\end{axis}
\hfill


\begin{axis}[
width = 7cm,
height=4cm,
enlarge x limits = 0.1,
enlarge y limits = 0.1,
ybar,
bar width=1pt,
ymin = 0,
ymax = 10,
at={(0.333333333333\linewidth,-480.0)},
compat=1.6,
]
\addplot+[ybar, bar shift =-2.5pt, red,
]
plot coordinates {
(02, 10)
(09, 4)
(01, 10)
(08, 3)
(07, 4)
(06, 6)
(10, 2)
(05, 8)
(04, 8)
(03, 10)
};
\label{plot:properties_hff_bu_38}
\addplot+[ybar, bar shift =-0.5pt, blue,
]
plot coordinates {
(02, 10)
(09, 4)
(01, 10)
(08, 3)
(07, 5)
(06, 7)
(10, 2)
(05, 8)
(04, 8)
(03, 10)
};
\label{plot:properties_hff_td_38}
\addplot+[ybar, bar shift =0.5pt, darkgreen,
]
plot coordinates {
(02, 10)
(09, 10)
(01, 10)
(08, 10)
(07, 10)
(06, 10)
(10, 10)
(05, 10)
(04, 10)
(03, 10)
};
\label{plot:properties_trap_prefop_bu_38}
\addplot+[ybar, bar shift =2.5pt, orange,
]
plot coordinates {
(02, 10)
(09, 10)
(01, 10)
(08, 10)
(07, 10)
(06, 10)
(10, 10)
(05, 10)
(04, 10)
(03, 10)
};
\label{plot:properties_trap_prefop_td_38}

\end{axis}
\hfill


\begin{axis}[
width = 7cm,
height=4cm,
enlarge x limits = 0.1,
enlarge y limits = 0.1,
ybar,
bar width=1pt,
ymin = 0,
ymax = 10,
at={(0.666666666667\linewidth,-480.0)},
compat=1.6,
]
\addplot+[ybar, bar shift =-2.5pt, red,
]
plot coordinates {
(09, 4)
(03, 9)
(04, 9)
(07, 8)
(08, 5)
(10, 3)
(06, 7)
(01, 10)
(05, 7)
(02, 10)
};
\label{plot:properties_hff_bu_38}
\addplot+[ybar, bar shift =-0.5pt, blue,
]
plot coordinates {
(02, 10)
(03, 10)
(01, 10)
(04, 9)
(08, 7)
(06, 9)
(07, 8)
(10, 4)
(09, 5)
(05, 8)
};
\label{plot:properties_hff_td_38}
\addplot+[ybar, bar shift =0.5pt, darkgreen,
]
plot coordinates {
(03, 10)
(05, 10)
(08, 10)
(07, 10)
(10, 10)
(06, 10)
(09, 10)
(01, 10)
(02, 10)
(04, 10)
};
\label{plot:properties_trap_prefop_bu_38}
\addplot+[ybar, bar shift =2.5pt, orange,
]
plot coordinates {
(02, 10)
(01, 10)
(03, 10)
(04, 10)
(07, 10)
(08, 10)
(10, 10)
(06, 10)
(09, 10)
(05, 10)
};
\label{plot:properties_trap_prefop_td_38}

\end{axis}
\hfill


\begin{axis}[
width = 7cm,
height=4cm,
enlarge x limits = 0.1,
enlarge y limits = 0.1,
ybar,
bar width=1pt,
ymin = 0,
ymax = 10,
at={(0.0\linewidth,-640.0)},
compat=1.6,
ylabel=goals 09,
]
\addplot+[ybar, bar shift =-2.5pt, red,
]
plot coordinates {
(02, 10)
(01, 10)
(09, 4)
(08, 4)
(07, 4)
(06, 6)
(10, 4)
(05, 6)
(04, 7)
(03, 8)
};
\label{plot:properties_hff_bu_38}
\addplot+[ybar, bar shift =-0.5pt, blue,
]
plot coordinates {
(02, 10)
(01, 10)
(09, 4)
(05, 6)
(08, 5)
(07, 4)
(06, 4)
(10, 4)
(04, 7)
(03, 8)
};
\label{plot:properties_hff_td_38}
\addplot+[ybar, bar shift =0.5pt, darkgreen,
]
plot coordinates {
(02, 10)
(01, 10)
(09, 10)
(05, 10)
(08, 10)
(07, 10)
(06, 10)
(10, 10)
(04, 10)
(03, 10)
};
\label{plot:properties_trap_prefop_bu_38}
\addplot+[ybar, bar shift =2.5pt, orange,
]
plot coordinates {
(02, 10)
(01, 10)
(09, 10)
(05, 10)
(08, 10)
(07, 10)
(06, 10)
(10, 10)
(04, 10)
(03, 10)
};
\label{plot:properties_trap_prefop_td_38}

\addplot+[only marks, mark=-, mark options={thick}, mark size=0.2cm, black]
plot coordinates {
(1, 7)
(2, 3)
(3, 5)
(4, 1)
(5, 4)
(6, 0)
(7, 0)
(8, 1)
(9, 0)
(10, 0)
};
\end{axis}
\hfill


\begin{axis}[
width = 7cm,
height=4cm,
enlarge x limits = 0.1,
enlarge y limits = 0.1,
ybar,
bar width=1pt,
ymin = 0,
ymax = 10,
at={(0.333333333333\linewidth,-640.0)},
compat=1.6,
]
\addplot+[ybar, bar shift =-2.5pt, red,
]
plot coordinates {
(02, 9)
(05, 7)
(10, 4)
(07, 6)
(03, 8)
(04, 8)
(08, 7)
(01, 10)
(09, 4)
(06, 5)
};
\label{plot:properties_hff_bu_38}
\addplot+[ybar, bar shift =-0.5pt, blue,
]
plot coordinates {
(03, 8)
(04, 8)
(07, 7)
(05, 7)
(02, 9)
(10, 6)
(08, 7)
(01, 10)
(09, 4)
(06, 6)
};
\label{plot:properties_hff_td_38}
\addplot+[ybar, bar shift =0.5pt, darkgreen,
]
plot coordinates {
(01, 10)
(07, 10)
(04, 10)
(05, 10)
(02, 10)
(10, 10)
(03, 10)
(08, 10)
(09, 10)
(06, 10)
};
\label{plot:properties_trap_prefop_bu_38}
\addplot+[ybar, bar shift =2.5pt, orange,
]
plot coordinates {
(07, 10)
(03, 10)
(02, 10)
(04, 10)
(05, 10)
(10, 10)
(08, 10)
(01, 10)
(09, 10)
(06, 10)
};
\label{plot:properties_trap_prefop_td_38}

\end{axis}
\hfill


\begin{axis}[
width = 7cm,
height=4cm,
enlarge x limits = 0.1,
enlarge y limits = 0.1,
ybar,
bar width=1pt,
ymin = 0,
ymax = 10,
at={(0.666666666667\linewidth,-640.0)},
compat=1.6,
]
\addplot+[ybar, bar shift =-2.5pt, red,
]
plot coordinates {
(03, 9)
(08, 5)
(09, 4)
(04, 9)
(02, 10)
(07, 4)
(05, 8)
(10, 4)
(06, 4)
(01, 10)
};
\label{plot:properties_hff_bu_38}
\addplot+[ybar, bar shift =-0.5pt, blue,
]
plot coordinates {
(08, 6)
(09, 5)
(03, 9)
(04, 10)
(02, 10)
(06, 7)
(07, 5)
(05, 9)
(10, 7)
(01, 10)
};
\label{plot:properties_hff_td_38}
\addplot+[ybar, bar shift =0.5pt, darkgreen,
]
plot coordinates {
(08, 10)
(09, 10)
(03, 10)
(04, 10)
(07, 10)
(02, 10)
(05, 10)
(10, 10)
(06, 10)
(01, 10)
};
\label{plot:properties_trap_prefop_bu_38}
\addplot+[ybar, bar shift =2.5pt, orange,
]
plot coordinates {
(08, 10)
(01, 10)
(09, 10)
(03, 10)
(04, 10)
(07, 10)
(05, 10)
(10, 10)
(02, 10)
(06, 10)
};
\label{plot:properties_trap_prefop_td_38}

\end{axis}
\hfill

\node[draw] (test) at (8,-14) {
\ref{plot:properties_hff_bu_38} properties-hff-bu
\ref{plot:properties_hff_td_38} properties-hff-td
\ref{plot:properties_trap_prefop_bu_38} properties-trap-prefop-bu
\ref{plot:properties_trap_prefop_td_38} properties-trap-prefop-td
};

\end{tikzpicture}
\hfill

\caption{rovers: reference optimal coverage; coverage first node in top down meta search
	tree always 10}
\end{figure*}

\begin{figure*}[ht]

\tiny
\begin{tikzpicture}


\begin{axis}[
width = 7cm,
height=4cm,
enlarge x limits = 0.1,
enlarge y limits = 0.1,
ybar,
bar width=1pt,
ymin = 0,
ymax = 10,
at={(0.0\linewidth,-0.0)},
compat=1.6,
ylabel=\small \#goals 4,
]
\addplot+[ybar, bar shift =-2.5pt, red,
]
plot coordinates {
(06, 10)
(09, 10)
(04, 10)
(08, 10)
(10, 10)
(05, 10)
(07, 10)
(03, 10)
(02, 10)
(01, 10)
};
\label{plot:properties_hff_bu_21}
\addplot+[ybar, bar shift =-1.0pt, blue,
]
plot coordinates {
(02, 10)
(01, 10)
(06, 10)
(09, 5)
(04, 10)
(08, 6)
(10, 3)
(05, 10)
(07, 10)
(03, 10)
};
\label{plot:properties_hff_td_21}
\addplot+[ybar, bar shift =1.0pt, darkgreen,
]
plot coordinates {
(02, 10)
(01, 10)
(06, 10)
(09, 10)
(04, 10)
(08, 10)
(10, 10)
(05, 10)
(07, 10)
(03, 10)
};
\label{plot:properties_trap_prefop_bu_21}
\addplot+[ybar, bar shift =2.5pt, orange,
]
plot coordinates {
(02, 10)
(06, 10)
(09, 10)
(04, 10)
(08, 10)
(10, 10)
(05, 10)
(07, 10)
(01, 10)
(03, 10)
};
\label{plot:properties_trap_prefop_td_21}

%start node sysW
\addplot+[only marks, mark = -, mark options = {thick}, mark size = 0.2cm, black,
]
plot coordinates {
(01, 10)
(02, 10)
(03, 10)
(04, 10)
(05, 10)
(06, 10)
(07, 10)
(08, 10)
(09, 10)
(10, 10)
};
%optimal
\addplot+[only marks, mark = -, mark options = {thick, red, dashed}, mark size = 0.2cm, black,
]
plot coordinates {
(01, 10)
(02, 10)
(03, 10)
(04, 10)
(05, 10)
(06, 10)
(07, 10)
(08, 10)
(09, 10)
(10, 10)
};
\end{axis}
\hfill


\begin{axis}[
width = 7cm,
height=4cm,
enlarge x limits = 0.1,
enlarge y limits = 0.1,
ybar,
bar width=1pt,
ymin = 0,
ymax = 10,
at={(0.333333333333\linewidth,-0.0)},
compat=1.6,
]
\addplot+[ybar, bar shift =-2.5pt, red,
]
plot coordinates {
(08, 9)
(04, 10)
(01, 10)
(06, 10)
(03, 10)
(07, 10)
(02, 10)
(05, 10)
(09, 5)
(10, 2)
};
\label{plot:properties_hff_bu_21}
\addplot+[ybar, bar shift =-1.0pt, blue,
]
plot coordinates {
(08, 10)
(03, 10)
(09, 3)
(04, 10)
(01, 10)
(06, 10)
(07, 10)
(02, 10)
(05, 10)
(10, 1)
};
\label{plot:properties_hff_td_21}
\addplot+[ybar, bar shift =1.0pt, darkgreen,
]
plot coordinates {
(03, 10)
(09, 4)
(04, 10)
(01, 10)
(06, 10)
(07, 10)
(02, 10)
(05, 10)
(08, 9)
(10, 2)
};
\label{plot:properties_trap_prefop_bu_21}
\addplot+[ybar, bar shift =2.5pt, orange,
]
plot coordinates {
(03, 10)
(04, 10)
(01, 10)
(06, 10)
(07, 10)
(02, 10)
(05, 10)
(09, 6)
(08, 10)
(10, 2)
};
\label{plot:properties_trap_prefop_td_21}

%start node sysW
\addplot+[only marks, mark = -, mark options = {thick}, mark size = 0.2cm, black,
]
plot coordinates {
(01, 10)
(02, 10)
(03, 10)
(04, 10)
(05, 10)
(06, 10)
(07, 10)
(08, 10)
(09, 10)
(10, 10)
};
%optimal
\addplot+[only marks, mark = -, mark options = {thick, red, dashed}, mark size = 0.2cm, black,
]
plot coordinates {
(01, 10)
(02, 10)
(03, 10)
(04, 10)
(05, 10)
(06, 10)
(07, 10)
(08, 10)
(09, 10)
(10, 10)
};
\end{axis}
\hfill


\begin{axis}[
width = 7cm,
height=4cm,
enlarge x limits = 0.1,
enlarge y limits = 0.1,
ybar,
bar width=1pt,
ymin = 0,
ymax = 10,
at={(0.666666666667\linewidth,-0.0)},
compat=1.6,
]
\addplot+[ybar, bar shift =-2.5pt, red,
]
plot coordinates {
(05, 10)
(03, 10)
(04, 10)
(08, 10)
(09, 9)
(10, 7)
(01, 10)
(06, 10)
(07, 10)
(02, 10)
};
\label{plot:properties_hff_bu_21}
\addplot+[ybar, bar shift =-1.0pt, blue,
]
plot coordinates {
(05, 10)
(10, 8)
(03, 10)
(04, 10)
(08, 9)
(09, 9)
(01, 10)
(06, 10)
(07, 9)
(02, 10)
};
\label{plot:properties_hff_td_21}
\addplot+[ybar, bar shift =1.0pt, darkgreen,
]
plot coordinates {
(05, 10)
(10, 1)
(03, 10)
(04, 10)
(08, 7)
(09, 3)
(01, 10)
(06, 10)
(07, 10)
(02, 10)
};
\label{plot:properties_trap_prefop_bu_21}
\addplot+[ybar, bar shift =2.5pt, orange,
]
plot coordinates {
(05, 10)
(10, 4)
(03, 10)
(04, 10)
(08, 10)
(09, 7)
(01, 10)
(06, 10)
(07, 10)
(02, 10)
};
%optimal
\addplot+[only marks, mark = -, mark options = {thick, red, dashed}, mark size = 0.2cm, black,
]
plot coordinates {
(01, 10)
(02, 10)
(03, 10)
(04, 10)
(05, 10)
(06, 10)
(07, 10)
(08, 10)
(09, 10)
(10, 10)
};
\label{plot:properties_trap_prefop_td_21}

%start node sysW
\addplot+[only marks, mark = -, mark options = {thick}, mark size = 0.2cm, black,
]
plot coordinates {
(01, 10)
(02, 10)
(03, 10)
(04, 10)
(05, 10)
(06, 10)
(07, 10)
(08, 10)
(09, 8)
(10, 5)
};
\end{axis}
\hfill


\begin{axis}[
width = 7cm,
height=4cm,
enlarge x limits = 0.1,
enlarge y limits = 0.1,
ybar,
bar width=1pt,
ymin = 0,
ymax = 10,
at={(0.0\linewidth,-160.0)},
compat=1.6,
ylabel=\small \#goals 5,
]
\addplot+[ybar, bar shift =-2.5pt, red,
]
plot coordinates {
(06, 6)
(03, 10)
(04, 10)
(08, 2)
(01, 10)
(09, 2)
(10, 1)
(05, 10)
(07, 3)
(02, 10)
};
\label{plot:properties_hff_bu_21}
\addplot+[ybar, bar shift =-1.0pt, blue,
]
plot coordinates {
(06, 3)
(04, 10)
(08, 1)
(02, 10)
(01, 10)
(09, 1)
(10, 1)
(05, 6)
(07, 1)
(03, 10)
};
\label{plot:properties_hff_td_21}
\addplot+[ybar, bar shift =1.0pt, darkgreen,
]
plot coordinates {
(06, 10)
(04, 10)
(08, 10)
(01, 10)
(09, 10)
(10, 10)
(05, 10)
(07, 10)
(02, 10)
(03, 10)
};
\label{plot:properties_trap_prefop_bu_21}
\addplot+[ybar, bar shift =2.5pt, orange,
]
plot coordinates {
(06, 10)
(04, 10)
(08, 7)
(09, 3)
(10, 1)
(05, 10)
(07, 10)
(01, 10)
(02, 10)
(03, 10)
};
\label{plot:properties_trap_prefop_td_21}

%start node sysW
\addplot+[only marks, mark = -, mark options = {thick}, mark size = 0.2cm, black,
]
plot coordinates {
(01, 10)
(02, 10)
(03, 10)
(04, 10)
(05, 10)
(06, 10)
(07, 10)
(08, 10)
(09, 10)
(10, 10)
};
%optimal
\addplot+[only marks, mark = -, mark options = {thick, red, dashed}, mark size = 0.2cm, black,
]
plot coordinates {
(01, 10)
(02, 10)
(03, 10)
(04, 10)
(05, 10)
(06, 10)
(07, 10)
(08, 10)
(09, 10)
(10, 10)
};
\end{axis}
\hfill


\begin{axis}[
width = 7cm,
height=4cm,
enlarge x limits = 0.1,
enlarge y limits = 0.1,
ybar,
bar width=1pt,
ymin = 0,
ymax = 10,
at={(0.333333333333\linewidth,-160.0)},
compat=1.6,
]
\addplot+[ybar, bar shift =-2.5pt, red,
]
plot coordinates {
(03, 10)
(10, 1)
(08, 5)
(01, 10)
(04, 10)
(07, 7)
(09, 2)
(05, 9)
(06, 9)
(02, 10)
};
\label{plot:properties_hff_bu_21}
\addplot+[ybar, bar shift =-1.0pt, blue,
]
plot coordinates {
(03, 10)
(10, 4)
(01, 10)
(08, 8)
(04, 10)
(07, 10)
(05, 10)
(09, 7)
(06, 10)
(02, 10)
};
\label{plot:properties_hff_td_21}
\addplot+[ybar, bar shift =1.0pt, darkgreen,
]
plot coordinates {
(03, 10)
(10, 1)
(01, 10)
(08, 1)
(04, 10)
(07, 6)
(05, 6)
(09, 2)
(06, 5)
(02, 10)
};
\label{plot:properties_trap_prefop_bu_21}
\addplot+[ybar, bar shift =2.5pt, orange,
]
plot coordinates {
(03, 10)
(10, 0)
(01, 10)
(04, 10)
(07, 7)
(05, 6)
(08, 5)
(09, 3)
(06, 6)
(02, 10)
};
\label{plot:properties_trap_prefop_td_21}

%start node sysW
\addplot+[only marks, mark = -, mark options = {thick}, mark size = 0.2cm, black,
]
plot coordinates {
(01, 10)
(02, 10)
(03, 10)
(04, 10)
(05, 6)
(06, 6)
(07, 7)
(08, 5)
(09, 3)
(10, 1)
};
%optimal
\addplot+[only marks, mark = -, mark options = {thick, red, dashed}, mark size = 0.2cm, black,
]
plot coordinates {
(01, 10)
(02, 10)
(03, 10)
(04, 10)
(05, 10)
(06, 10)
(07, 10)
(08, 10)
(09, 10)
(10, 10)
};
\end{axis}
\hfill


\begin{axis}[
width = 7cm,
height=4cm,
enlarge x limits = 0.1,
enlarge y limits = 0.1,
ybar,
bar width=1pt,
ymin = 0,
ymax = 10,
at={(0.666666666667\linewidth,-160.0)},
compat=1.6,
]
\addplot+[ybar, bar shift =-2.5pt, red,
]
plot coordinates {
(06, 10)
(08, 9)
(02, 10)
(04, 10)
(09, 6)
(01, 10)
(07, 9)
(03, 10)
(05, 10)
(10, 3)
};
\label{plot:properties_hff_bu_21}
\addplot+[ybar, bar shift =-1.0pt, blue,
]
plot coordinates {
(06, 9)
(02, 10)
(04, 10)
(09, 9)
(10, 9)
(08, 9)
(01, 10)
(07, 9)
(03, 10)
(05, 9)
};
\label{plot:properties_hff_td_21}
\addplot+[ybar, bar shift =1.0pt, darkgreen,
]
plot coordinates {
(06, 7)
(02, 10)
(04, 10)
(09, 0)
(10, 0)
(08, 2)
(01, 10)
(07, 5)
(03, 10)
(05, 10)
};
\label{plot:properties_trap_prefop_bu_21}
\addplot+[ybar, bar shift =2.5pt, orange,
]
plot coordinates {
(06, 8)
(02, 10)
(04, 10)
(09, 1)
(10, 0)
(08, 5)
(01, 10)
(07, 7)
(03, 10)
(05, 10)
};
\label{plot:properties_trap_prefop_td_21}

%start node sysW
\addplot+[only marks, mark = -, mark options = {thick}, mark size = 0.2cm, black,
]
plot coordinates {
(01, 10)
(02, 10)
(03, 10)
(04, 10)
(05, 10)
(06, 8)
(07, 7)
(08, 5)
(09, 2)
(10, 1)
};
%optimal
\addplot+[only marks, mark = -, mark options = {thick, red, dashed}, mark size = 0.2cm, black,
]
plot coordinates {
(01, 10)
(02, 10)
(03, 10)
(04, 10)
(05, 10)
(06, 10)
(07, 10)
(08, 10)
(09, 10)
(10, 10)
};
\end{axis}
\hfill


\begin{axis}[
width = 7cm,
height=4cm,
enlarge x limits = 0.1,
enlarge y limits = 0.1,
ybar,
bar width=1pt,
ymin = 0,
ymax = 10,
at={(0.0\linewidth,-320.0)},
compat=1.6,
ylabel=\small \#goals 6,
]
\addplot+[ybar, bar shift =-2.5pt, red,
]
plot coordinates {
(08, 0)
(04, 4)
(10, 0)
(07, 0)
(03, 6)
(01, 10)
(06, 0)
(09, 0)
(05, 0)
(02, 8)
};
\label{plot:properties_hff_bu_21}
\addplot+[ybar, bar shift =-1.0pt, blue,
]
plot coordinates {
(08, 0)
(04, 1)
(10, 0)
(07, 0)
(02, 8)
(03, 5)
(01, 10)
(06, 0)
(09, 0)
(05, 0)
};
\label{plot:properties_hff_td_21}
\addplot+[ybar, bar shift =1.0pt, darkgreen,
]
plot coordinates {
(08, 8)
(04, 9)
(10, 8)
(02, 9)
(07, 8)
(03, 9)
(01, 10)
(06, 8)
(09, 8)
(05, 8)
};
\label{plot:properties_trap_prefop_bu_21}
\addplot+[ybar, bar shift =2.5pt, orange,
]
plot coordinates {
(08, 0)
(04, 6)
(10, 0)
(02, 8)
(07, 2)
(03, 8)
(01, 10)
(06, 4)
(09, 0)
(05, 5)
};
\label{plot:properties_trap_prefop_td_21}

%start node sysW
\addplot+[only marks, mark = -, mark options = {thick}, mark size = 0.2cm, black,
]
plot coordinates {
(01, 10)
(02, 10)
(03, 8)
(04, 8)
(05, 8)
(06, 8)
(07, 8)
(08, 8)
(09, 9)
(10, 9)
};
%optimal
\addplot+[only marks, mark = -, mark options = {thick, red, dashed}, mark size = 0.2cm, black,
]
plot coordinates {
(01, 10)
(02, 10)
(03, 10)
(04, 10)
(05, 10)
(06, 10)
(07, 10)
(08, 10)
(09, 10)
(10, 10)
};
\end{axis}
\hfill


\begin{axis}[
width = 7cm,
height=4cm,
enlarge x limits = 0.1,
enlarge y limits = 0.1,
ybar,
bar width=1pt,
ymin = 0,
ymax = 10,
at={(0.333333333333\linewidth,-320.0)},
compat=1.6,
]
\addplot+[ybar, bar shift =-2.5pt, red,
]
plot coordinates {
(09, 1)
(10, 0)
(04, 10)
(07, 6)
(03, 10)
(01, 10)
(08, 2)
(06, 8)
(02, 10)
(05, 8)
};
\label{plot:properties_hff_bu_21}
\addplot+[ybar, bar shift =-1.0pt, blue,
]
plot coordinates {
(04, 10)
(09, 5)
(03, 10)
(01, 10)
(08, 8)
(10, 4)
(06, 10)
(02, 10)
(05, 10)
(07, 10)
};
\label{plot:properties_hff_td_21}
\addplot+[ybar, bar shift =1.0pt, darkgreen,
]
plot coordinates {
(04, 3)
(09, 0)
(03, 4)
(01, 10)
(08, 0)
(10, 0)
(06, 0)
(02, 10)
(05, 1)
(07, 2)
};
\label{plot:properties_trap_prefop_bu_21}
\addplot+[ybar, bar shift =2.5pt, orange,
]
plot coordinates {
(04, 3)
(09, 0)
(07, 3)
(03, 5)
(01, 10)
(08, 2)
(10, 0)
(06, 1)
(02, 10)
(05, 1)
};
\label{plot:properties_trap_prefop_td_21}

%start node sysW
\addplot+[only marks, mark = -, mark options = {thick}, mark size = 0.2cm, black,
]
plot coordinates {
(01, 10)
(02, 10)
(03, 5)
(04, 3)
(05, 1)
(06, 1)
(07, 3)
(08, 2)
(09, 0)
(10, 0)
};
%optimal
\addplot+[only marks, mark = -, mark options = {thick, red, dashed}, mark size = 0.2cm, black,
]
plot coordinates {
(01, 10)
(02, 10)
(03, 10)
(04, 10)
(05, 10)
(06, 10)
(07, 10)
(08, 10)
(09, 10)
(10, 10)
};
\end{axis}
\hfill


\begin{axis}[
width = 7cm,
height=4cm,
enlarge x limits = 0.1,
enlarge y limits = 0.1,
ybar,
bar width=1pt,
ymin = 0,
ymax = 10,
at={(0.666666666667\linewidth,-320.0)},
compat=1.6,
]
\addplot+[ybar, bar shift =-2.5pt, red,
]
plot coordinates {
(01, 10)
(08, 7)
(06, 10)
(02, 10)
(05, 10)
(10, 1)
(07, 9)
(03, 10)
(09, 2)
(04, 10)
};
\label{plot:properties_hff_bu_21}
\addplot+[ybar, bar shift =-1.0pt, blue,
]
plot coordinates {
(01, 10)
(08, 10)
(06, 10)
(02, 10)
(05, 10)
(07, 10)
(10, 10)
(03, 10)
(09, 10)
(04, 10)
};
\label{plot:properties_hff_td_21}
\addplot+[ybar, bar shift =1.0pt, darkgreen,
]
plot coordinates {
(01, 9)
(06, 1)
(02, 8)
(05, 3)
(08, 0)
(07, 0)
(10, 0)
(03, 3)
(09, 0)
(04, 4)
};
\label{plot:properties_trap_prefop_bu_21}
\addplot+[ybar, bar shift =2.5pt, orange,
]
plot coordinates {
(01, 9)
(08, 0)
(06, 1)
(02, 8)
(05, 4)
(10, 0)
(07, 2)
(03, 4)
(09, 0)
(04, 5)
};
\label{plot:properties_trap_prefop_td_21}

%start node sysW
\addplot+[only marks, mark = -, mark options = {thick}, mark size = 0.2cm, black,
]
plot coordinates {
(01, 9)
(02, 8)
(03, 4)
(04, 5)
(05, 4)
(06, 1)
(07, 2)
(08, 0)
(09, 0)
(10, 0)
};
%optimal
\addplot+[only marks, mark = -, mark options = {thick, red, dashed}, mark size = 0.2cm, black,
]
plot coordinates {
(01, 10)
(02, 10)
(03, 10)
(04, 10)
(05, 10)
(06, 10)
(07, 10)
(08, 10)
(09, 10)
(10, 10)
};
\end{axis}
\hfill


\begin{axis}[
width = 7cm,
height=4cm,
enlarge x limits = 0.1,
enlarge y limits = 0.1,
ybar,
bar width=1pt,
ymin = 0,
ymax = 10,
at={(0.0\linewidth,-480.0)},
compat=1.6,
ylabel=\small \#goals 7,
]
\addplot+[ybar, bar shift =-2.5pt, red,
]
plot coordinates {
(02, 0)
(10, 0)
(06, 0)
(07, 0)
(03, 0)
(01, 3)
(09, 0)
(05, 0)
(08, 0)
(04, 0)
};
\label{plot:properties_hff_bu_21}
\addplot+[ybar, bar shift =-1.0pt, blue,
]
plot coordinates {
(02, 1)
(10, 0)
(06, 0)
(07, 0)
(03, 0)
(01, 4)
(09, 0)
(05, 0)
(08, 0)
(04, 0)
};
\label{plot:properties_hff_td_21}
\addplot+[ybar, bar shift =1.0pt, darkgreen,
]
plot coordinates {
(02, 2)
(10, 1)
(06, 1)
(07, 1)
(03, 1)
(01, 3)
(09, 1)
(05, 1)
(08, 1)
(04, 1)
};
\label{plot:properties_trap_prefop_bu_21}
\addplot+[ybar, bar shift =2.5pt, orange,
]
plot coordinates {
(02, 2)
(10, 0)
(06, 0)
(07, 0)
(01, 3)
(03, 1)
(09, 0)
(05, 1)
(08, 0)
(04, 1)
};
\label{plot:properties_trap_prefop_td_21}

%start node sysW
\addplot+[only marks, mark = -, mark options = {thick}, mark size = 0.2cm, black,
]
plot coordinates {
(01, 3)
(02, 2)
(03, 1)
(04, 2)
(05, 1)
(06, 1)
(07, 1)
(08, 2)
(09, 3)
(10, 2)
};
%optimal
\addplot+[only marks, mark = -, mark options = {thick, red, dashed}, mark size = 0.2cm, black,
]
plot coordinates {
(01, 10)
(02, 10)
(03, 10)
(04, 10)
(05, 10)
(06, 10)
(07, 10)
(08, 10)
(09, 10)
(10, 9)
};
\end{axis}
\hfill


\begin{axis}[
width = 7cm,
height=4cm,
enlarge x limits = 0.1,
enlarge y limits = 0.1,
ybar,
bar width=1pt,
ymin = 0,
ymax = 10,
at={(0.333333333333\linewidth,-480.0)},
compat=1.6,
]
\addplot+[ybar, bar shift =-2.5pt, red,
]
plot coordinates {
(07, 0)
(04, 5)
(10, 0)
(09, 0)
(01, 10)
(03, 7)
(05, 2)
(08, 1)
(06, 1)
(02, 8)
};
\label{plot:properties_hff_bu_21}
\addplot+[ybar, bar shift =-1.0pt, blue,
]
plot coordinates {
(07, 6)
(04, 9)
(01, 10)
(10, 4)
(03, 10)
(05, 9)
(08, 7)
(09, 5)
(06, 7)
(02, 9)
};
\label{plot:properties_hff_td_21}
\addplot+[ybar, bar shift =1.0pt, darkgreen,
]
plot coordinates {
(07, 0)
(04, 0)
(10, 0)
(03, 0)
(01, 2)
(05, 0)
(08, 0)
(09, 0)
(06, 0)
(02, 1)
};
\label{plot:properties_trap_prefop_bu_21}
\addplot+[ybar, bar shift =2.5pt, orange,
]
plot coordinates {
(07, 0)
(10, 0)
(03, 1)
(04, 0)
(01, 2)
(05, 0)
(08, 0)
(09, 0)
(06, 1)
(02, 1)
};
\label{plot:properties_trap_prefop_td_21}

%start node sysW
\addplot+[only marks, mark = -, mark options = {thick}, mark size = 0.2cm, black,
]
plot coordinates {
(01, 2)
(02, 1)
(03, 1)
(04, 0)
(05, 0)
(06, 1)
(07, 0)
(08, 0)
(09, 0)
(10, 0)
};
%optimal
\addplot+[only marks, mark = -, mark options = {thick, red, dashed}, mark size = 0.2cm, black,
]
plot coordinates {
(01, 10)
(02, 10)
(03, 10)
(04, 10)
(05, 10)
(06, 10)
(07, 10)
(08, 10)
(09, 10)
(10, 9)
};
\end{axis}
\hfill


\begin{axis}[
width = 7cm,
height=4cm,
enlarge x limits = 0.1,
enlarge y limits = 0.1,
ybar,
bar width=1pt,
ymin = 0,
ymax = 10,
at={(0.666666666667\linewidth,-480.0)},
compat=1.6,
xlabel=\#properties
]
\addplot+[ybar, bar shift =-2.5pt, red,
]
plot coordinates {
(04, 9)
(05, 5)
(06, 2)
(02, 10)
(07, 0)
(10, 0)
(08, 2)
(03, 10)
(09, 1)
(01, 10)
};
\label{plot:properties_hff_bu_21}
\addplot+[ybar, bar shift =-1.0pt, blue,
]
plot coordinates {
(04, 10)
(05, 10)
(06, 10)
(02, 10)
(10, 9)
(07, 9)
(08, 9)
(03, 10)
(09, 9)
(01, 10)
};
\label{plot:properties_hff_td_21}
\addplot+[ybar, bar shift =1.0pt, darkgreen,
]
plot coordinates {
(04, 0)
(05, 0)
(06, 0)
(02, 0)
(10, 0)
(07, 0)
(08, 0)
(03, 1)
(09, 0)
(01, 0)
};
\label{plot:properties_trap_prefop_bu_21}
\addplot+[ybar, bar shift =2.5pt, orange,
]
plot coordinates {
(04, 0)
(03, 1)
(05, 0)
(08, 0)
(06, 0)
(02, 1)
(10, 0)
(07, 0)
(09, 0)
(01, 2)
};
\label{plot:properties_trap_prefop_td_21}

%start node sysW
\addplot+[only marks, mark = -, mark options = {thick}, mark size = 0.2cm, black,
]
plot coordinates {
(01, 2)
(02, 1)
(03, 1)
(04, 0)
(05, 0)
(06, 0)
(07, 0)
(08, 0)
(09, 0)
(10, 0)
};
\label{plot:baseline_sysW_node}
%optimal
\addplot+[only marks, mark = -, mark options = {thick, red, dashed}, mark size = 0.2cm, black,
]
plot coordinates {
(01, 10)
(02, 10)
(03, 10)
(04, 10)
(05, 10)
(06, 10)
(07, 10)
(08, 10)
(09, 10)
(10, 9)
};
\label{plot:baseline_lmcut}
\end{axis}
\hfill

\node[] (c1) at (2.5,2.8) {\small $x=1.0$};
\node[] (c15) at (8.4,2.8) {\small $x=1.5$};
\node[] (c2) at (14.3,2.8) {\small $x=2.0$};

\node[draw] (test) at (8,-11) {
\ref{plot:properties_hff_bu_21} $\hff$ SysS
\ref{plot:properties_hff_td_21} $\hff$ SysW
\ref{plot:properties_trap_prefop_bu_21} trap SysS
\ref{plot:properties_trap_prefop_td_21} trap SysW
\ref{plot:baseline_lmcut} \hlmcut
\ref{plot:baseline_sysW_node} trap
};

\end{tikzpicture}
\hfill

\caption{blocksworld}
\end{figure*}
