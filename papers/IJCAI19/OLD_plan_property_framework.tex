\section{Plan-Property Dependency Framework}

\joerg{re-think terminology / TBD}

\joerg{think through different planning frameworks, in particular ones
  with continous phenomena, like time; everything below remains
  finite, n particular the BPDG and PDG; what becopmes harder or
  undecidable though is checking validity of a plan-property
  dependeny. probably discuss this below.}

Planning task $\cal T$ is a tuple one of whose elements is a finite
set $A$ of actions. A plan in $\cal T$ is any structure over $A$ that
serves as a specification of what to do (e.g. a sequence or partially
ordered set). In what follows, by $\Pi$ we denote a set of plans of
interest, for example all those plans that are applicable in the
initial state, or those that are applicable in the initial state and
achieve some goal. Note: we do not restrict plans to achieve a goal,
it could be arbitrary action sequences.


\subsection{Plan Properties and their Dependencies}


%% What is a plan property?
%
%%     Plan property: polytime-computable Boolean fn of task Pi and applicable action sequence vec_a

\begin{definition}[Plan Property]
Given a planning task $\cal T$. A \emph{plan property} $f$ in $\cal T$
is any function $f : \Pi \mapsto \{true, false\}$ that can be computed
in time polynomial in the size of its input.
\end{definition}

\joerg{discuss/motivate why this makes sense}


%% What is a plan-property dependency?
%
%%     Basic dependency: p entails q if, for every applicable action
%%     sequence vec_a, if p(veca)=true then q(veca)=true, We say in
%%     this case that the basic dependency "p entails q" is valid in
%%     Pi.
%
%%     (Complex/general) dependency: propositional formula phi over
%%     plan properties. phi holds if, for every applicable action
%%     sequence vec_a, phi evaluates to true on vec_a. We say in this
%%     case that phi is valid in Pi. Basic dependencies are the
%%     special case "p ==> q". Motivation: this could be inside a
%%     single new plan prop. However, plan props are black boxes to
%%     the framework; in the exploration of composed dependencies,
%%     i.e. when solving the explanation problem specified below, the
%%     explanation facility will establish the interesting
%%     combinations across these blackboxes.

\begin{definition}[Plan-Property Dependency]
Given a planning task $\cal T$ and a finite set $F$ of plan properties
in $\cal T$. A \emph{plan-property dependency} $\phi$ is a
propositional formula over the atoms $\{p_f \mid f \in F\}$. We say
that a plan $\pi$ \emph{satisfies} $\phi$, written $\pi \models \phi$,
if $\phi$ evaluates to true under the truth-value assignment
$\{(p_f,f(\pi)) \mid f \in F\}$. We say that $\phi$ is \emph{valid} in
$\cal T$ if it is satisfied by all $\pi \in \Pi$. 

We say that $\phi$ is \emph{basic} if it takes the form $f \implies
g$, for $f, g \in F$.
\end{definition}

\joerg{discuss/motivate why this makes sense; acknoweledge that
  potentially very high complexity to decide validity (corresponds to
  proving unsolability in classical planning)}


\joerg{show that dependencies are more than standard propositional
  logic: even if a formula is not universally true, it may be true for
  all action sequences due to the underlying planning task. In fact
  this will typically be the case, e.g. in a simple mutex where
  reaching one fact A excludes reaching another fact B.}

\joerg{point out that the set $\Pi$ of interest can be the subset of
  applicable action sequences that achieve a set of ``hard/fixed''
  plan properties; we will make use of this in our concrete
  instantiation and experiments}

\joerg{propositional logic could be inside an individual plan
  property. However, plan props are black boxes to the framework; in
  the exploration of composed dependencies, i.e. when solving the
  plan-property axiomatization problem specified below, the
  explanation facility will establish the interesting combinations
  across these blackboxes.}




\subsection{Our Plan-Space Explanation Problems}



%% basic: find all pairwise dependencies, ie, compute the graph whose
%% nodes are the properties and whose edges are the basic dependencies

\begin{definition}[Plan-Property Dependency Graph (PDG)]
Given a planning task $\cal T$ and a set $F$ of plan properties in
$\cal T$. The \emph{plan-property dependency graph (PDG)} is the
directed graph whose vertices are $F$ and that contains an arc $f
\rightarrow g$ whenever the basic dependency $f \implies g$ is valid
in $\cal T$.
\end{definition}

\joerg{motivate usefulness, discuss complexity (quadratic number of
  unsolvability proofs).}


%% Entailment over two dependencies phi and psi: phi entails psi if, for
%% every action sequence, if phi evaluates to true then so does psi. Note
%% that this is equivalent to validity of the dependency "phi ==> psi".

\begin{definition}[Plan-Property Entailment]
Given a planning task $\cal T$, a set $F$ of plan properties in $\cal
T$, and two plan-property dependencies $\phi, \psi$. 

We say that $\phi$ \emph{entails} $\psi$ if, for every plan $\pi \in
\Pi$, if $\pi \models \phi$ then $\pi \models \psi$. We say that
$\phi$ and $\psi$ are \emph{equivalent} if they entail each other; we
denote by $[\phi]$ the equivalence class of $\phi$.
\end{definition}

\joerg{note that this is equivalent to validity of the dependency
  $\phi \implies \psi$, and note that it is not the same as standard
  propsitional entailment, again due to the underlying restriction of
  interpretations to $\Pi$. ... also note that entailment over
  equivalence clasyses remains a partial order as in standard
  propositional logic}




%% general: find all strongest dependencies: the root nodes in the DAG
%% over equivalence classes of dependencies

\joerg{$\Phi$: plan-property dependencies of interest; default can be
  the set of all formulas over $F$ so that user has to fix only
  $F$. But more restricted $\Phi$ ie particular classes of formulas,
  can be of interest too (and will be in our goal-fact analysis in the
  next section)}

\begin{definition}[Plan-Property Dependency Axiomatization (PDA)]
Given a planning task $\cal T$ and a set $\Phi$ of plan-property
dependencies in $\cal T$. 

Let $O$ be the partial order over the equivalence classes $[\phi]$
where $\phi \in \Phi$ is valid in $\cal T$, and where $[\phi] <
[\psi]$ iff $\phi$ entails $\psi$.
%
A set $\Phi[{\cal T}] \subseteq \Phi$ is a \emph{plan-property
  dependency axiomatization (PDA)} of $\Phi$ for $\cal T$ if, for
every minimal element $[\phi_r]$ of $O$, $\Phi[{\cal T}]$ contains
exactly one member of $[\phi_r]$.
\end{definition}

\joerg{discuss/motivate why this makes sense; discuss complexity,
  point out encouraging results with goal-fact dependencies in our
  experiments.}

\joerg{add some remarks regarding ``axiomatization'' terminology and
  latticew/properties,m see detailed tex comments here}
%
%% \joerg{reg terminology: ``axiomatization'' captures this well I think,
%%   in the sense that $\Pi$ basically identifies a subset of truth-value
%%   assignments to $F$, ie, of logical interpretations of our
%%   propositional variables; we can view this subset of interpretations
%%   as the possible worlds given $\Pi$. Then $O$ is the entailment order
%%   over the things that hold in all possible worlds; and a set of
%%   axioms is a smallest set of formulas that fully characterizes these
%%   things.}
%
%% \joerg{reg lattices: $O$ is a ``meet-semilattice'' if $\Phi$ is closed
%%   under conjunction: in a lattice, every two elements need to have a
%%   common ancestor. in standard logic, entailment over formula
%%   equivalence classes is a lattice because for any $\phi$ and $\psi$
%%   $\phi \wedge \psi$ is a common ancestor; actually a bounded lattice,
%%   with a unique element that is an ancestor to (that implies)
%%   everything else. the same is true here if $\Phi$ is closed under
%%   conjunction (because if each of $\phi$ and $\psi$ is valid, then so
%%   is $\phi \wedge \psi$). In this case, the PDA is the unique common
%%   ancestor, which can basically be thought of as a conjunction of
%%   axioms. In lattice terminology, the unique common ancestor is called
%%   a ``least element'', ``minimum'', or ``bottom'' element; but I think
%%   none of these names makes much sense in our context so I would leave
%%   it at the above definition, maybe briefly remarking here that $O$ is
%%   a meet-semilattice if $\Phi$ is closed under conjunction.}

\joerg{discuss relation to model checking here somewhere? discuss
  possibility to automatically identify relevant $F$ and $\Phi$?}

\joerg{add definition and discussion of ``derivability'' (or some
  such) across PDASs, where PDA for $\Phi_B$ can be derived from PDA
  for $\Phi_A$ as in our two different goal-fact dependenciues in the
  next section.}
