\section{Goal Dependencies in Oversubscription Planning}
\label{goaldep}

%% \joerg{0.75 -- 1 page(s) joerg}

We now instantiate our framework with a concrete use case:
dependencies between goals in oversubscription planning, where the
question addressed is which combinations of (soft) goals exclude which
other combinations. In Section~\ref{compilation}, we will show how to
compile a more powerful plan property language into this special case.




\subsection{Planning Framework}
\label{goaldep:planning}

Most of the techniques we introduce in what follows are applicable to
a broad range of planning frameworks. Nevertheless, for a concrete
exposition, henceforth we consider the \emph{finite-domain
  representation (FDR)} framework
\cite{backstrom:nebel:ci-95,helmert:ai-09}, with finite-domain state
variables as used in the Fast Downward system \cite{helmert:jair-06}
on which our implementation is based.

An FDR task \defined{\task} is a tuple $\task =
(\vars,\acts,\cost,\init,\goal)$ where \vars\ is the set of
\defined{variables}, \acts\ is the set of \defined{actions}, $\cost:
\acts \mapsto \reals^+_0$ is the action \defined{cost} function,
\init\ is the \defined{initial state}, and \goal\ is the
\defined{goal}. A \defined{state}, in particular \init, is a complete
assignment to $\vars$; \goal\ is a partial assignment to \vars; each
action $a \in \acts$ has a \defined{precondition} $\pre_a$ and an
effect $\eff_a$, both partial assignments to \vars. We will refer to
variable-value pairs $v=d$ as \defined{facts}, and we will identify
partial variable assignments with sets of facts.
%
An action $a$ is \defined{applicable} in a state $s$ if $\pre_a
\subseteq s$. The outcome state $s\apply{a}$ is like $s$ except that
$s\apply{a}(v) = \eff_a(v)$ for those $v$ on which $\eff_a$ is
defined. The outcome state of an iteratively applicable action
sequence $\plan$ is denoted $s\apply{\plan}$.

We address an oversubscription variant of FDR, where an
\defined{oversubscription planning (OSP) task} is a tuple $\task =
(\vars,\allowbreak\acts,\allowbreak\cost,\allowbreak\init,\allowbreak\goal,\allowbreak\costbound)$
exactly like an FDR task but with an additional \defined{cost bound}
$\costbound \in \reals^+_0$. Intuitively, the goals \goal\ are
``soft'', and the challenge is to achieve a maximally valuable subset
of \goal\ within the cost bound. OSP frameworks in the literature
employ notions (\eg\ goal-fact rewards) of what it means to be
``maximally valuable''
\cite{smith:icaps-04,domshlak:mirkis:jair-15}. Here we assume instead
that the user's preferences over the soft goals are difficult to
specify and/or elicitate, so that an in-depth characterization of the
trade-offs between different goal sets -- their dependencies -- is of
interest. In the terms of our framework, this means that the set
\plans\ of \defined{plans} is simply the set of all action sequences
$\plan = \langle a_1, \dots, a_n\rangle$ applicable in \init\ and
where $\sum_{i=1}^n \cost(a_i) \leq \costbound$. An analysis over
suitable sets of properties \props\ and dependencies \deps\ then
yields the desired trade-off information.






\subsection{Plan Properties}
\label{goaldep:properties}

The plan properties we consider here are characterized by
propositional formulas over goals:

\begin{definition}[Goal Properties]
Let $\task =
(\vars,\allowbreak\acts,\allowbreak\cost,\allowbreak\init,\allowbreak\goal,\allowbreak\costbound)$
be an OSP task, and \plans\ its set of plans. 

An \defined{atomic goal property} for \task\ is a function $\prop_g :
\plans \mapsto \{\true, \false\}$ where $g \in \goal$ and
$\prop_g(\plan) = \true$ iff $g \in \init\apply{\plan}$.
%
A \defined{composed goal property} for \task\ is a function
$\prop_\phi : \plans \mapsto \{\true, \false\}$ where $\phi$ is a
propositional formula over the atoms \goal, and $\prop_\phi(\plan) =
\true$ iff $\phi$ evaluates to true given the values of
$\prop_g(\plan)$.
\end{definition}

We identify goal properties $\prop_\phi$ with the characterizing
formulas $\phi$. We consider a class of properties and dependencies
identifying exclusions between goal conjunctions:

\begin{definition}[Goal Exclusion]
Let $\task =
(\vars,\allowbreak\acts,\allowbreak\cost,\allowbreak\init,\allowbreak\goal,\allowbreak\costbound)$
be an OSP task, and \plans\ its set of plans.  

The \defined{PDO for goal exclusion (PDO-GE)} is the PDO for
\plans, the property set $\geprops := \{\bigwedge_{a \in A}
g\allowbreak\mid\allowbreak A \subseteq G\} \cup \{\neg \bigwedge_{g
  \in B} b \allowbreak\mid\allowbreak B \subseteq G\}$, and the
dependency set $\gedeps := \{(\bigwedge_{a \in A} a,\allowbreak\neg
\bigwedge_{b \in B} b) \allowbreak\mid\allowbreak A \cap B =
\emptyset\}$.
\end{definition}

We restrict focus to goal conjunctions and negations thereof, and we
are interested only in implications of the form
$\entails{\plans}{\bigwedge_{a \in A} a}{\neg \bigwedge_{b \in B} b}$
stating that, if we achieve all of $A$, we have to forego at least one
of $B$. The PDO-GE then explains to the user how exactly different
goal subsets exclude each other, identifying the fine-grained
trade-off.

Observe that, given the restriction to $\gedeps$, the equivalence
classes in the PDO-GE are singletons. Hence there is a unique cPDO-GE,
that we identify with the PDO-GE itself.

For compacting the information presented to a user, we use the
sufficient criterion for entailment where $\bigwedge_{a \in A'} g
\entailsuff \bigwedge_{a \in A} a$ iff $A' \supseteq A$ and $\neg
\bigwedge_{b \in B} b \entailsuff \neg \bigwedge_{b \in B'} g$ iff $B
\subseteq B'$. The dominant PDO-GE thus selects the entailments with
minimal left-hand side conjunctions excluding minimal right-hand side
conjunctions.
%
%The question remains how to compute the dominant PDO-GE.

We remark that another class of dependencies that may be of interest
is \defined{disjunctive goal exclusion}, that take the form
$\entails{\plans}{\bigwedge_{a \in A} a}{\neg \bigvee_{b \in B} b}$,
where the strongest dependencies have small left-hand side
conjunctions but large right-hand side disjunctions. Observe that the
dominant PDO for such dependencies can be derived from the dominant
PDO-GE: $\entails{\plans}{\bigwedge_{a \in A} a}{\neg \bigvee_{b \in
    B} b}$ holds if and only if $\entails{\plans}{\bigwedge_{a \in A}
  a}{\neg b}$ holds for every $b \in B$.





\subsection{Computing the Dominant PDO-GE}
\label{goaldep:computing}

The dominant PDO-GE can be read off the \defined{minimal unsolvable
  goal subsets (MUGS)}, where $\goal' \subseteq \goal$ is a MUGS if
$\goal'$ cannot be achieved but every $\goal'' \subsetneq \goal'$ can:

\begin{proposition}[PDO-GE from MUGS]
Let $\task =
(\vars,\allowbreak\acts,\allowbreak\cost,\allowbreak\init,\allowbreak\goal,\allowbreak\costbound)$
be an OSP task, and \plans\ its set of plans.  

Then $\entails{\plans}{\bigwedge_{a \in A} a}{\neg \bigwedge_{b \in B}
  b}$ is in the dominant PDO-GE if and only if $A \cup B$ is a MUGS.
\end{proposition}

\begin{proof}
A \plans-entailment $\entails{\plans}{\bigwedge_{a \in A} a}{\neg
  \bigwedge_{b \in B} b}$ clearly holds iff $A \cup B$ is
unsolvable. Dominant entailments in the PDO-GE result from
set-inclusion minimal $A$ and $B$, corresponding to the set-inclusion
minimality of MUGS.
\end{proof}


%% \joerg{mention relation of minimum unsolvable goal subsets/of PDE-GE
%%   to pareto frontier somewhere?}

%% \joerg{mention alternate bottom-up computation, starting at empty goal
%%   set and adding goals til unsat. bottom-up alg generates set of
%%   across-line unsolvabl sets; set-inclusion minimal amongst those are
%%   the minimal unsolvable sets (they are unsolvable; any subset must be
%%   solvable by construction; vice versa, anz minimal unsolv set results
%%   from adding a goal to a maximal unsolvable one, so all the
%%   maxunssets are there)}
%
% Joerg email top-down vs bottom-up
%
%% Both "top down" as we do now and "bottom up" as you suggest can be
%% used to identify the minimum unsolvable goal sets (in the bottom-up
%% case a post-processing step is needed, selecting from the
%% unsolvable goal sets generated during meta-tree search the
%% set-inclusion minimal ones; but this step is easy ie polynomial in
%% the number of goal sets considered).
%
%% Which of the two is expected to work better is largely a function
%% of how large the solvable goal subsets will be. It seems plausible
%% that one or the other algorithm will often have substantial
%% advantages. In particular, for severely limited resources one would
%% expect bottom-up to be better.
%
%% So ideally we should experiment with both top-down and
%% bottom-up. And in practice it would seem quite reasonablee to run
%% both in parallel, so that a "best-of" also makes a lot of sense
%% here.
%
%% (One can even think about interleaving bottom-up with top-down in
%% an intelligent way, but that's for future work)

Our computational problem thus boils down to computing all MUGS. This
can be done through a search over goal sets, that we refer to as
\defined{systematic weakening (SysW)}:
\begin{enumerate}[(1)]
\item the start node of the search is \goal; 
\item each search step selects an open node $\goal'$, calls a planner
  to test whether $\goal'$ is solvable in \task, caches the result,
  and expands $\goal'$ if it is unsolvable;
\item the children of a node $\goal'$ are those $\goal'' \subset
  \goal'$ where $|\goal''| = |\goal'|-1$.
\end{enumerate}
Upon termination, the MUGS are those nodes $\goal'$ all of whose
children are solvable. 

Dually, \defined{systematic strenghtening (SysS)} starts from
$\emptyset$, with search steps expanding solvable nodes, and children
adding one more goal fact. Upon termination, the MUGS can be easily
obtained from the unsolvable search nodes.

% Rebecca's notes on the implementation:
%
%% We fix the order of the goal facts: [g0, g1, …, gn]
%% Then each goal subsets gets an unique id based on the contained goal facts:
%% For example the subset [g0, g2] has in binary the id 101. The set bits indicate which goal facts are contained. (This limits our algorithm to 31 soft goals.)
%
%% During search we cache all results and prune already solved goal subsets based on their id.
%% The ids induce an expansion order on all possible nodes/subset, such that you always know if node with id1 has been solved before node with id2.
%
%% If we expand a node in the meta search there exist three possible states for the children
%% (which can easily be generated by setting one of the 1 bits of the parent id to 0 parent: 101 → children 001,100):
%
%% 1: the child node is in the expansion order after its parent → add the node to the openlist
%
%% 2. the child node is in the expansion oder before its parent:
%
%%     2.1: the node with the given id exists already→ use this result
%
%%     2.2: the node does not exists
%%         reason: a superset of the goal facts has been proved solvable and therefore is not expanded further
%%         → you know that this node is solvable without running the planner

In both SysW and SysS, every goal set can be reached from the start
node by every permutation of goal-fact removal/addition steps. We
avoid duplicate planner calls by caching. We give goal sets unique
integer IDs, which serve for fast cache lookup, and which we also use
to fix the expansion order, so that we always know whether or not we
have generated a node before.
%
We furthermore create synergy with recent nogood learning techniques,
\emph{\hc-conjunction learning} \cite{steinmetz:hoffmann:ai-17} and
\emph{trap learning} \cite{steinmetz:hoffmann:ijcai-17}, which refine
dead-end detection methods (nogoods) based on the unsolvable states
encountered in state space search on a planning task. As the children
tasks in our searches are closely related to their parents, the
refined nogoods are likely to be useful still. So we
\defined{transfer} the nogoods along search paths, resulting in
iteratively stronger and stronger nogoods. For both \hc-conjunction
learning and trap learning, and for both SysW and SysS, simple methods
suffice to identify those nogoods that remain valid in a child task.
%
%% We furthermore identify non-trivial synergy with recent nogood
%% learning techniques \cite{steinmetz:hoffmann:ai-17}, which refine a
%% critical-path heuristic \hc\ during forward search on a task \task,
%% based on the unsolvable states encountered. The resulting heuristic
%% \hc\ reasons about an enlarged set of conjunctions, in a manner geared
%% at quickly identifying unsolvable states in \task. Observe that this
%% naturally lends itself to \defined{\hc-transfer} in systematic
%% weakening: the refined \hc\ from an unsolvable node $\goal'$ is likely
%% to be a good unsolvability detector for the closely related children
%% tasks with goals $\goal'' \subset \goal'$ where $|\goal''| =
%% |\goal'|-1$. We thus transfer the learned \hc\ functions downwards
%% along the search paths in systematic weakening, iteratively yielding
%% stronger and stronger unsolvability detectors.
%% %
%% % Joerg: proably more confusing than helpful
%% %
%% %% Similarly, we transfer the \defined{clauses} (nogood pruning rules)
%% %% learned in this approach, transferring thjose that depend only on the
%% %% goals still present in children nodes.









%% \subsection{Disjunctive Goal Exclusion}
%% \label{goaldep:disjunctive}

%% \joerg{this plays a very minor role in the paper. hence I
%%   briefly merged it into the above discussion. remove this section for
%%   paper}

%% Another class of dependencies that may be of interest is
%% \defined{disjunctive goal exclusion}:

%% \begin{definition}[Disjunctive Goal Exclusion]
%% Let $\task =
%% (\vars,\allowbreak\acts,\allowbreak\cost,\allowbreak\init,\allowbreak\goal,\allowbreak\costbound)$
%% be an OSP task, and \plans\ its set of plans.  

%% The \defined{PDO for disjunctive goal exclusion (PDO-DGE)} is the PDO for
%% \plans, the property set $\dgeprops := \{\bigwedge_{a \in A}
%% g\allowbreak\mid\allowbreak A \subseteq G\} \cup \{\neg \bigvee_{g
%%   \in B} b \allowbreak\mid\allowbreak B \subseteq G\}$, and the
%% dependency set $\dgedeps := \{(\bigwedge_{a \in A} a,\allowbreak\neg
%% \bigwedge_{b \in B} b) \allowbreak\mid\allowbreak A \cap B =
%% \emptyset\}$.
%% \end{definition}

%% Observe that $\neg \bigvee_{b \in B} b$ is equivalent to $\bigwedge_{b
%%   \in B} \neg b$, so that $\entails{\plans}{\bigwedge_{a \in A}
%%   a}{\neg \bigvee_{b \in B} b}$ holds iff
%% $\entails{\plans}{\bigwedge_{a \in A} a}{\neg b}$ holds for every $b
%% \in B$. Indeed, the dominant PDO-DGE can be derived from the dominant
%% PDO-GE. Precisely, say that an entailment
%% $\entails{\plans}{\bigwedge_{a \in A} a}{\neg \bigvee_{b \in B} b}$ is
%% \defined{constructible} from the dominant PDO-GE if there exist $A[b]
%% \subseteq A$, for $b \in B$, such that $\entails{\plans}{\bigwedge_{a
%%     \in A[b]} a}{\neg b}$ is in the dominant PDO-GE and $\bigcup_{b
%%   \in B} A[b] = A$. We have:

%% \begin{proposition}[PDO-DGE from PDO-GE]\label{pro:pdodge-from-pdoge}
%% Let $\task =
%% (\vars,\allowbreak\acts,\allowbreak\cost,\allowbreak\init,\allowbreak\goal,\allowbreak\costbound)$
%% be an OSP task, and \plans\ its set of plans.  

%% Then $\entails{\plans}{\bigwedge_{a \in A} a}{\neg \bigvee_{b \in B}
%%   b}$ is in the dominant PDO-DGE if and only if it is constructible
%% from the dominant PDO-GE, and there is no $A' \subsetneq A$ such that
%% $\entails{\plans}{\bigwedge_{a \in A'} a}{\neg \bigvee_{b \in B} b}$
%% is constructible from the dominant PDO-GE.
%% \end{proposition}

%% \begin{proof}
%% %
%% Observe first that, for any entailment $\entails{\plans}{\bigwedge_{a
%%     \in A} a}{\neg \bigvee_{b \in B} b}$ that holds, an entailment
%% $\entails{\plans}{\bigwedge_{a \in A'} a}{\neg \bigvee_{b \in B} b}$
%% with $A' \subseteq A$ is constructible from the dominant PDO-GE. This
%% is because, for every entailment $\entails{\plans}{\bigwedge_{a \in A}
%%   a}{\neg b}$ that holds, the dominant PDO-GE contains an entailment
%% $\entails{\plans}{\bigwedge_{a \in A[b]} a}{\neg b}$ with $A[b]
%% \subseteq A$.

%% The 'only if' direction follows directly from this. For the 'if'
%% direction, it suffices to observe additionally that, trivially, every
%% entailment constructible from the dominant PDO-GE does in fact hold.
%% %
%% % Joerg: outdated attempt
%% %
%% %% For the 'if' direction, say that $\entails{\plans}{\bigwedge_{a \in A}
%% %%   a}{\neg b_0}$ is in the dominant PDO-GE, and that
%% %% $\entails{\plans}{\bigwedge_{a \in A[b]} a}{\neg b}$ is in the
%% %% dominant PDO-GE for all $b_0 \neq b \in B$ with $A[b] \subseteq
%% %% A$. Then, trivially, $\entails{\plans}{\bigwedge_{a \in A} a}{\neg
%% %%   \bigvee_{b \in B} b}$. This entailment is dominant because 1)
%% %% $\entails{\plans}{\bigwedge_{a \in A} a}{\neg b_0}$ is dominant so $A$
%% %% cannot be smaller, and 2) any valid implication of a literal $\neg b$
%% %% is represented in the dominant PDE-GE so $B$ cannot be larger.
%% %
%% %% 'only if': Assume for contradiction that
%% %% $\entails{\plans}{\bigwedge_{a \in A} a}{\neg \bigvee_{b \in B} b}$ is
%% %% in the dominant PDO-DGE but this claim does not hold. If there exists
%% %% $b \in B$ for which no entailment $\entails{\plans}{\bigwedge_{a \in
%% %%     A[b]} a}{\neg b}$ is in the dominant PDO-GE, then $\bigwedge_{a
%% %%   \in A} a$ does not entail $\neg \bigvee_{b \in B} b$, in
%% %% contradiction. If $A[b] \subsetneq A$ for every $b$.
%% \end{proof}

%% Observe that the dominant PDO-DGE may be exponentially larger than the
%% dominant PDO-GE: in the worst case, every combination of
%% $\entails{\plans}{\bigwedge_{a \in A} a}{\neg b}$ entailments in the
%% dominant PDO-GE yields a different set-inclusion minimal left-hand
%% side. However, using Proposition~\ref{pro:pdodge-from-pdoge}, the
%% dominant PDO-DGE can be constructed from the dominant PDO-GE in time
%% polynomial in the size of the output, simply by iteratively
%% enumerating all entailments constructible from the dominant PDO-GE and
%% checking for subsumptions.









































































%%%%%%%%%%%%%%%%%%%%%%%%%%%%%%%%%%%%%%%%%%%%%%%%%%%%%%%%%%%%%%%%%%%%%
%%%%%%%%%%%%%%%%%%%%%%%%%%%%%%%%%%%%%%%%%%%%%%%%%%%%%%%%%%%%%%%%%%%%%
%%%%%%%%%%%%%%%%%%%%%%%%%%%%%%%%%%%%%%%%%%%%%%%%%%%%%%%%%%%%%%%%%%%%%
%%% JOERG: MY INITIAL NOTES


%% \joerg{question: does our method compute the PDO alongside the PDA?
%%   and is the PDA unique?} for 1.: We compute a cPDA: First, say G'
%%   is maximal satisfiable; we prove that [G'] is in the cGPA. Assume
%%   satisfiable G'' entails G'. Then G' \cup G'' is satisfiable, so
%%   we must have G'' \subseteq G' by maximality of G'. But then, G'
%%   entails G'' so they're equivalent. Vive versa, say that [G'] is
%%   minimal satisfiable in PDO, and say that G'_0 is a set-inclusion
%%   maximal superset of G' in [G']; then there is no satisfiable
%%   superset G'' of G'_0 as otherwise [G'] would not be minimal; so
%%   G'_0 is maximal satisfiable. Finally, each two maximal
%%   satisfiable sets belog to different members of the PDA: if G'
%%   \neq G'' maximal satisfiable, then neither implies the other as
%%   otherwise G' \cup G'' would be larger satisfiable. We do not
%%   compute the PDO as that may involve entailments due to planning
%%   semantics, eg q entails p if the only way to achieve q is via
%%   achieving p first.



%% conjunctive exclusion properties $\props^{CE}$, $\bigwedge_{a \in A} a
%% \wedge \bigwedge_{b \in B} b$; 3. disjunctive exclusion properties
%% $\props^{DE}$, $\bigwedge_{a \in A} a \wedge \bigvee_{b \in B}
%% g$. (alternative: take elements of intended exclusion implications as
%% properties instead. don't do here as we don't actually compute a PDA
%% for these, there may be other implications we don't see. general: if
%% implications $A \rightarrow B$ are of interest, can use properties $A
%% \wedge \neg B$ where PDA identifies maximal such implications ie where
%% A is weakest and $\neg B$ is strongest. mention this explicitly in
%% this subsec? introduce explusion implications of interest first, then
%% specify encoding of these as plan props and cPDA analysis on those?

%% Prove
%% that the PDA for $\props^{G} \cup \props^{CE}$ and $\props^{G} \cup
%% \props^{DE}$ are polynomially derivable from that (by specifying the
%% required derivation functions $f$ and proving them correct).






%% Initialize $\cal G :+ \emptyset$; assume $\goal = \{g_1, \dots,
%% g_n\}$; call Recursive-PDO-CE($\goal$, $1$):

%% \begin{tabbing}
%% Bo\=ol SystematicWeakening($G'$, $i$) \\
%% \> \textbf{if} $G'$ is solvable in \task\ \textbf{then return \true\ endif}\\
%% \> flag := \true\\
%% \> \textbf{for} \= $j = i \dots n$ where $g_j \in G'$ \textbf{do}\\
%% \> \> \textbf{if} \= not Recursive-PDO-CE($G' \setminus \{g_j\}$, $j+1$) \textbf{then}\\
%% \> \> \> flag := \false\\
%% \> \> \textbf{endif}\\
%% \> \textbf{endfor}\\
%% \> \textbf{if} flag \textbf{then} $\cal G := \cal G \cup \{G'\}$\\
%% \> return \false
%% \end{tabbing}











%%%%%%%%%%%%%%%%%%%%%%%%%%%%%%%%%%%%%%%%%%%%%%%%%%%%%%%%%%%%%%%%%%%%%
%%%%%%%%%%%%%%%%%%%%%%%%%%%%%%%%%%%%%%%%%%%%%%%%%%%%%%%%%%%%%%%%%%%%%
%%%%%%%%%%%%%%%%%%%%%%%%%%%%%%%%%%%%%%%%%%%%%%%%%%%%%%%%%%%%%%%%%%%%%
%%% INITIAL TEXT IN THIS SECTION


%% \subsection{???}

%% \rebecca{start with a concrete example, in the truck domain}

%% \emph{If goal subset A is true at the end of the plan, then at least one element of goal subset B
%% must be false at the end of the plan.}

%% \rebecca{if we are looking fore the minimal sets $A \cup B$ than B always contains 
%% only one element}

%% \begin{definition}[Conjunctive Exclusion]
%% Given a planning task $\Pi = (V,A,c,I,G)$ $(A,B)$ with $A, B \subseteq G$ is a 
%% GFD1 if $\Pi$ with the goal $A \cup B$, i.e., $\bigwedge_{p \in A \cup B} p$
%% is unsolvable.
%% The subsumption relation is given by $\forall A,B,A',B': (A,B) \leq (A',B')$ iff $A \cup B \subseteq 
%% A' \cup B'$
%% \end{definition}

%% \paragraph{Algorithm}
%% All minimal GFD1s can be computed through a search tree that starts at node $N_0$ containing
%% all goal facts $G$ and where each search step on a node $N_i$ tests solvability of 
%% $\Pi_i = (V,A,c,I,N_i)$. If $\Pi_i$ is not solvable, we generate one child 
%% node $N'$ for every subset of $N_i$ obtained by removing one fact.  
%% Upon termination all nodes with only solvable children are the \emph{minimal unsolvable 
%% goal subsets (MUGS)}. (A,B) is a minimal GFD1 iff $A \cup B \in \text{MUGS}$, 
%% $A \cap B = \emptyset$ and $|B| = 1$

%% \rebecca{A is always solvable}

%% \rebecca{add example, for fuel level 5}

%% \subsection{???}

%% \textit{If goal subset A is true at the end of the plan, 
%% then ALL elements of goal subset B must be false at the end of the plan.}\\

%% \begin{definition}[Disjunctive Exclusion]
%% 	Given a planning task $\Pi = (V,A,c,I,G)$ the tuples
%% 	$(A,B)$ with $A,B \subseteq G $ is a GFD2 if 
%% 	$\Pi$ with the goal $\bigwedge_{p \in A} p \wedge (\bigvee_{q \in B} q)$
%% 	is unsolvable. 
%% 	The subsumption relation is given by $\forall A,B,A',B': (A,B) \leq (A',B')$ iff $A \subseteq A'$
%% 	and $B \supseteq B'$.
%% \end{definition}	

%% 	\noindent
%% 	Given all minimal GFD1s for $\Pi$ all minimal GFD2s can be derived according 
%% 	to the following relation.

%% 	\rebecca{proof}

%% 	\vspace{-0.3cm}
%% 	\begin{align*}
%% 		GFD2 &:= \{(A,B) | 
%% 				\exists P \in GFD1:(
%% 				   A \subseteq P \wedge |P \setminus A | = 1 \wedge\\
%% 				   &\forall P' \in PPD1:
%% 					  A \subseteq P' \rightarrow P' \setminus A \subseteq B
%% 				)
%% 			 \}
%% 	\end{align*}

%% \paragraph{Algorithm}
%% From the minimal GFD1s we get a set:
%% 	$D = \{(A,B) | \exists P \in GFD1s, p \in P: A = P \setminus p \wedge
%% 	B = \{p\}\}$

%% These GFD2s are not necessary optimal, the B could be larger.

%% \rebecca{find a name for A and B}

%% \noindent
%% To get the maximal set of goals which can not be achieved if we achieve A, 
%% you have to merge all B's which belong to an A' which is a subset of A. 

%% \begin{align*}
%% 	(A, \bigcup_{(A', B') \in \{(A'', B'') \in D | A' \subseteq A\}} B' \cup B)
%% \end{align*}

%% \rebecca{to many ticks}

%% \rebecca{add example, from GFD1 to GFD2}

%% \noindent
%% If the planning task is not solvable for a goal fact at all, you can add 
%% this goal fact to all B's.\\
