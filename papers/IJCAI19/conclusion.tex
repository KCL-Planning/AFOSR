\section{Conclusion}
\label{conclusion}

%% \joerg{use some of the following text pars as a brief post-hoc
%%   discussion of related work? ... probably not no space. domaina
%%   nalysis and model checking are briefly mentioned int he intro; Brian
%%   is briefly mentioned in the goal dependencies section. I don't think
%%   we can do better than this.}
%
%% % current intro text
%% %
%% %% We remark that our approach can be viewed as a generalization of
%% %% domain/task analysis (as done \eg\ by \cite{fox:long:jair-98}), and as
%% %% an instance of model checking applied to planning models (related
%% %% \eg\ to \cite{vaquero:etal:keq-13}). Our contribution lies in
%% %% introducing this intermediate problem suited to XAIP as outlined, and
%% %% instantiating that formulation with initial technology showing promise
%% %% in practice.
%
%% Another alternate view of our approach is as a form of domain analysis
%% (actually: task analysis), identifying particular properties of plan
%% space ahead of time. Indeed, various popular task analyses can be cast
%% as instances of our framework. A fact pair $(p,q)$ is mutually
%% exclusive \cite{blum:furst:ai-97} iff $p$-true-at-end entails $\neg
%% q$-true-at-end in the space of all applicable action sequences; a fact
%% $p$ is a landmark \cite{hoffmann:etal:jair-04} iff $\true$ entails
%% $p$-true-at-some-point; other examples presumably exist. From this
%% point of view, we generalize previous concepts to a broader
%% perspective aimed at addressing arbitrary user questions. At the same
%% time, our approach itself can be viewed as an instance of model
%% checking of planning models \cite{clarke:etal:01},\footnote{There has
%%   been little work on this subject; Vaquero et
%%   al.\ \shortcite{vaquero:etal:keq-13} use Petri nets to capture and
%%   check dynamic aspects of planning models in itSIMPLE.}
%% systematically checking all entailments between plan properties. Again
%% the value of our framework lies in its suitability for XAIP (plus
%% computational gains from considering all \props\ dependencies in
%% unison rather than running individual entailment checks).
%
%% Identifying the trade-off between soft goals in an iterative
%% oversubscription planning process relates, in terms of purpose, to the
%% work by Yu et al.\ \cite{yu:etal:jair-17}. A major difference is that
%% Yu et al.\ address conditional temporal problems, a form of
%% conditional temporal plans, instead of PDDL-style compactly specified
%% general planning problems. This more restricted focus allows to
%% leverage previous conflict analysis methods in that area. It remains a
%% question for future work whether such conflict analysis could inspire
%% more effective analysis methods in our framework.





more general properties than goal dependencies

more concrete use cases (robotics, pentesting) user studies

beyond classical planning

An interesting yet challenging question for future work is how one can
automatically identify relevant plan properties \props\ to analyze.
... mention WHY and HOW questions from NASA discussions? ie WHY
finding additional properties zooming in on a given dependency, HOW
finding a minimal relaxation of enforced props under which a given
dependency disappears.
