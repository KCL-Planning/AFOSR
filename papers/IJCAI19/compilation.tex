\section{Compilations into Goal Dependencies}
\label{compilation}

%% \joerg{there first needs to be a separate definition of
%%   \defined{action constraints}. Then a definition similar to the one
%%   below, that specifies the compilation. Then a formal Proposition
%%   claiming that the compilation is correct: given task \task\ with
%%   induced plan set \plans\ and plan property set \props\ consisting of
%%   goal facts and action constraints, if \task', \plans', and \props'
%%   is the compiled task, induced plan set, and property set, then a PDA
%%   for \plans\ and \props\ can be constructed in polynomial time from
%%   one for \plans' and \props'. In the final paper, the proof can be a
%%   short textual explanation, not in a formal proof environment,
%%   basically just describing how to do this and why it works; but
%%   please first make a more formal detailed proof that explicitly gives
%%   the polynomial time algorithm and argues in detail why indeed the
%%   result is a PDA for \plans\ and \props.}

The analysis of goal properties just described can be used to analyze
more complex properties that can be compiled into goal facts. Given
the well-known power of compilation in planning languages
(\eg\ \cite{gazen:knoblock:ecp-97,nebel:jair-00,edelkamp:icaps-06,palacios:geffner:jair-09,baier:etal:ai-09}),
there is large potential in this idea. As an example, here we consider
what we refer to as action-set properties:

% Joerg: not using atomic/composed here as that would link back in to
% the discussionbefore where we specify onl the atomic props and
% analyze the compiosed ones; but this is niot the case here/in our
% experiments: instead we explicitly list a small set of composed
% properties to be analyzed.
%
\begin{definition}[Action-Set Properties]\label{def:action-set-properties}
Let $\task =
(\vars,\allowbreak\acts,\allowbreak\cost,\allowbreak\init,\allowbreak\goal,\allowbreak\costbound)$
be an OSP task, \plans\ its set of plans, and $\acts_1, \dots, \acts_n
\subseteq \acts$.

An \defined{action-set property} for \task\ and $\acts_1, \dots,
\acts_n$ is a function $\prop_\phi : \plans \mapsto \{\true, \false\}$
where $\phi$ is a propositional formula over the atoms $\acts_1,
\dots, \acts_n$, and $\prop_\phi(\plan) = \true$ iff $\phi$ evaluates
to \true\ under the truth value assignment where $\acts_i$ is
\true\ iff $\plan$ contains at least one action from $\acts_i$.
\end{definition}

As before, we identify action-set properties $\prop_\phi$ with the
characterizing formulas $\phi$. Arguably, action-set properties are
practically relevant. They allow to express things like ``objective x
is covered by satellite y'' (if this is desirable but could be traded
against other soft preferences), ``route X is not used'' (if that
route is problematic, \eg\ due to frequent traffic issues),
``passengers A and B ride in the same vehicle'' (if that is
desirable), etc. At the same time, the simple syntax of action-set
properties lends itself to effective compilation, as follows.

Given \task, \plans, and $\acts_1, \dots, \acts_n$ as in
Definition~\ref{def:action-set-properties}, to obtain a compiled task
$\task'$
\begin{itemize}
\item[1)] introduce Boolean flags $\mathit{isUsed}_i$ that are
  initially false and set to \true\ by any action from $\acts_i$;
\item[2)] introduce formula-evaluation state variables and actions
  evaluating each $\prop_\phi$ based on these (following
  \cite{gazen:knoblock:ecp-97,nebel:jair-00}), setting Boolean flags
  $\mathit{isTrue}_\phi$ storing the outcome values;
\item[3)] introduce separate planning vs.\ formula-evaluation phases
  and a switch action allowing to go from the former to the latter.
\end{itemize}
Then the planning-phase prefixes in $\task'$ are in one-to-one
correspondence with \plans, and given such a prefix \plan\ the
evaluation phase in $\task'$ can achieve $\mathit{isTrue}_\phi$ iff
$\prop_\phi(\plan) = \true$.

Now say that we want to analyze the dependencies across a given set
$\props$ of action-set properties (\eg\ possible undesirable
consequences of not using route X). We are given \task, \plans, and
$\props$; we want to compute the PDO for property exclusion over
$\props$, \ie, the dependencies of the form
$\entails{\plans}{\bigwedge_{\phi \in A} \phi}{\neg \bigwedge_{\psi
    \in B} \psi}$. With the above, this can be done by instead
computing the PDO-GE for $\task'$ with goal set
$\{\mathit{isTrue}_\phi \mid \phi \in \props\}$, and identifying each
$\mathit{isTrue}_\phi$ with $\phi$ in the outcome.

Clearly, similar compilation techniques can in principle be used for
much more powerful property languages, in particular LTL for which
compilations into planning are known
\cite{edelkamp:icaps-06,baier:etal:ai-09}. It remains a question for
future work whether and how this can be made practically effective;
here, we focus on the simpler action-set properties as a first
demonstration of the approach.





%%%%%%%%%%%%%%%%%%%%%%%%%%%%%%%%%%%%%%%%%%%%%%%%%%%%%%%%%%%%%%%%%%%%%%%%%%%
%%%%%%%%%%%%%%%%%%%%%%%%%%%%%%%%%%%%%%%%%%%%%%%%%%%%%%%%%%%%%%%%%%%%%%%%%%%
%%%%%%%%%%%%%%%%%%%%%%%%%%%%%%%%%%%%%%%%%%%%%%%%%%%%%%%%%%%%%%%%%%%%%%%%%%%
%%% REBECCA'S FIRST VERSION

%% The plan properties \textit{"Only one truck is used."} or \textit{"Package A and 
%% B are delivered by the same truck"} restrict the actions which can be used to 
%% achieve a goal. To include properties in our framework which are described by a
%% propositional logic with the atoms \emph{plan uses at least one action from $A_i$} 
%% where $A_i \subset A$, we introduce a compilation from such expressions.

%% \begin{definition}
%% 	Let $\Pi = (V, A, c, I)$ be the original planning task and $A_P=\{A_0, \cdots, A_n\}$ 
%% 	with $A_i \subseteq A$ 
%% 	%\joerg{does it really have to be a partition? and does that restriction make sense? I think neither is the case: in the compilation, a single action could set several flags, one for each action set it belongs to; and certainly one can think of constraints where the relevant action sets overlap} and a propositional formal $\mathcal{P}$ over the atoms $\mathcal{F} = \{f_i | A_i \in A_p\}$. 
%% 	Then the action set compilation $\Pi' = (V', A', c, I')$ is defined as: 
%% 	$V' = V \cup \{\text{used}_i | A_i \in A_P\} \cup \{\text{sat}_{\mathcal{P}}, \text{eval-phase}\}$ 
%% 	with $\mathcal{D}_{\text{used}_i} = \mathcal{D}_{\text{sat}_{\mathcal{P}}} = \mathcal{D}_{\text{eval-phase}} = \{0,1\}$, 
%% 	$I' = I \cup \{\text{used}_i = 0 | A_i \in A_P\} \cup \{\text{sat}_{P} = 0, \text{eval-phase} = 0\}$ 
%% 	and $A' = \{ a' | a \in A, 
%% 	\text{pre}_{a'} = \text{pre}_{a} \cup \{\text{eval-phase} = 0\} , \text{eff}_{a'} = \text{eff}_a \cup \{\text{used}_i = 1 | a \in A_i\}\} \cup 
%% 	\{\text{cp} | \text{pre}_{cp} = \{\text{eval-phase} = 0\}, \text{eff} = \{\text{eval-phase} = 1 \}\} \cup 
%% 	\{\text{actions to eval } \mathcal{P}\}$. 
%% \end{definition}

%% For every action set $A_i$ in $A_P$ we introduce a new fact $\text{used}_i$ which is initially 
%% \emph{false} and changed to \emph{true} by any action in $A_i$. The new variable $sat_{\mathcal{P}}$ indicates 
%% if the property $\mathcal{P}$ is satisfied at the end of the plan. To prevent an 
%% evaluation of the property before the end of the plan we introduce an additional binary variable
%% which indicates if we are in the execution or evaluation phase of the action sequence. Once the \textit{change-phase} (cp) action
%% is applied no original actions can be executed anymore and the actions sequence is evaluated.


