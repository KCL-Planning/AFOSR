\section{Compilations into Goal Dependencies}
\label{compilation}

\rebecca{ 1-1.5 page(s) Michael/Rebecca }

\rebecca{Define a FDR task}

\subsection{Action Set Compilation}

\joerg{there first needs to be a separate definition of
  \defined{action constraints}. Then a definition similar to the one
  below, that specifies the compilation. Then a formal Proposition
  claiming that the compilation is correct: given task \task\ with
  induced plan set \plans\ and plan property set \props\ consisting of
  goal facts and action constraints, if \task', \plans', and \props'
  is the compiled task, induced plan set, and property set, then a PDA
  for \plans\ and \props\ can be constructed in polynomial time from
  one for \plans' and \props'. In the final paper, the proof can be a
  short textual explanation, not in a formal proof environment,
  basically just describing how to do this and why it works; but
  please first make a more formal detailed proof that explicitly gives
  the polynomial time algorithm and argues in detail why indeed the
  result is a PDA for \plans\ and \props.}

The plan properties \textit{"Only one truck is used."} or \textit{"Package A and 
B are delivered by the same truck"} restrict the actions which can be used to 
achieve a goal. To include properties in our framework which are described by a
propositional logic with the atoms \emph{plan uses at least one action from $A_i$} 
where $A_i \subset A$, we introduce a compilation from such expressions.

\begin{definition}
	Let $\Pi = (V, A, c, I)$ be the original planning task and $A_P=\{A_0, \cdots, A_n\}$ 
	with $A_i \subseteq A$ 
	%\joerg{does it really have to be a partition? and does that restriction make sense? I think neither is the case: in the compilation, a single action could set several flags, one for each action set it belongs to; and certainly one can think of constraints where the relevant action sets overlap} and a propositional formal $\mathcal{P}$ over the atoms $\mathcal{F} = \{f_i | A_i \in A_p\}$. 
	Then the action set compilation $\Pi' = (V', A', c, I')$ is defined as: 
	$V' = V \cup \{\text{used}_i | A_i \in A_P\} \cup \{\text{sat}_{\mathcal{P}}, \text{eval-phase}\}$ 
	with $\mathcal{D}_{\text{used}_i} = \mathcal{D}_{\text{sat}_{\mathcal{P}}} = \mathcal{D}_{\text{eval-phase}} = \{0,1\}$, 
	$I' = I \cup \{\text{used}_i = 0 | A_i \in A_P\} \cup \{\text{sat}_{P} = 0, \text{eval-phase} = 0\}$ 
	and $A' = \{ a' | a \in A, 
	\text{pre}_{a'} = \text{pre}_{a} \cup \{\text{eval-phase} = 0\} , \text{eff}_{a'} = \text{eff}_a \cup \{\text{used}_i = 1 | a \in A_i\}\} \cup 
	\{\text{cp} | \text{pre}_{cp} = \{\text{eval-phase} = 0\}, \text{eff} = \{\text{eval-phase} = 1 \}\} \cup 
	\{\text{actions to eval } \mathcal{P}\}$. 
\end{definition}

For every action set $A_i$ in $A_P$ we introduce a new fact $\text{used}_i$ which is initially 
\emph{false} and changed to \emph{true} by any action in $A_i$. The new variable $sat_{\mathcal{P}}$ indicates 
if the property $\mathcal{P}$ is satisfied at the end of the plan. To prevent an 
evaluation of the property before the end of the plan we introduce an additional binary variable
which indicates if we are in the execution or evaluation phase of the action sequence. Once the \textit{change-phase} (cp) action
is applied no original actions can be executed anymore and the actions sequence is evaluated.


