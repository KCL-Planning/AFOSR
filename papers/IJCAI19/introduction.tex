\section{Introduction}
\label{introduction}

\joerg{IJCAI'19 page limit: 6 pages main text, 1 page refs}

\joerg{1 page Joerg/Dan; Dan to make first go, then Joerg to make a
  pass. ... as our key point here is introducing a new
  framework/approach, we need to be real careful to push the right
  knobs / to set the right expectations and avoid misunderstandings or
  reviewer questions a la ``isn't this the same as ...''}



\cite{fox:etal:ijcai-ws-17}


\joerg{discuss relation to domain analysis / model checking?
  basically, our approach can be viewed as a form of domain analysis
  checking the truth of formulas over plan space. it seems to me that
  ``domain analysis'' should definitely be mentioned in our
  introduction, as another frame of reference besides
  XAIP. ... regarding model checking I suppose it's fine to simply say
  that our analysis can be viewed as a systematic form of checking a
  set of properties of interest, and we structure, analyze, and
  implement this form of analysis as makes sense in planning. ... as a
  concrete literature link, I know that some people from Brasil worked
  on using model checkers as a domaiun analysis tool, in the ICKEPS
  context; but I don't have the concrete references in mind/ don't
  know whether and where this has been published. ... we could perhaps
  mention in this context (currently mentioned in my text snippet
  below, but this could be moved elsewhere) that ultimately one may
  want to identify interesting plan properties automatically, which is
  then clearly beyond anything addressed in model checking.}


\joerg{need to set the stage for, and synchronize with, the generic
  framework in the next section ... here I include a text I have
  already written as a form of lead-up to the framework; I have spent
  quite some time getting this right, and I like it now, but I'm not
  sure how it would sit/will sit in the introduction as a whole; may
  need further adaptations. TBD... Dan let me know if you wanna have a
  chat about this}

Our framework assumes a planning task \task, inducing a space of plans
\plans. The target is to answer user questions about properties of the
plans \plans, where a property is some Boolean function on plans. For
example, a question ``Why does the plan satisfy $A$ rather than $B$?''
is addressed by analyzing what would happen in the opposite case, \ie,
determining what other properties are entailed by a plan property of
the form ``$\neg A \wedge B$''. We assume that the set \props\ of plan
properties whose dependencies are of interest within \plans\ is given
(or specified compactly) in the input. An interesting yet challenging
question for future work is how one can automatically identify
relevant plan properties \props\ to analyze.

Observe that \plans\ itself may be naturally defined as the set of
plans satisfying a given set of plan properties. For example, these
properties may ask to achieve a set of goal facts. In such a setting,
it makes sense to distinguish between \defined{enforced} plan
properties, that induce \plans; vs.\ \defined{analyzed} plan
properties, whose entailment relations within \plans\ we wish to
identify. 

Enforced vs.\ analyzed properties can play different roles depending
on the application scenario. In a quest for understanding causal
relationships between phenomena in the plan (\eg\ dependencies between
action subsets used), the analyzed properties capture these
phenomena. Another use case is a quest to identify a preferred plan in
oversubscription planning
\cite{smith:icaps-04,domshlak:mirkis:jair-15}, where the analyzed
properties capture ``soft goals'', and the enforced properties are
``hard goals''. The analysis then identifies the precise trade-offs
between the soft goals; one may include additional analyzed properties
aimed at identifying the causes behind these trade-offs. In that
setting, our approach also supports an iterative planning process
along the lines suggested by \cite{smith:aaai-12}: if the consequences
of analyzed property $p = \neg A \wedge B$ are tolerable to the user,
she may choose to enforce $p$, gradually narrowing the space
\plans\ of candidate plans. 


%% \joerg{Text snippet from previous abstract below. ... I think that
%%   interactive planning is the 2nd point, not the 1st one (as in the
%%   proposal: it's something enabled by our approach but the approach
%%   remains relevant without it), so I have removed this from the
%%   abstract. ... Also, explanation in terms of the search space is a
%%   bit itchy as here algorithm-specific, rather than task-specific,
%%   aspects come into play; need to think carefully abouyt whether to
%%   mention this and if so how.}
%% %
%% In an interactive planning process, it is important for human users to
%% understand the decision rationale behind the suggested plans: Why is
%% this plan better than the alternatives?  In principle this can be
%% explained through the search space explored by the planner.  But how
%% to make a vast search space understandable to a human user? 


\joerg{mention/emphasize WHY and HOW questions from NASA discussions
  somewhere? hookup for WHY being ``additional analyzed properties
  aimed at identifying the causes behind these trade-offs''? HOW would
  fit into discussion of enforced vs analyzed properties and their
  role in interactive planning; both fit into discussion of how to
  find properties \props\ in the first place}
