\section{Introduction}
\label{introduction}

%% \joerg{To save space, I think this text par could be significantly
%%   shortened, removing the discussions of XAI in general and of
%%   previous model-based techniques}
%% %
%% There is growing enthusiasm around Explainable AI (XAI), with great
%% attention not only in academia but also in the private sector, mainly
%% motivated by the need for users of AI systems to become more confident
%% in the behaviour of AI systems, and to trust the decisions they make
%% (\eg\ \cite{citekeyrelatedworks}). 
%% %
%% \joerg{adding this due to a recent AIJ board discussion where some
%%   people where adamant that XAI has been done long before
%%   already})
%% %
%% Model-based techniques are well-suited to provide explanations in
%% principle (and research towards such capabilities has a long tradition
%% \cite{citekeyrelatedworks}).
%% %
%% AI Planning in particular is relevant to XAI as a decision-making
%% methodology. Consequently, research on Explainable AI Planning (XAIP)
%% has received increasing interest in recent years
%% (\eg\ \cite{citekeyrelatedworks}).
%
Explainable AI (XAI) is concerned with making AI systems' decisions
more lucid and thus trustworthy. AI planning is relevant to XAI as a
decision-making methodology, model-based and thus suited to provide
explanations in principle. Consequently, research on explainable AI
planning (XAIP) has received increasing interest in recent years
(\eg\ \cite{seegebarth:etal:icaps-12,smith:aaai-12,langley:etal:aaai-17,fox:etal:ijcai-ws-17,chakraborti:etal:ijcai-17,chakraborti:icaps-19}).

A recent analysis \cite{miller:ai-19} of lessons to be learned for XAI
from social sciences highlights that user questions are often
\emph{contrastive}. A question ``Why this?'' actually means ``Why this
\emph{rather than something else} that I would expect?''. To address
such queries, explanatory systems should analyse alternative
solutions, and support the user in understanding the consequences of
the ``something else'' in question.
%
AI planning fits well for this kind of analysis. Two prior works
designed variants thereof
\cite{fox:etal:ijcai-ws-17,miller:corr-18}. The work by Fox et al.\ is
the starting point of our work here.

Fox et al.\ suggest, given a plan $\plan$ and a user question ``Why
does $\plan$ start with action $A$ rather than $B$?'', to generate a
new plan $\plan'$ starting with $B$, and answer the question based on
comparing the two plans: undesirable properties of $\plan'$ serve to
explain the previous decision. While this idea is natural, a key
weakness is
%
%%the potentially arbitrary nature of $\plan'$.
%
%% One difficulty is that the planner might choose to
%% simply undo $B$ and re-insert $A$. More generally, t
%
that there may be differences between \plan\ and $\plan'$ unrelated to
the use of $A$ vs.\ $B$. Many comparison aspects (\eg\ which other
actions are used, or which ``soft'' objectives are satisfied) may be
affected by arbitrary decisions in the planner's search.
%
%% If optimal planning techniques are used, comparisons of the
%% objective-function values of \plan\ vs.\ $\plan'$ are well
%% justified. Yet all other comparisons (\eg\ which other actions are
%% used, or which ``soft'' objectives are satisfied) may be affected
%% by arbitrary decisions in the planner's search.

Here we address the same kind of explanation problem, but we replace
the \emph{existential} answer generating a single alternative plan
$\plan'$ with a \emph{universal} answer determining shared properties
of \emph{all} possible such alternatives. In this way, the analysis we
propose aims at explaining the space of possible plans, rather than
pointing out examples.

Our proposed analysis works at the level of \defined{plan properties}:
Boolean functions on plans that capture aspects of plans the user
cares about (whether or not the plan starts with a particular action,
whether or not a particular soft objective is satisfied, etc). We
assume that the set \props\ of plan properties of interest is given as
part of the input.\footnote{An interesting yet challenging question
  for future work is how one can automatically identify relevant plan
  properties.} Our analysis then determines the \defined{dependencies}
across plan properties, \ie, \defined{plan-space entailments} which we
define as follows. The ``plan space'' is the set \plans\ of candidate
plans to be considered (canonically, the set of plans for an input
planning task). A plan property $p$ \defined{entails} another property
$p'$ in \plans\ if every $\plan \in \plans$ that satisfies $p$ also
satisfies $p'$. A user question ``Why does the current plan
\plan\ satisfy $p$ rather than $q$?'' can then be answered in terms of
the properties $q'$ not true in \plan\ but entailed by $q$: things
that will \emph{necessarily} change when satisfying $q$.

Our approach also supports iterative planning, along the lines
suggested by Smith \shortcite{smith:aaai-12}. Given a current plan
$\plan \in \plans$ and a user question ``Why achieve $p$ rather than
$q$?'', if the consequences of $q$ are tolerable to the user, she may
choose to enforce $q$, gradually narrowing the plan-candidate space
\plans.
%
% Joerg: Text highlighting enforced vs analyzed; simplified to save space
%
%% Observe that \plans\ itself may be viewed as being defined through a
%% set of \emph{enforced} plan properties (like achieving a set of goal
%% facts). Such enforced properties are then distinguished from the
%% \emph{analyzed} properties whose dependencies we wish to identify.
%% %
%% These two classes of properties can play different roles depending on
%% the application scenario. In contrastive explanations as outlined
%% above, the enforced properties are fixed. However, our approach also
%% supports an iterative planning process for oversubscription planning
%% (\eg\ \cite{smith:icaps-04,domshlak:mirkis:jair-15}), along the lines
%% suggested by Smith \shortcite{smith:aaai-12}. The analyzed properties
%% then capture ``soft goals'', while the enforced properties capture
%% ``hard goals''. Given a currently suggested plan $\plan \in \plans$
%% and a user question ``Why $p$ rather than $q$?'', if the consequences
%% of analyzed property $q$ are tolerable to the user, she may choose to
%% enforce $q$, gradually narrowing the plan-candidate space \plans.
%% %
%% % Joerg: shortetened to save space
%% %
%% %% Observe that \plans\ itself may be naturally defined as the set of
%% %% plans satisfying a given set of plan properties. For example, these
%% %% properties may ask to achieve a set of goal facts. In such a setting,
%% %% it makes sense to distinguish between \defined{enforced} plan
%% %% properties, that induce \plans; vs.\ \defined{analyzed} plan
%% %% properties, whose entailment relations within \plans\ we wish to
%% %% identify. 
%% %
%% %% Enforced vs.\ analyzed properties can play different roles depending
%% %% on the application scenario. In classical planning, the analyzed
%% %% properties may capture relevant plan phenomena in a user quest to
%% %% understand causal relationships between these phenomena
%% %% (\eg\ dependencies between action subsets used). Another use case is a
%% %% user quest to identify a preferred plan in oversubscription planning
%% %% (\eg\ \cite{smith:icaps-04,domshlak:mirkis:jair-15}), where the
%% %% analyzed properties capture ``soft goals'', and the enforced
%% %% properties are ``hard goals''. The analysis then identifies the
%% %% precise trade-offs between the soft goals.
%% %% %
%% %% % Joerg: too complicated/more distracting than helpful
%% %% %
%% %% %% ; one may include additional analyzed properties aimed at identifying
%% %% %% the causes behind these trade-offs.
%% %% %
%% %% In that setting, our approach also supports an iterative planning
%% %% process along the lines suggested by Smith \shortcite{smith:aaai-12}:
%% %% given a currently suggested plan $\plan \in \plans$ and a user
%% %% question ``Why $p$ rather than $q$?'', if the consequences of analyzed
%% %% property $q$ are tolerable to the user, she may choose to enforce $q$,
%% %% gradually narrowing the candidate space \plans.

We remark that our approach can be viewed as an intermediate between
domain/task analysis (\eg\ \cite{fox:long:jair-98}), which our
approach generalizes; and model checking applied to planning models,
which our approach is an instance of (related to
\cite{vaquero:etal:keq-13}). 
%
% Joerg: Detailed discussion of domain analysis and model checking;
% simplified to save space/not be distracting here.
%
%% Another alternate view of our approach is as a form of domain analysis
%% (actually: task analysis), identifying particular properties of plan
%% space ahead of time. Indeed, various popular task analyses can be cast
%% as instances of our framework. A fact pair $(p,q)$ is mutually
%% exclusive \cite{blum:furst:ai-97} iff $p$-true-at-end entails $\neg
%% q$-true-at-end in the space of all applicable action sequences; a fact
%% $p$ is a landmark \cite{hoffmann:etal:jair-04} iff $\true$ entails
%% $p$-true-at-some-point; other examples presumably exist. From this
%% point of view, we generalize previous concepts to a broader
%% perspective aimed at addressing arbitrary user questions. At the same
%% time, our approach itself can be viewed as an instance of model
%% checking of planning models \cite{clarke:etal:01},\footnote{There has
%%   been little work on this subject; Vaquero et
%%   al.\ \shortcite{vaquero:etal:keq-13} use Petri nets to capture and
%%   check dynamic aspects of planning models in itSIMPLE.}
%% systematically checking all entailments between plan properties. Again
%% the value of our framework lies in its suitability for XAIP (plus
%% computational gains from considering all \props\ dependencies in
%% unison rather than running individual entailment checks).


Our contributions are as follows. 
%
%% lies in formulating this intermediate problem suited to XAIP as
%% outlined, and instantiating that formulation with initial
%% technology showing promise in practice.
%
We conceptualize the explainability problems we address, through a
generic framework making minimal assumptions on the planning context
(Section~\ref{framework}). We instantiate the framework with goal-fact
conjunction dependencies in oversubscription planning
(\eg\ \cite{smith:icaps-04,domshlak:mirkis:jair-15}), and devise
analysis algorithms for that purpose (Section~\ref{goaldep}). We show
that more general plan properties -- in particular,
\defined{action-set properties} -- can be compiled into goal facts and
thus into that analysis (Section~\ref{compilation}).
%
We give an illustrative example (Section~\ref{illustrative-example}),
and we evaluate our techniques on international planning competition
(IPC) benchmarks modified for oversubscription planning, and on IPC
benchmarks extended with action-set properties
(Section~\ref{experiments}).
%
% Joerg: I contemplated making this more concrete, along the lines of
% my previous pitch ``similar scalability as optimal planning'', but
% really the picture is complicated and that pitch incurs the risk of
% wrong expectations.
%
We find that, in a variety of benchmark studies, the suggested
analyses can be feasible and produce compact answers for human
inspection.

