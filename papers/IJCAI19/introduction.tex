\section{Introduction}
\label{introduction}

\joerg{IJCAI'19 page limit: 6 pages main text, 1 page refs}

\joerg{1 page Joerg/Dan; Dan to make first go, then Joerg to make a
  pass. ... as our key point here is introducing a new
  framework/approach, we need to be real careful to push the right
  knobs / to set the right expectations and avoid misunderstandings or
  reviewer questions a la ``isn't this the same as ...''}



\cite{fox:etal:ijcai-ws-17}


\joerg{discuss relation to domain analysis / model checking?
  basically, our approach can be viewed as a form of domain analysis
  checking the truth of formulas over plan space. it seems to me that
  ``domain analysis'' should definitely be mentioned in our
  introduction, as another frame of reference besides
  XAIP. ... regarding model checking, our analysis can be viewed as a
  systematic form of checking a set of properties of interest; we
  structure this form of analysis as makes sense in explainable
  planning; computationally we exploit synergies in addrssing the
  entire set of properties rather than checking each possible
  dependency in isolation; in the long-term, a highly relevant
  research line is how to discover relevant properties in the first
  place. ... as a concrete literature link, I know that some people
  from Brasil worked on using model checkers as a domaiun analysis
  tool, in the ICKEPS context; but I don't have the concrete
  references in mind/ don't know whether and where this has been
  published. ... we could perhaps mention in this context (currently
  mentioned in my text snippet below, but this could be moved
  elsewhere) that ultimately one may want to identify interesting plan
  properties automatically, which is then clearly beyond anything
  addressed in model checking.}


\joerg{need to set the stage for, and synchronize with, the generic
  framework in the next section ... here I include a text I have
  already written as a form of lead-up to the framework; I have spent
  quite some time getting this right, and I like some of it now, but
  I'm not sure how it would sit/will sit in the introduction as a
  whole; will need further adaptations. TBD... Dan let me know if you
  wanna have a chat about this}

There is a growing enthusiasm around Explainable AI (XAI), with great attention not only in academia but also in the private sector, mainly motivated by the need to allow the user of AI systems to become confident on their behaviour and to trust the decisions they made. [add links to key related work and workshops].

A recent analysis by Tim Miller on the insights we can get from the social science for making explanations more effective, highlighted that questions for explanations are \textit{contrastive}, meaning that if the user asks "Why this?" they actually mean "Why this \textit{rather than something else}?".
 This implies that explanatory systems should be able to analyse the space of alternative solutions, and allow the user to understand the consequences of that \textit{something else} happening.

Within XAI, Planning is starting to play a very relevant role, and [add more related work and workshops]...
Crucially, planning fits very well with the paradigm of contrastive explanations \cite{xaip_magazzeni, contrastive_miller}.

In this paper, we propose a formal approach for providing contrastive explanations by considering the space of (alternative) plans through plan-property dependencies. In this setting, a property of the plan corresponds to actions that the user would expect to see or not to see in the plan, and the plan-property dependency analysis allows explaining to the user that in order for the desired property to hold, something else must necessarily change in the plan, as formally guaranteed by the framework we present in this paper.

Our framework assumes a planning task \task, inducing a space of plans
\plans. The target is to answer user questions about properties of the
plans \plans, where a property is some Boolean function on plans. For
example, a question ``Why does the plan satisfy $A$ rather than $B$?''
is addressed by analyzing what would happen in the opposite case, \ie,
determining what other properties are entailed by a plan property of
the form ``$\neg A \wedge B$'' \joerg{this needs more discussion of
  inhowfar this is an explanation, and needs to be placed into the
  explanation literature}. 

\joerg{this statement sits awkwardly here; remove?}
%



We assume that the set \props\ of plan properties whose dependencies
are of interest within \plans\ is given in the input. An interesting
yet challenging question for future work is how one can automatically
identify relevant plan properties \props\ to analyze.

Observe that \plans\ itself may be naturally defined as the set of
plans satisfying a given set of plan properties. For example, these
properties may ask to achieve a set of goal facts. In such a setting,
it makes sense to distinguish between \defined{enforced} plan
properties, that induce \plans; vs.\ \defined{analyzed} plan
properties, whose entailment relations within \plans\ we wish to
identify. 

Enforced vs.\ analyzed properties can play different roles depending
on the application scenario. In classical planning, the analyzed
properties may capture relevant plan phenomena in a user quest to
understand causal relationships between these phenomena
(\eg\ dependencies between action subsets used). Another use case is a
user quest to identify a preferred plan in oversubscription planning
\cite{smith:icaps-04,domshlak:mirkis:jair-15}, where the analyzed
properties capture ``soft goals'', and the enforced properties are
``hard goals''. The analysis then identifies the precise trade-offs
between the soft goals; one may include additional analyzed properties
aimed at identifying the causes behind these trade-offs. In that
setting, our approach also supports an iterative planning process
along the lines suggested by \cite{smith:aaai-12}: if the consequences
of analyzed property $p = \neg A \wedge B$ are tolerable to the user,
she may choose to enforce $p$, gradually narrowing the space
\plans\ of candidate plans.


%% \joerg{Text snippet from previous abstract below. ... I think that
%%   interactive planning is the 2nd point, not the 1st one (as in the
%%   proposal: it's something enabled by our approach but the approach
%%   remains relevant without it), so I have removed this from the
%%   abstract. ... Also, explanation in terms of the search space is a
%%   bit itchy as here algorithm-specific, rather than task-specific,
%%   aspects come into play; need to think carefully abouyt whether to
%%   mention this and if so how.}
%% %
%% In an interactive planning process, it is important for human users to
%% understand the decision rationale behind the suggested plans: Why is
%% this plan better than the alternatives?  In principle this can be
%% explained through the search space explored by the planner.  But how
%% to make a vast search space understandable to a human user? 

