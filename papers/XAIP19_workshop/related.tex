\section{Related Work}
\label{related}

\joerg{I normally don't include sections like this, as I find them
  boring; I usually just cover the brod literature context in the
  intro, and specific relations at those technical places where they
  are pertinent. Here though, matters may be different in that, in the
  intro, we can and should only spin the stroy for XAIP and perhaps
  domain analysis; while at the technical level people may wonder if
  what we do here concretely is not just a work on oversubscription
  planning. So we may want to add a section here discussing technical
  and conceptual relations to works -- in oversubscription planning,
  and also in conflict analysis see pointers below -- which are not
  naturally part of the XAIP story in the intro. ... TBD when writing
  the introduction}


%% \joerg{discuss relation to domain analysis / model checking?
%%   basically, our approach can be viewed as a form of domain analysis
%%   checking the truth of formulas over plan space. it seems to me that
%%   ``domain analysis'' should definitely be mentioned in our
%%   introduction, as another frame of reference besides
%%   XAIP. ... regarding model checking, our analysis can be viewed as a
%%   systematic form of checking a set of properties of interest; we
%%   structure this form of analysis as makes sense in explainable
%%   planning; computationally we exploit synergies in addressing the
%%   entire set of properties rather than checking each possible
%%   dependency in isolation; in the long-term, a highly relevant
%%   research line is how to discover relevant properties in the first
%%   place. ... as a concrete literature link, I know that some people
%%   from Brasil worked on using model checkers as a domaiun analysis
%%   tool, in the ICKEPS context; but I don't have the concrete
%%   references in mind/ don't know whether and where this has been
%%   published. ... we could perhaps mention in this context (currently
%%   mentioned in my text snippet below, but this could be moved
%%   elsewhere) that ultimately one may want to identify interesting plan
%%   properties automatically, which is then clearly beyond anything
%%   addressed in model checking.}
%
An alternate view of our approach is as a form of domain analysis
(actually: task analysis), identifying particular properties of plan
space ahead of time. Indeed, various popular task analyses can be cast
as instances of our framework. A fact pair $(p,q)$ is mutually
exclusive \cite{blum:furst:ai-97} iff $p$-true-at-end entails $\neg
q$-true-at-end in the space of all applicable action sequences; a fact
$p$ is a landmark \cite{hoffmann:etal:jair-04} iff $\true$ entails
$p$-true-at-some-point; other examples presumably exist. From this
point of view, we generalize previous concepts to a broader
perspective aimed at addressing arbitrary user questions. 

At the same time, our approach itself can be viewed as an instance of
model checking of planning models
\cite{clarke:etal:01},\footnote{There has been little work on this
  subject; Vaquero et al.\ \shortcite{vaquero:etal:keq-13} use Petri
  nets to capture and check dynamic aspects of planning models in
  itSIMPLE.}  systematically checking all entailments between plan
properties. Again the value of our framework lies in its suitability
for XAIP (plus computational gains from considering all
\props\ dependencies in unison rather than running individual
entailment checks).


%% Related work by Brian Williams (need to find citations): reasoning
%% about user goals; oversubscription planning in constraint-based
%% planning formulation; suggest goals to drop based on conflict
%% analysis; constructive procedure to find these goals, given commitment
%% (subset of goals) and most important implications (subsets of other
%% goals dropping which makes the whole thing consistent ie goal
%% achievable). In short, the service provided is related to that
%% provided by our goalfact dependency analysis; but in a
%% direct/constructive manner, as opposed to our meta search tree, as
%% enabled by working in a constraint formulation and building on
%% conflict analysis in that setting.
%
Identifying the trade-off between soft goals in an iterative
oversubscription planning process relates, in terms of purpose, to the
work by Yu et al.\ \cite{yu:etal:jair-17}. A major difference is that
Yu et al.\ address conditional temporal problems, a form of
conditional temporal plans, instead of PDDL-style compactly specified
general planning problems. This more restricted focus allows to
leverage previous conflict analysis methods in that area. It remains a
question for future work whether such conflict analysis could inspire
more effective analysis methods in our framework.


Related work by Philippe Laborie? resource constrained planning ixtet
have some form of conflict analysis with critical set for resources
consumption, relates to our analysis in goal facts RCP?
