\section{Computing Goal-Exclusion Explanations}
\label{algorithms}

%% \joerg{0.5--1 page: spell out ijcai version; emphasize novel aspects,
%%   discuss/mention related work in conflict analysis; explicitly
%%   distinguish global vs local analysis, probably subsections; start
%%   with global, motivated by pointing out that local will be tackled by
%%   simple variant of this same algorithm; illustrative example with
%%   goal fact dependencies from previous section, use for brief
%%   illustration of bottom-up/top-down}

We now explain our algorithms for computing goal-exclusion
explanations. We start with the algorithms for GEs, from which those
for LEs follow easily.




\subsection{Global Explanations}
\label{algorithms:ge}

The goal-exclusion GE can be read off the \defined{minimal unsolvable
  goal subsets (MUGS)}, where $\goal \subseteq \goalsoft$ is a MUGS
if $\goal$ cannot be achieved but every $\goal' \subsetneq \goal$
can:

\begin{proposition}[GE from MUGS]
\label{pro:ge-from-mugs}
Let $\task =
(\vars,\allowbreak\acts,\allowbreak\cost,\allowbreak\init,\allowbreak\goalhard,\allowbreak\goalsoft,\allowbreak\costbound)$
be an OSP task, and \plans\ its set of plans.

Then $\entails{\plans}{\bigwedge_{a \in A} a}{\neg \bigwedge_{b \in B}
  b}$ is non-dominated if and only if $A \cup B$ is a MUGS.
\end{proposition}

\begin{proof}
A \plans-entailment $\entails{\plans}{\bigwedge_{a \in A} a}{\neg
  \bigwedge_{b \in B} b}$ clearly holds iff $A \cup B$ is
unsolvable. Non-dominated entailments result from set-inclusion
minimal $A$ and $B$, corresponding to the set-inclusion minimality of
MUGS.
\end{proof}

% Joerg email top-down vs bottom-up
%
%% Both "top down" as we do now and "bottom up" as you suggest can be
%% used to identify the minimum unsolvable goal sets (in the bottom-up
%% case a post-processing step is needed, selecting from the
%% unsolvable goal sets generated during meta-tree search the
%% set-inclusion minimal ones; but this step is easy ie polynomial in
%% the number of goal sets considered).
%
%% Which of the two is expected to work better is largely a function
%% of how large the solvable goal subsets will be. It seems plausible
%% that one or the other algorithm will often have substantial
%% advantages. In particular, for severely limited resources one would
%% expect bottom-up to be better.
%
%% So ideally we should experiment with both top-down and
%% bottom-up. And in practice it would seem quite reasonablee to run
%% both in parallel, so that a "best-of" also makes a lot of sense
%% here.
%
%% (One can even think about interleaving bottom-up with top-down in
%% an intelligent way, but that's for future work)

Observe that, from a representational point of view, this means that
we can represent our explanations even more compactly, by showing MUGS
$\goal$ instead of the corresponding exclusion entailments with $A
\cup B = \goal$. In our example, the only MUGS is $\{at(P_0,L_4),
at(P_1,L_3), at(P_2,L_5)\}$.

From a computational point of view, our problem now boils down to
computing all MUGS. This can be done through a search over goal sets,
that we refer to as \defined{systematic weakening (SysW)}:
\begin{enumerate}[(1)]
\item the start node of the search is \goal; 
\item each search step selects an open node $\goal$, calls a planner
  to test whether $\goal$ is solvable in \task, caches the result,
  and expands $\goal$ if it is unsolvable;
\item the children of a node $\goal$ are those $\goal' \subset
  \goal$ where $|\goal'| = |\goal|-1$.
\end{enumerate}
Upon termination, the MUGS are those nodes $\goal$ all of whose
children are solvable. 

Dually, \defined{systematic strenghtening (SysS)} starts from
$\emptyset$, with search steps expanding solvable nodes, and children
adding one more goal fact. Upon termination, the MUGS can be easily
obtained from the unsolvable search nodes.

In both algorithms, the worst-case number of planner calls is
exponential in $|\goalsoft|$, \ie, the number of goals whose
dependencies are being analyzed. Intuitively, SysW is better suited if
solvable goal subsets are large (and thus encountered early), while
SysS is better suited if they are small (unsolvable subsets
encountered early). For illustration, in the example SysW expands the
single search node \goalsoft, while SysS expands all solvable subsets
before reaching \goalsoft.

% Rebecca's notes on the implementation:
%
%% We fix the order of the goal facts: [g0, g1, …, gn]
%% Then each goal subsets gets an unique id based on the contained goal facts:
%% For example the subset [g0, g2] has in binary the id 101. The set bits indicate which goal facts are contained. (This limits our algorithm to 31 soft goals.)
%
%% During search we cache all results and prune already solved goal subsets based on their id.
%% The ids induce an expansion order on all possible nodes/subset, such that you always know if node with id1 has been solved before node with id2.
%
%% If we expand a node in the meta search there exist three possible states for the children
%% (which can easily be generated by setting one of the 1 bits of the parent id to 0 parent: 101 → children 001,100):
%
%% 1: the child node is in the expansion order after its parent → add the node to the openlist
%
%% 2. the child node is in the expansion oder before its parent:
%
%%     2.1: the node with the given id exists already→ use this result
%
%%     2.2: the node does not exists
%%         reason: a superset of the goal facts has been proved solvable and therefore is not expanded further
%%         → you know that this node is solvable without running the planner


Search enhancements are important for scalability. In particular, goal
sets can be reached from the start node by permutations of goal-fact
removal/addition steps. For effective results re-use, we assign goal
sets unique integer IDs and fix an ascending expansion order. Namely,
when expanding a node $\goal$ and considering a possible child
$\goal'$, there are three possibilities: 1.\ $id(\goal') > id(\goal)$:
$\goal'$ not seen yet, insert into open list; 2.\ $id(\goal') <
id(\goal)$ and result for $\goal'$ cached: use that result;
3.\ $id(\goal') < id(\goal)$ and result for $\goal'$ not cached: then
the search was previously cut off above $\goal'$, so we know that
$\goal'$ is solvable (SysW) respectively unsolvable (SysS).

As another search enhancement, we created synergy with recent nogood
learning techniques, \defined{conjunction learning}
\cite{steinmetz:hoffmann:ai-17} and \defined{trap learning}
\cite{steinmetz:hoffmann:ijcai-17}. These techniques refine dead-end
detection methods (nogoods) based on the unsolvable states encountered
in state space search on a planning task. As the children tasks in our
searches are closely related to their parents, the refined nogoods are
likely to be useful still. So we \defined{transfer} the nogoods along
search paths, resulting in iteratively stronger and stronger
nogoods. For both forms of learning, the nogoods learned depend on the
goal, so that only some of the nogoods remain valid for transfer in
SysW where children remove goals. We designed the following methods to
identify that nogood subset. In conjunction learning, we remember the
subset $\goal$ of goals which became unreachable in a learning step,
so that validity is preserved so long as at least one member of
$\goal$ is still present. In trap learning, the validity test checks,
for every end node of the trap, whether that node is still either
mutex with the goal, or is still proved unsolvable by the heuristic
function.

% Joerg: hC remains valid anyhow; when learning a clause from hC,
% Marcel remembers the set of goals that became unreachable; validity
% checking is then simply whether one of these goals is still
% there. For traps, the validity check means to test for every end
% node of the trap whether that node either is mutex with the goal
% still, or is cut off by the h fn still.
%
%% We furthermore identify non-trivial synergy with recent nogood
%% learning techniques \cite{steinmetz:hoffmann:ai-17}, which refine a
%% critical-path heuristic \hc\ during forward search on a task \task,
%% based on the unsolvable states encountered. The resulting heuristic
%% \hc\ reasons about an enlarged set of conjunctions, in a manner geared
%% at quickly identifying unsolvable states in \task. Observe that this
%% naturally lends itself to \defined{\hc-transfer} in systematic
%% weakening: the refined \hc\ from an unsolvable node $\goal$ is likely
%% to be a good unsolvability detector for the closely related children
%% tasks with goals $\goal' \subset \goal$ where $|\goal'| =
%% |\goal|-1$. We thus transfer the learned \hc\ functions downwards
%% along the search paths in systematic weakening, iteratively yielding
%% stronger and stronger unsolvability detectors.
%% %
%% % Joerg: proably more confusing than helpful
%% %
%% %% Similarly, we transfer the \defined{clauses} (nogood pruning rules)
%% %% learned in this approach, transferring thjose that depend only on the
%% %% goals still present in children nodes.

We acknowledge that the basic structure of SysW and SysS is simple,
and relates to various prior works addressing conflict analysis in one
or the other form. Our contribution here consists in assembling the
algorithms in a form suited to our purposes, plus small enhancements
as described. We briefly discuss related work later on, in a separate
section.






\subsection{Local Explanations}
\label{algorithms:le}

For local explanation, we are given a soft-goal subset $A$ whose
non-rhs-dominated entailments we wish to identify. This corresponds to
fixing $A$ and minimizing only $B$ -- which in turn corresponds to
moving $A$ into the hard goals and computing MUGS on the modified
task.

Precisely, assume an OSP task $\task =
(\vars,\allowbreak\acts,\allowbreak\cost,\allowbreak\init,\allowbreak\goalhard,\allowbreak\goalsoft,\allowbreak\costbound)$,
with set of plans \plans. Consider the modified task $\task' :=
(\vars,\allowbreak\acts,\allowbreak\cost,\allowbreak\init,\allowbreak\goalhard
\cup A,\allowbreak\goalsoft \setminus
A,\allowbreak\costbound)$. Clearly, we have the goal exclusion
$\entails{\plans}{\bigwedge_{a \in A} a}{\neg \bigwedge_{b \in B} b}$
iff $B$ is unsolvable in $\task'$; and $B$ is minimal in $\task'$ iff
the goal exclusion is non-rhs-dominated. So the MUGS in $\task'$ yield
exactly the goal-exclusion LE for $A$ as per
Definition~\ref{def:local-explanation}.

% Joerg: version using stronger incorrect result that B is MUGS
% in task'
%
%% For local explanation, we are given a soft-goal subset $A$ whose
%% non-dominated entailments we wish to identify. This can be done by
%% computing the goal-exclusion GE and collecting all
%% $\entails{\plans}{\bigwedge_{a \in A'} a}{\neg \bigwedge_{b \in B}
%% b}$ where $A \supseteq A'$ ==> NOT TRUE, B CN BE SMALLER AS A GETS
%% BIGGER. But it can be done more effectively by restricting the
%% computation to the relevant part of the GE.  Namely, we can compute
%% MUGS in a modified task where $A$ is moved to \goalhard:
%
%% \begin{proposition}[LE from MUGS]
%% \label{pro:le-from-mugs}
%% Let $\task =
%% (\vars,\allowbreak\acts,\allowbreak\cost,\allowbreak\init,\allowbreak\goalhard,\allowbreak\goalsoft,\allowbreak\costbound)$
%% be an OSP task, \plans\ its set of plans, and $A \subseteq \goalsoft$. 
%
%% Then there exists a non-dominated goal exclusion
%% $\entails{\plans}{\bigwedge_{a \in A'} a}{\neg \bigwedge_{b \in B} b}$
%% where $A \supseteq A'$ if and only if $B$ is unsolvable in $\task' :=
%% (\vars,\allowbreak\acts,\allowbreak\cost,\allowbreak\init,\allowbreak\goalhard
%% \cup A,\allowbreak\goalsoft \setminus A,\allowbreak\costbound)$.
%% \end{proposition}
%
%% \begin{proof}
%% From left to right: $A' \cup B$ is a MUGS in $\task$ by
%% Proposition~\ref{pro:ge-from-mugs}. Therefore, as $A \supseteq A'$,
%% $B$ is unsolvable in $\task'$. As $A' \cup B$ is minimal in $\task$,
%% $B$ is minimal in $\task'$ \joerg{hm. is this true? If we add more
%%   stuff to $A'$, we might be able to drop some stuff from $B$,
%%   right? which means that $B$ is not necessarily minimal in $\task'$}.
%
%% From right to left: As $B$ is unsolvable in $\task'$, $A \cup B$ is
%% unsolvable in $\task$, \ie, $\entails{\plans}{\bigwedge_{a \in A'}
%%   a}{\neg \bigwedge_{b \in B} b}$. As $B$ is minimal in $\task'$, the
%% right-hand side of this goal exclusion is minimal in
%% $\task$. Minimizing its left-hand side results in the desired
%% non-dominated goal exclusion.
%% \end{proof}

Observe that $\task'$ has less soft goals, so MUGS computation becomes
easier, reflecting the advantage of local analysis over global
analysis. That advantage grows with $|A|$, \ie, intuitively, with how
specific the user question is. At the extreme end, $A = \goalsoft$ and
both SysW and SysS on $\task'$ contain a single search node testing
solvability of $\goalhard \cup \goalsoft$.\footnote{As previously
  mentioned, given a current plan \plan\ one could restrict the answer
  to entailments $\neg \bigwedge_{b \in B} b$ where $\plan \models
  \bigwedge_{b \in B} b$. This corresponds to removing all $g$ with
  $\plan \not\models g$ from \goalsoft\ in $\task'$, further
  simplifying the computation.}
















%%%%%%%%%%%%%%%%%%%%%%%%%%%%%%%%%%%%%%%%%%%%%%%%%%%%%%%%%%%%%%%%%%%%%%%%%
%%%%%%%%%%%%%%%%%%%%%%%%%%%%%%%%%%%%%%%%%%%%%%%%%%%%%%%%%%%%%%%%%%%%%%%%%
%%%%%%%%%%%%%%%%%%%%%%%%%%%%%%%%%%%%%%%%%%%%%%%%%%%%%%%%%%%%%%%%%%%%%%%%%

%% \subsection{Disjunctive Goal Exclusion}
%% \label{goaldep:disjunctive}

%% \joerg{this plays a very minor role in the paper. hence I
%%   briefly merged it into the above discussion. remove this section for
%%   paper}

%% Another class of dependencies that may be of interest is
%% \defined{disjunctive goal exclusion}:

%% \begin{definition}[Disjunctive Goal Exclusion]
%% Let $\task =
%% (\vars,\allowbreak\acts,\allowbreak\cost,\allowbreak\init,\allowbreak\goalhard,\allowbreak\goalsoft,\allowbreak\costbound)$
%% be an OSP task, and \plans\ its set of plans.  

%% The \defined{PDO for disjunctive goal exclusion (PDO-DGE)} is the PDO for
%% \plans, the property set $\dgeprops := \{\bigwedge_{a \in A}
%% g\allowbreak\mid\allowbreak A \subseteq G\} \cup \{\neg \bigvee_{g
%%   \in B} b \allowbreak\mid\allowbreak B \subseteq G\}$, and the
%% dependency set $\dgedeps := \{(\bigwedge_{a \in A} a,\allowbreak\neg
%% \bigwedge_{b \in B} b) \allowbreak\mid\allowbreak A \cap B =
%% \emptyset\}$.
%% \end{definition}

%% Observe that $\neg \bigvee_{b \in B} b$ is equivalent to $\bigwedge_{b
%%   \in B} \neg b$, so that $\entails{\plans}{\bigwedge_{a \in A}
%%   a}{\neg \bigvee_{b \in B} b}$ holds iff
%% $\entails{\plans}{\bigwedge_{a \in A} a}{\neg b}$ holds for every $b
%% \in B$. Indeed, the dominant PDO-DGE can be derived from the dominant
%% PDO-GE. Precisely, say that an entailment
%% $\entails{\plans}{\bigwedge_{a \in A} a}{\neg \bigvee_{b \in B} b}$ is
%% \defined{constructible} from the dominant PDO-GE if there exist $A[b]
%% \subseteq A$, for $b \in B$, such that $\entails{\plans}{\bigwedge_{a
%%     \in A[b]} a}{\neg b}$ is in the dominant PDO-GE and $\bigcup_{b
%%   \in B} A[b] = A$. We have:

%% \begin{proposition}[PDO-DGE from PDO-GE]\label{pro:pdodge-from-pdoge}
%% Let $\task =
%% (\vars,\allowbreak\acts,\allowbreak\cost,\allowbreak\init,\allowbreak\goalhard,\allowbreak\goalsoft,\allowbreak\costbound)$
%% be an OSP task, and \plans\ its set of plans.  

%% Then $\entails{\plans}{\bigwedge_{a \in A} a}{\neg \bigvee_{b \in B}
%%   b}$ is in the dominant PDO-DGE if and only if it is constructible
%% from the dominant PDO-GE, and there is no $A' \subsetneq A$ such that
%% $\entails{\plans}{\bigwedge_{a \in A'} a}{\neg \bigvee_{b \in B} b}$
%% is constructible from the dominant PDO-GE.
%% \end{proposition}

%% \begin{proof}
%% %
%% Observe first that, for any entailment $\entails{\plans}{\bigwedge_{a
%%     \in A} a}{\neg \bigvee_{b \in B} b}$ that holds, an entailment
%% $\entails{\plans}{\bigwedge_{a \in A'} a}{\neg \bigvee_{b \in B} b}$
%% with $A' \subseteq A$ is constructible from the dominant PDO-GE. This
%% is because, for every entailment $\entails{\plans}{\bigwedge_{a \in A}
%%   a}{\neg b}$ that holds, the dominant PDO-GE contains an entailment
%% $\entails{\plans}{\bigwedge_{a \in A[b]} a}{\neg b}$ with $A[b]
%% \subseteq A$.

%% The 'only if' direction follows directly from this. For the 'if'
%% direction, it suffices to observe additionally that, trivially, every
%% entailment constructible from the dominant PDO-GE does in fact hold.
%% %
%% % Joerg: outdated attempt
%% %
%% %% For the 'if' direction, say that $\entails{\plans}{\bigwedge_{a \in A}
%% %%   a}{\neg b_0}$ is in the dominant PDO-GE, and that
%% %% $\entails{\plans}{\bigwedge_{a \in A[b]} a}{\neg b}$ is in the
%% %% dominant PDO-GE for all $b_0 \neq b \in B$ with $A[b] \subseteq
%% %% A$. Then, trivially, $\entails{\plans}{\bigwedge_{a \in A} a}{\neg
%% %%   \bigvee_{b \in B} b}$. This entailment is dominant because 1)
%% %% $\entails{\plans}{\bigwedge_{a \in A} a}{\neg b_0}$ is dominant so $A$
%% %% cannot be smaller, and 2) any valid implication of a literal $\neg b$
%% %% is represented in the dominant PDE-GE so $B$ cannot be larger.
%% %
%% %% 'only if': Assume for contradiction that
%% %% $\entails{\plans}{\bigwedge_{a \in A} a}{\neg \bigvee_{b \in B} b}$ is
%% %% in the dominant PDO-DGE but this claim does not hold. If there exists
%% %% $b \in B$ for which no entailment $\entails{\plans}{\bigwedge_{a \in
%% %%     A[b]} a}{\neg b}$ is in the dominant PDO-GE, then $\bigwedge_{a
%% %%   \in A} a$ does not entail $\neg \bigvee_{b \in B} b$, in
%% %% contradiction. If $A[b] \subsetneq A$ for every $b$.
%% \end{proof}

%% Observe that the dominant PDO-DGE may be exponentially larger than the
%% dominant PDO-GE: in the worst case, every combination of
%% $\entails{\plans}{\bigwedge_{a \in A} a}{\neg b}$ entailments in the
%% dominant PDO-GE yields a different set-inclusion minimal left-hand
%% side. However, using Proposition~\ref{pro:pdodge-from-pdoge}, the
%% dominant PDO-DGE can be constructed from the dominant PDO-GE in time
%% polynomial in the size of the output, simply by iteratively
%% enumerating all entailments constructible from the dominant PDO-GE and
%% checking for subsumptions.









































































%%%%%%%%%%%%%%%%%%%%%%%%%%%%%%%%%%%%%%%%%%%%%%%%%%%%%%%%%%%%%%%%%%%%%
%%%%%%%%%%%%%%%%%%%%%%%%%%%%%%%%%%%%%%%%%%%%%%%%%%%%%%%%%%%%%%%%%%%%%
%%%%%%%%%%%%%%%%%%%%%%%%%%%%%%%%%%%%%%%%%%%%%%%%%%%%%%%%%%%%%%%%%%%%%
%%% JOERG: MY INITIAL NOTES


%% \joerg{question: does our method compute the PDO alongside the PDA?
%%   and is the PDA unique?} for 1.: We compute a cPDA: First, say G'
%%   is maximal satisfiable; we prove that [G'] is in the cGPA. Assume
%%   satisfiable G'' entails G'. Then G' \cup G'' is satisfiable, so
%%   we must have G'' \subseteq G' by maximality of G'. But then, G'
%%   entails G'' so they're equivalent. Vive versa, say that [G'] is
%%   minimal satisfiable in PDO, and say that G'_0 is a set-inclusion
%%   maximal superset of G' in [G']; then there is no satisfiable
%%   superset G'' of G'_0 as otherwise [G'] would not be minimal; so
%%   G'_0 is maximal satisfiable. Finally, each two maximal
%%   satisfiable sets belog to different members of the PDA: if G'
%%   \neq G'' maximal satisfiable, then neither implies the other as
%%   otherwise G' \cup G'' would be larger satisfiable. We do not
%%   compute the PDO as that may involve entailments due to planning
%%   semantics, eg q entails p if the only way to achieve q is via
%%   achieving p first.



%% conjunctive exclusion properties $\props^{CE}$, $\bigwedge_{a \in A} a
%% \wedge \bigwedge_{b \in B} b$; 3. disjunctive exclusion properties
%% $\props^{DE}$, $\bigwedge_{a \in A} a \wedge \bigvee_{b \in B}
%% g$. (alternative: take elements of intended exclusion implications as
%% properties instead. don't do here as we don't actually compute a PDA
%% for these, there may be other implications we don't see. general: if
%% implications $A \rightarrow B$ are of interest, can use properties $A
%% \wedge \neg B$ where PDA identifies maximal such implications ie where
%% A is weakest and $\neg B$ is strongest. mention this explicitly in
%% this subsec? introduce explusion implications of interest first, then
%% specify encoding of these as plan props and cPDA analysis on those?

%% Prove
%% that the PDA for $\props^{G} \cup \props^{CE}$ and $\props^{G} \cup
%% \props^{DE}$ are polynomially derivable from that (by specifying the
%% required derivation functions $f$ and proving them correct).






%% Initialize $\cal G :+ \emptyset$; assume $\goal = \{g_1, \dots,
%% g_n\}$; call Recursive-PDO-CE($\goal$, $1$):

%% \begin{tabbing}
%% Bo\=ol SystematicWeakening($G'$, $i$) \\
%% \> \textbf{if} $G'$ is solvable in \task\ \textbf{then return \true\ endif}\\
%% \> flag := \true\\
%% \> \textbf{for} \= $j = i \dots n$ where $g_j \in G'$ \textbf{do}\\
%% \> \> \textbf{if} \= not Recursive-PDO-CE($G' \setminus \{g_j\}$, $j+1$) \textbf{then}\\
%% \> \> \> flag := \false\\
%% \> \> \textbf{endif}\\
%% \> \textbf{endfor}\\
%% \> \textbf{if} flag \textbf{then} $\cal G := \cal G \cup \{G'\}$\\
%% \> return \false
%% \end{tabbing}











%%%%%%%%%%%%%%%%%%%%%%%%%%%%%%%%%%%%%%%%%%%%%%%%%%%%%%%%%%%%%%%%%%%%%
%%%%%%%%%%%%%%%%%%%%%%%%%%%%%%%%%%%%%%%%%%%%%%%%%%%%%%%%%%%%%%%%%%%%%
%%%%%%%%%%%%%%%%%%%%%%%%%%%%%%%%%%%%%%%%%%%%%%%%%%%%%%%%%%%%%%%%%%%%%
%%% INITIAL TEXT IN THIS SECTION


%% \subsection{???}

%% \rebecca{start with a concrete example, in the truck domain}

%% \emph{If goal subset A is true at the end of the plan, then at least one element of goal subset B
%% must be false at the end of the plan.}

%% \rebecca{if we are looking fore the minimal sets $A \cup B$ than B always contains 
%% only one element}

%% \begin{definition}[Conjunctive Exclusion]
%% Given a planning task $\Pi = (V,A,c,I,G)$ $(A,B)$ with $A, B \subseteq G$ is a 
%% GFD1 if $\Pi$ with the goal $A \cup B$, i.e., $\bigwedge_{p \in A \cup B} p$
%% is unsolvable.
%% The subsumption relation is given by $\forall A,B,A',B': (A,B) \leq (A',B')$ iff $A \cup B \subseteq 
%% A' \cup B'$
%% \end{definition}

%% \paragraph{Algorithm}
%% All minimal GFD1s can be computed through a search tree that starts at node $N_0$ containing
%% all goal facts $G$ and where each search step on a node $N_i$ tests solvability of 
%% $\Pi_i = (V,A,c,I,N_i)$. If $\Pi_i$ is not solvable, we generate one child 
%% node $N'$ for every subset of $N_i$ obtained by removing one fact.  
%% Upon termination all nodes with only solvable children are the \emph{minimal unsolvable 
%% goal subsets (MUGS)}. (A,B) is a minimal GFD1 iff $A \cup B \in \text{MUGS}$, 
%% $A \cap B = \emptyset$ and $|B| = 1$

%% \rebecca{A is always solvable}

%% \rebecca{add example, for fuel level 5}

%% \subsection{???}

%% \textit{If goal subset A is true at the end of the plan, 
%% then ALL elements of goal subset B must be false at the end of the plan.}\\

%% \begin{definition}[Disjunctive Exclusion]
%% 	Given a planning task $\Pi = (V,A,c,I,G)$ the tuples
%% 	$(A,B)$ with $A,B \subseteq G $ is a GFD2 if 
%% 	$\Pi$ with the goal $\bigwedge_{p \in A} p \wedge (\bigvee_{q \in B} q)$
%% 	is unsolvable. 
%% 	The subsumption relation is given by $\forall A,B,A',B': (A,B) \leq (A',B')$ iff $A \subseteq A'$
%% 	and $B \supseteq B'$.
%% \end{definition}	

%% 	\noindent
%% 	Given all minimal GFD1s for $\Pi$ all minimal GFD2s can be derived according 
%% 	to the following relation.

%% 	\rebecca{proof}

%% 	\vspace{-0.3cm}
%% 	\begin{align*}
%% 		GFD2 &:= \{(A,B) | 
%% 				\exists P \in GFD1:(
%% 				   A \subseteq P \wedge |P \setminus A | = 1 \wedge\\
%% 				   &\forall P' \in PPD1:
%% 					  A \subseteq P' \rightarrow P' \setminus A \subseteq B
%% 				)
%% 			 \}
%% 	\end{align*}

%% \paragraph{Algorithm}
%% From the minimal GFD1s we get a set:
%% 	$D = \{(A,B) | \exists P \in GFD1s, p \in P: A = P \setminus p \wedge
%% 	B = \{p\}\}$

%% These GFD2s are not necessary optimal, the B could be larger.

%% \rebecca{find a name for A and B}

%% \noindent
%% To get the maximal set of goals which can not be achieved if we achieve A, 
%% you have to merge all B's which belong to an A' which is a subset of A. 

%% \begin{align*}
%% 	(A, \bigcup_{(A', B') \in \{(A'', B'') \in D | A' \subseteq A\}} B' \cup B)
%% \end{align*}

%% \rebecca{to many ticks}

%% \rebecca{add example, from GFD1 to GFD2}

%% \noindent
%% If the planning task is not solvable for a goal fact at all, you can add 
%% this goal fact to all B's.\\
