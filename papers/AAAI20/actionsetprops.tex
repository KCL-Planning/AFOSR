\section{Action-Set Properties}
\label{actionsetprops}

%% \joerg{0.5--1 page: similar to ijcai version; focus on action-set
%%   props, but do briefly mention more general compolations like LTL;
%%   nomytsrey example in full here, with props and MUGS as in ijcai,
%%   elaborate/revise text for local vs global explanations}

%% \joerg{ijcai text snippets:}

The analyses just described can, in principle, be used to analyze
dependencies between arbitrary plan properties, so long as these can
be compiled into goal facts. Given the well-known power of compilation
in planning languages
(\eg\ \cite{gazen:knoblock:ecp-97,nebel:jair-00,edelkamp:icaps-06,palacios:geffner:jair-09,baier:etal:ai-09}),
there is large potential in this idea. As an example, here we consider
what we refer to as action-set properties:

% Joerg: not using atomic/composed here as that would link back in to
% the discussionbefore where we specify onl the atomic props and
% analyze the compiosed ones; but this is niot the case here/in our
% experiments: instead we explicitly list a small set of composed
% properties to be analyzed.
%
\begin{definition}[Action-Set Properties]\label{def:action-set-properties}
Let $\task =
(\vars,\allowbreak\acts,\allowbreak\cost,\allowbreak\init,\allowbreak\goalhard,\allowbreak\goalsoft,\allowbreak\costbound)$
be an OSP task, \plans\ its set of plans, and $\acts_1, \dots, \acts_n
\subseteq \acts$.

An \defined{action-set property} for \task\ and $\acts_1, \dots,
\acts_n$ is a function $\prop_\phi : \plans \mapsto \{\true, \false\}$
where $\phi$ is a propositional formula over the atoms $\acts_1,
\dots, \acts_n$, and $\prop_\phi(\plan) = \true$ iff $\phi$ evaluates
to \true\ under the truth value assignment where $\acts_i$ is
\true\ iff $\plan$ contains at least one action from $\acts_i$.
\end{definition}

As before, we identify action-set properties $\prop_\phi$ with the
characterizing formulas $\phi$. Arguably, action-set properties are
practically relevant. They allow to express things like ``objective x
is covered by satellite y'', %
% (if this is desirable but could be traded
%against other soft preferences) 
%
``route x is not used'',
% (if that route is problematic, \eg\ due to frequent traffic issues),
%
``passengers x and y ride in the same vehicle'',
% (if that is desirable), 
%
if these are desirable but could be traded against other soft
preferences. At the same time, the simple syntax of action-set
properties lends itself to effective compilation, as follows.

Given \task, \plans, and $\acts_1, \dots, \acts_n$ as in
Definition~\ref{def:action-set-properties}, to obtain a compiled task
$\task'$
\begin{itemize}
\item introduce Boolean flags $\mathit{isUsed}_i$ that are initially
  \false\ and set to \true\ by any action from $\acts_i$;
\item introduce formula-evaluation state variables and actions
  evaluating each $\prop_\phi$ based on these (following
  \cite{gazen:knoblock:ecp-97,nebel:jair-00}), setting Boolean flags
  $\mathit{isTrue}_\phi$ storing the outcome values;
\item introduce a separate 1.\ \emph{planning phase}
  vs.\ 2.\ \emph{formula-evaluation phase}, and a switch action
  allowing to go from 1.\ to 2.\ if \goalhard\ is satisfied.
%
% joerg: the latter isnt strictly needed as the evaluation ophase
% cannot meddle with the goal anyway; but it makes the following
% argument simpler.
%
\end{itemize}
Then the planning-phase prefixes in $\task'$ are in one-to-one
correspondence with \plans, and given such a prefix \plan\ the
evaluation phase in $\task'$ can achieve $\mathit{isTrue}_\phi$ iff
$\prop_\phi(\plan) = \true$.

Now say that we want to analyze the dependencies across a given set
$\props$ of action-set properties (\eg\ possible undesirable
consequences of not using route X). We are given \task, \plans, and
$\props$; we want to identify dependencies of the form
$\entails{\plans}{\bigwedge_{\phi \in A} \phi}{\neg \bigwedge_{\psi
    \in B} \psi}$. With the above, this can be done by considering
$\task'$ with soft goals $\{\mathit{isTrue}_\phi \mid \phi \in
\props\}$, and identifying each $\mathit{isTrue}_\phi$ with $\phi$ in
the outcome.

In the NoMystery domain, for example, the following three kinds of
action-set properties can make sense: \emph{uses $T_i$ $(L_x,L_y)$}
(truck $T_i$ drives at least once from $L_x$ to $L_y$ or vice versa);
\emph{doesn't use $T_i$ $(L_x,L_y)$} (the opposite); \emph{same truck
  $P_x$ $P_y$} (both packages are delivered by the same truck). In our
example task, say we consider the concrete properties 1.\ uses $T_0$
$(L_2,L_3)$; 2.\ same truck $P_1$ $P_2$; 3.\ uses $T_0$ $(L_4,L_3)$;
4.\ same truck $P_2$ $P_0$; 5.\ doesn't use $T_0$ $(L_0,L_5)$;
6.\ uses $T_1$ $(L_5,L_4)$. Say initial fuel is $16$ for $T_0$ and $7$
for $T_1$. Say we fix the package destinations as hard
goals. Computing the MUGS over the soft goals representing the six
action-set properties, it turns out there are seven minimal unsolvable
subsets of these properties, each of size three. A user could, for
example, ask ``Why do you not avoid the road $L_0-L_5$ (which has a
lot of traffic at the moment)?'', translating into the question ``Why
do you not achieve property 5?''. One of the non-rhs-dominated
entailments of property 5\ is $\neg($property 2 $\wedge$ property
4$)$, so that the generated local explanation would include the
statement ``Because that would necessitate to either not use the same
truck for $P_1$ and $P_2$, or not use the same truck $P_2$ and
$P_0$''. In other words: ``Because if you don't use that road, then
you cannot deliver all packages with a single truck.''

%% \joerg{note/discuss in action-set props section: + plan props beyond
%%   goals: more general than OSP, can be oversubscribed even if all
%%   goals achievable in conjunction.}
%
% Joerg: ah well. not clear. after all, why would the user analyze
% dependencies between these properties? presumablym the best would be
% if they (ie their desired pos/neg versions) could all be set as
% goals. so in a sense this is still OSP, in a setting where some of
% the goals are action-set properties.
%
%% Note that, in this setup, our analysis actually goes beyond OSP as
%% traditionally conceived. In the example task just described, all goals
%% -- the package locations -- can be achieved in conjunction. The
%% analysis is not used to identify goal conflicts, but 

We remark that, in a preliminary exploration, we implemented a
compilation for LTL plan properties based on previous work
\cite{edelkamp:icaps-06,baier:etal:ai-09}. Our results are promising,
yet indicate that algorithmic optimizations are needed to obtain
reasonable performance for complex properties. This remains a topic
for future work.





