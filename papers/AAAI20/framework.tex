\section{Goal-Property Dependencies in OSP}
\label{framework}

%% \joerg{1--1.5 page: previous generic framework, now instantiated to
%%   OSP; illustrate defs with goal-fact dependencies in nomystery
%%   example}

%% \joerg{ijcai text snippets:}

We next spell out our framework for plan properties, entailment
relations between them, and the forms of explanations we aim at. We do
so in FDR-based OSP as defined above, but in principle our definitions
are generic and can be instantiated for arbitrary plan properties and
planning frameworks. 
%
%We will briefly discuss the generic view in the text.



\subsection{Plan Properties and Property Entailment}

The plan properties we consider here are formulas over the soft goals:

\begin{definition}[Plan Properties]
\label{def:osp-plan-properties}
Let $\task =
(\vars,\allowbreak\acts,\allowbreak\cost,\allowbreak\init,\allowbreak\goalhard,\allowbreak\goalsoft,\allowbreak\costbound)$
be an OSP task, and \plans\ its set of plans.

A \defined{plan property} is a function $\prop_\phi : \plans \mapsto
\{\true, \false\}$ where $\phi$ is a propositional formula over the
atoms \goalsoft, and $\prop_\phi(\plan) = \true$ iff $\phi$ evaluates
to \true\ under the truth value assignment where $g \in \goalsoft$ is
\true\ iff $g \in \init\apply{\plan}$.
%
$\prop_\phi$ is \defined{conjunctive} if $\phi$ has the form
$\bigwedge_{a \in A} a\allowbreak$ or $\neg \bigwedge_{b \in B} b$.
\end{definition}

Our current analyses consider conjunctive plan properties only. We
will identify $\prop_\phi$ with the characterizing formula $\phi$.

In general, a plan property can be any function mapping a task and an
action sequence to a Boolean value. Examples are temporal plan
trajectory constraints, action-set properties (formulas over subsets
of actions touched/not touched by the plan), deadlines, bounds on
resource consumption, etc. To the extent that such more general
properties can be compiled into goal facts, conjunctive plan
properties can be used to analyze their dependencies. We will explore
this possibility here with a compilation for action-set properties.

We formalize plan-property dependencies as entailment in the space of
plans \plans:

\begin{definition}[\plans-Entailment]
\label{def:pi-entailment}
Let $\task =
(\vars,\allowbreak\acts,\allowbreak\cost,\allowbreak\init,\allowbreak\goalhard,\allowbreak\goalsoft,\allowbreak\costbound)$
be an OSP task, and \plans\ its set of plans.

We say that $\plan \in \plans$ \defined{satisfies} a plan property
$\phi$, written $\plan \models \phi$, if $\prop_\phi(\plan) =
\true$. We denote by $\modelsof{\plans}{\phi} := \{\plan \mid \plan
\in \plans, \plan \models \phi\}$ the subset of plans that satisfy
$\phi$.
%
%% We say that \phi\ is \defined{satisfiable} in \plans\ if
%% $\modelsof{\plans}{\phi} \neq \emptyset$.
%
We say that $\phi$ \defined{\plans-entails} $\psi$, written
$\entails{\plans}{\phi}{\psi}$, if $\modelsof{\plans}{\phi} \subseteq
\modelsof{\plans}{\psi}$.
%
\end{definition}

This definition views $\plans$ in the role traditionally taken by a
knowledge base, identifying a set of ``possible worlds'' within which
entailment over plan properties is considered.
%
Observe that, given this, \plans-entailment is more than standard
entailment: $\phi \Rightarrow \psi$ implies that
$\entails{\plans}{\phi}{\psi}$, but not vice versa. \plans-entailment
captures entailments specific to the ``knowledge base'' \plans. For
example, in our illustrative NoMystery task, say that all goals are
soft, $T_0$ has initial fuel supply $12$, and $T_1$ has no fuel. Then
$\entails{\plans}{at(P_0,L_4)}{\neg (at(P_1,L_3) \wedge at(P_2,L_5))}$
because, if we achieve the goal for $P_0$, there is insufficient fuel
to transport both other packages. If we set the initial fuel supply of
$T_0$ to $16$, on the other hand, then the knowledge base changes --
\plans\ becomes more permissive -- and that entailment no longer
holds.
%
%$\notentails{\plans}{at(P_0,L_4)}{\neg (at(P_1,L_3) \wedge
%  at(P_2,L_5))}$.

Note that the definition of \plans-entailment is agnostic to the
specification of \plans. The definition applies unchanged to arbitrary
planning frameworks and plan sets \plans. Even an explicitly listed
\plans\ could make sense in some applications.

Our primary focus here will be on goal exclusions:

\begin{definition}[Goal Exclusion]
Let $\task =
(\vars,\allowbreak\acts,\allowbreak\cost,\allowbreak\init,\allowbreak\goalhard,\allowbreak\goalsoft,\allowbreak\costbound)$
be an OSP task, and \plans\ its set of plans.

A \defined{goal exclusion} is an entailment of the form
$\entails{\plans}{\bigwedge_{a \in A} a}{\neg \bigwedge_{b \in B}
  b}$. The entailment is \defined{non-dominated} if there is no pair
$(A',B')$ where $A' \subseteq A$, $B' \subseteq B$, $(A',B') \neq
(A,B)$, and $\entails{\plans}{\bigwedge_{a \in A'} a}{\neg
  \bigwedge_{b \in B'} b}$. The entailment is \defined{non-dominated
  given $A$} if there is no $B'$ where $B' \subsetneq B$ and
$\entails{\plans}{\bigwedge_{a \in A'} a}{\neg \bigwedge_{b \in B'}
  b}$.
\end{definition}

Goal exclusions are of interest in OSP as they reflect the detailed
(soft-)goal trade-offs in the user's quest for a good plan. A
non-dominated goal exclusion has subset-minimal $A$ and $B$. This
dominates entailments with larger $A$ and/or $B$ as it has a weaker
left-hand side $A$ (smaller conjunction) entailing a stronger
right-hand side $\neg B$ (smaller disjunction, after moving the
negation inside). If $A$ is fixed, then only $B$ needs be minimal. We
will use non-dominated entailments to give more compact explanations.

% Note: non-dominated goal exclusions have disjoint A,B: if g in A cap
% B, then A ==> B \ {g} is still valid as g is true whenever A is.


\subsection{Local and Global Explanations}

As previously hinted, we propose to employ the concept of
plan-property entailment for the purpose of giving local and global
explanations of the plan space \plans. 

For local explanations, we assume a user question of the form ``Why do
you not achieve property $\phi$?'', which we answer with the set of
plan properties $\psi$ entailed by $\phi$:

\begin{definition}[Local Explanation (LE)]
\label{def:local-explanation}
Let $\task =
(\vars,\allowbreak\acts,\allowbreak\cost,\allowbreak\init,\allowbreak\goalhard,\allowbreak\goalsoft,\allowbreak\costbound)$
be an OSP task, and \plans\ its set of plans.

For a plan property $\phi$, the \defined{local explanation (LE)} for
$\phi$ is the set $\{\psi \mid \entails{\plans}{\phi}{\psi}\}$ of plan
properties \plans-entailed by $\phi$.
%
For $\phi = \bigwedge_{a \in A} a$, the \defined{goal-exclusion LE}
for $\phi$ is $\{\psi \mid \psi = \neg \bigwedge_{b \in B} b,
\entails{\plans}{\phi}{\psi} \text{ is non-dominated given } A\}$.
\end{definition}

Such an answer makes sense if the entailed properties $\psi$ are
undesirable. This is the case, in particular, for goal-exclusion local
explanations. In our example, the answer to $\phi = at(P_0,L_4)$ ``Why
do you not achieve the goal for $P_0$?'' would be $\psi = \neg
(at(P_1,L_3) \wedge at(P_2,L_5))$ ``Because that would necessitate to
forego the goal for either $P_1$ or $P_2$''.

From a general perspective, plan properties here serve as an
abstraction level at which to explain \plans\ to a user. The
underlying assumption is that \plans\ is large and/or the mechanisms
that generate \plans\ are complex, so that an abstract form of
explanation is needed. The abstraction level can be controlled through
the number and granularity of plan properties. Given this, while here
we simply talk about all formulas over soft-goal facts, it can make
sense to instead fix a more specific set \props\ of plan properties
the user has a vested interest in (raising the new sub-problem how to
choose \props).

If the user question ``Why do you not achieve property $\phi$?''
refers to a given plan candidate \plan, then it makes sense to return
only those entailed $\psi$ where $\plan \not\models \psi$, \ie,
currently false plan properties that would become true when enforcing
$\phi$. That answer is easy to obtain from the more exhaustive answer
as per Definition~\ref{def:local-explanation}. We will consider only
the latter in our experiments, avoiding the need for, and bias by, a
particular method for generating candidate plans \plan. \joerg{or
  should we actually run experiments with local questions from
  concrete plans?}

Our notion of global explanation arises directly from the above.
Instead of showing the implications of one specific plan property, one
can show all plan-property implications:

\begin{definition}[Global Explanation (GE)]
Let $\task =
(\vars,\allowbreak\acts,\allowbreak\cost,\allowbreak\init,\allowbreak\goalhard,\allowbreak\goalsoft,\allowbreak\costbound)$
be an OSP task, and \plans\ its set of plans.

Say that $\phi$ and $\psi$ are \defined{\plans-equivalent} if
$\modelsof{\plans}{\phi} = \modelsof{\plans}{\psi}$, and denote by
$\equiv{\plans}{\phi}$ the \plans-equivalence class of $\phi$.
%
The \defined{global explanation (GE)} for \task\ is the partial order
$\pdo{\plans}$ over the classes $\equiv{\plans}{\phi}$ where
$\equiv{\plans}{\phi} \pdo{\plans} \equiv{\plans}{\psi}$ iff
$\entails{\plans}{\phi}{\psi}$.
%
%% A \defined{concrete GE (cGE)} replaces each equivalence class
%% $\equiv{\plans}{\prop}$ with exactly one $\prop \in
%% \equiv{\plans}{\prop}$.
\end{definition}

The only somewhat non-obvious design decision here is to group plan
properties into equivalence classes, to reduce the size and redundancy
of the GE. That said, while this definition makes sense at the formal
level, it has several practicality issues. Computationally, with many
plan properties (here: all propositional formulas over \goalsoft),
many equivalence classes (here: subsets of satisfying $\plan \in
\plans$), and with the complexity of deciding even a single entailment
relation (here: \pspace-complete as it subsumes plan existence
\cite{bylander:ai-94}), producing a GE is likely to be
infeasible. Conceptually, the size of a GE makes it more than
questionable whether the GE can be processed by a human user.

It therefore makes sense to focus on more limited forms of GEs, and to
find ways to represent these more compactly. Here, we focus on
non-dominated goal exclusions:

\begin{definition}[Goal-Exclusion GE]
Let $\task =
(\vars,\allowbreak\acts,\allowbreak\cost,\allowbreak\init,\allowbreak\goalhard,\allowbreak\goalsoft,\allowbreak\costbound)$
be an OSP task, and \plans\ its set of plans.

The \defined{goal-exclusion GE} for \task\ is the partial order
$\pdo{\plans}$ over conjunctive plan properties where $\phi
\pdo{\plans} \psi$ iff $\entails{\plans}{\phi}{\psi}$ is a
non-dominated goal exclusion.
%
%% $\phi = \bigwedge_{a \in A} a$, $\psi = \neg \bigwedge_{b \in B}
%% b$, $\entails{\plans}{\bigwedge_{a \in A} a}{\neg \bigwedge_{b \in
%% B} b}$, and there does not exist $(A',B')$ such that $A' \subseteq
%% A$, $B' \subseteq B$, $(A',B') \neq (A,B)$, and
%% $\entails{\plans}{\bigwedge_{a \in A'} a}{\neg \bigwedge_{b \in B'}
%% b}$.
\end{definition}

Here we focus exclusively on goal exclusions, ignoring all other
\plans-entailments. Plan properties can then no longer entail each
other, so that the equivalence classes trivialize.
%
% Joerg: now introduced above
%
%% In words, (1) we consider only entailments where achieving a goal
%% conjunction $A$ necessitates to forego at least one element of a goal
%% conjunction $B$; and (2) we show only those entailments where $A$ and
%% $B$ are subset-minimal. For (1), the motivation is that users are
%% likely to be particularly interested in this form of dependency, as it
%% reflects the detailed (soft-)goal trade-offs in the user's quest for a
%% good plan. Thanks to the exclusive focus on (1), plan properties can
%% no longer entail each other, so that the equivalence classes
%% trivialize. For (2), entailments $\entails{\plans}{\bigwedge_{a \in A}
%%   a}{\neg \bigwedge_{b \in B} b}$ with minimal $A$ and $B$ dominate
%% non-minimal ones, with a weaker left-hand side $A$ (smaller
%% conjunction) entailing a stronger right-hand side $\neg B$ (smaller
%% disjunction, after moving the negation inside). 
%
% Joerg: just leave out, probably better to not ``wake sleeping
% hounds'', should get away with compactness being a good thing
%
%% We assume here that, in an actual user interface, this form of
%% domination is natural to, and cognitively easy for, human users when
%% processing the goal-exclusion GE; this remains to be verified through
%% user studies in future work. For now, o
%
% Joerg: now introduced above
%
Our results on benchmarks suggest that the goal-exclusion GE can be
feasible to compute, and often produces answers sufficiently compact
for human inspection.

\joerg{can we give an example goal-exclusion GE from nomystery here?
  with goals only ie no action-set properties? we're free to set fuel
  levens small, to keep this feasible. e.g. with T0 12 and T1 0 as
  above?}

