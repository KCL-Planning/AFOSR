\section{Contrastive Explanation of Plan Space through Plan-Space Dependencies}
\label{xpp}

\joerg{todo Joerg then Dan: spell out, cite literature.}

Covering recent work by the lecturers towards this end, we set up a
formal framework allowing to provide such answers in a systematic way;
we instantiate that framework with the special case of questions about
goal-conjunction achievability in oversubscription planning (where not
all goals can be achieved and thus a trade-off needs to be found); and
we discuss the compilation of more powerful question languages into
that special case. Linking to the state of the art in research on
effective planning methods, we briefly cover recent techniques for
nogood learning in state space search, as a key enabler to efficiency
in the suggested analyses.

\joerg{relevant snippets from IJCAI intro:}


Our approach here is most closely related to, and inspired by, the
challenges set by David Smith in what he called \emph{Planning as an
  Iterative Process} \cite{smith:aaai-12}. This introduces a broad
vision of users interacting with the planning process, to elaborate
their preferences, to understand the planner's decision rationales, to
interactively produce the final plan. 



Here we address the same kind of explanation problem, but we replace
the \emph{existential} answer generating a single alternative plan
$\plan'$ with a \emph{universal} answer determining shared properties
of \emph{all} possible such alternatives. In this way, the analysis we
propose aims at explaining the space of possible plans, rather than
pointing out examples.

Our proposed analysis works at the level of \defined{plan properties}:
Boolean functions on plans that capture aspects of plans the user
cares about (whether or not the plan starts with a particular action,
whether or not a particular soft objective is satisfied, etc). We
assume that the set \props\ of plan properties of interest is given as
part of the input.\footnote{An interesting yet challenging question
  for future work is how one can automatically identify relevant plan
  properties.} Our analysis then determines the \defined{dependencies}
across plan properties, \ie, \defined{plan-space entailments} which we
define as follows. The ``plan space'' is the set \plans\ of candidate
plans to be considered (canonically, the set of plans for an input
planning task). A plan property $p$ \defined{entails} another property
$p'$ in \plans\ if every $\plan \in \plans$ that satisfies $p$ also
satisfies $p'$. A user question ``Why does the current plan
\plan\ satisfy $p$ rather than $q$?'' can then be answered in terms of
the properties $q'$ not true in \plan\ but entailed by $q$: things
that will \emph{necessarily} change when satisfying $q$.

Our approach also supports iterative planning, along the lines
suggested by Smith \cite{smith:aaai-12}. Given a current plan
$\plan \in \plans$ and a user question ``Why achieve $p$ rather than
$q$?'', if the consequences of $q$ are tolerable to the user, she may
choose to enforce $q$, gradually narrowing the plan-candidate space
\plans.
%
% Joerg: Text highlighting enforced vs analyzed; simplified to save space
%
%% Observe that \plans\ itself may be viewed as being defined through a
%% set of \emph{enforced} plan properties (like achieving a set of goal
%% facts). Such enforced properties are then distinguished from the
%% \emph{analyzed} properties whose dependencies we wish to identify.
%% %
%% These two classes of properties can play different roles depending on
%% the application scenario. In contrastive explanations as outlined
%% above, the enforced properties are fixed. However, our approach also
%% supports an iterative planning process for oversubscription planning
%% (\eg\ \cite{smith:icaps-04,domshlak:mirkis:jair-15}), along the lines
%% suggested by Smith \cite{smith:aaai-12}. The analyzed properties
%% then capture ``soft goals'', while the enforced properties capture
%% ``hard goals''. Given a currently suggested plan $\plan \in \plans$
%% and a user question ``Why $p$ rather than $q$?'', if the consequences
%% of analyzed property $q$ are tolerable to the user, she may choose to
%% enforce $q$, gradually narrowing the plan-candidate space \plans.
%% %
%% % Joerg: shortetened to save space
%% %
%% %% Observe that \plans\ itself may be naturally defined as the set of
%% %% plans satisfying a given set of plan properties. For example, these
%% %% properties may ask to achieve a set of goal facts. In such a setting,
%% %% it makes sense to distinguish between \defined{enforced} plan
%% %% properties, that induce \plans; vs.\ \defined{analyzed} plan
%% %% properties, whose entailment relations within \plans\ we wish to
%% %% identify. 
%% %
%% %% Enforced vs.\ analyzed properties can play different roles depending
%% %% on the application scenario. In classical planning, the analyzed
%% %% properties may capture relevant plan phenomena in a user quest to
%% %% understand causal relationships between these phenomena
%% %% (\eg\ dependencies between action subsets used). Another use case is a
%% %% user quest to identify a preferred plan in oversubscription planning
%% %% (\eg\ \cite{smith:icaps-04,domshlak:mirkis:jair-15}), where the
%% %% analyzed properties capture ``soft goals'', and the enforced
%% %% properties are ``hard goals''. The analysis then identifies the
%% %% precise trade-offs between the soft goals.
%% %% %
%% %% % Joerg: too complicated/more distracting than helpful
%% %% %
%% %% %% ; one may include additional analyzed properties aimed at identifying
%% %% %% the causes behind these trade-offs.
%% %% %
%% %% In that setting, our approach also supports an iterative planning
%% %% process along the lines suggested by Smith \cite{smith:aaai-12}:
%% %% given a currently suggested plan $\plan \in \plans$ and a user
%% %% question ``Why $p$ rather than $q$?'', if the consequences of analyzed
%% %% property $q$ are tolerable to the user, she may choose to enforce $q$,
%% %% gradually narrowing the candidate space \plans.

We remark that our approach can be viewed as an intermediate between
domain/task analysis (\eg\ \cite{fox:long:jair-98}), which our
approach generalizes; and model checking applied to planning models,
which our approach is an instance of (related to
\cite{vaquero:etal:keq-13}). 
%
% Joerg: Detailed discussion of domain analysis and model checking;
% simplified to save space/not be distracting here.
%
%% Another alternate view of our approach is as a form of domain analysis
%% (actually: task analysis), identifying particular properties of plan
%% space ahead of time. Indeed, various popular task analyses can be cast
%% as instances of our framework. A fact pair $(p,q)$ is mutually
%% exclusive \cite{blum:furst:ai-97} iff $p$-true-at-end entails $\neg
%% q$-true-at-end in the space of all applicable action sequences; a fact
%% $p$ is a landmark \cite{hoffmann:etal:jair-04} iff $\true$ entails
%% $p$-true-at-some-point; other examples presumably exist. From this
%% point of view, we generalize previous concepts to a broader
%% perspective aimed at addressing arbitrary user questions. At the same
%% time, our approach itself can be viewed as an instance of model
%% checking of planning models \cite{clarke:etal:01},\footnote{There has
%%   been little work on this subject; Vaquero et
%%   al.\ \cite{vaquero:etal:keq-13} use Petri nets to capture and
%%   check dynamic aspects of planning models in itSIMPLE.}
%% systematically checking all entailments between plan properties. Again
%% the value of our framework lies in its suitability for XAIP (plus
%% computational gains from considering all \props\ dependencies in
%% unison rather than running individual entailment checks).

