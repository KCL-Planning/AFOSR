\section{Contrastive Explanation of Plan Space through Plan-Space Dependencies}
\label{xpp}

We finally consider a line of work, conducted by the authors, starting
from the idea to answer questions ``Why does $\plan$ start with action
$A$ rather than $B$?'' by generating a new plan $\plan'$ starting with
$B$ and highlighting undesirable properties of $\plan'$. A key
weakness of this approach is that there may be differences between
\plan\ and $\plan'$ unrelated to the use of $A$ vs.\ $B$. Many
comparison aspects (\eg\ which other actions are used, or which
``soft'' objectives are satisfied) may be affected by arbitrary
decisions in the planner's search. Therefore, the idea is to replace
the \emph{existential} answer generating a single alternative plan
$\plan'$ with a \emph{universal} answer pertaining to \emph{all}
possible such alternatives.

This can be done at the level of \emph{plan properties}: Boolean
functions on plans that capture aspects of plans the user cares about
(whether or not the plan starts with a particular action, whether or
not a particular soft objective is satisfied, etc). Given a set of
plan properties, one can determine dependencies across these
properties, \ie, plan-space entailments: a plan property $p$ entails
another property $p'$ if every plan that satisfies $p$ also satisfies
$p'$. A user question ``Why does the current plan \plan\ satisfy $p$
rather than $q$?'' can then be answered in terms of the properties
$q'$ not true in \plan\ but entailed by $q$: things that will
necessarily change when satisfying $q$.

We put forward, and explain in the lecture, a generic framework for
this kind of analysis, as well an instantiation and experiments in the
context of oversubscription planning
\cite{smith:icaps-04,domshlak:mirkis:jair-15} where resources are
insufficient to achieve all goals, and plan properties of obvious
interest are those goals achieved by a plan. A first paper on this
approach is published at XAIP'19 and serves as reference for the
reader interested in details \cite{eifler:etal:xaip-19}.
